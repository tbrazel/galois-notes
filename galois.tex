\documentclass[11pt]{amsart}
\usepackage[page,toc,titletoc,title]{appendix}

% Bibliography stuff
\usepackage[doi=false,isbn=false,url=false,style=alphabetic]{biblatex}
\bibliography{citations.bib}


% Packages
\usepackage{amsmath,amssymb,amsthm,amsfonts,thmtools}
\usepackage[shortlabels]{enumitem}[cleveref]
\usepackage{amsfonts}
\usepackage[margin=1in]{geometry}
\usepackage{float}

% Avoid footnote patch error
\usepackage[final,nopatch=footnote]{microtype}

\RequirePackage{color}
\RequirePackage{tikz}
\RequirePackage{tikz-cd}

% For arrows
\RequirePackage{mathtools}

% For script letters
\RequirePackage{mathrsfs}

% For boxes around cheatsheets
\usepackage{mdframed}
\mdfdefinestyle{cheatsheet}{%
    linecolor=black,
    outerlinewidth=2pt,
    roundcorner=20pt,
    innertopmargin=4pt,
    innerbottommargin=4pt,
    innerrightmargin=40pt,
    innerleftmargin=40pt,
    leftmargin = 100pt,
    rightmargin = 100pt
    backgroundcolor=gray!50!white}

% Parindent/parskip
% \setlength{\parindent}{0pt}

% Custom color names
\usepackage{xcolor}
\definecolor{darkgreen}{rgb}{0,0.30,0}
\definecolor{darkred}{rgb}{0.75,0,0}
\definecolor{darkblue}{rgb}{0,0,0.6} 
\definecolor{custompurple}{RGB}{62, 34, 127}


% Citation colors
\def\customcitecolor{darkred}
\def\customlinkcolor{darkred}

% Hyperref settings
\usepackage[%
    colorlinks,
    citecolor=\customcitecolor,%
    linkcolor=\customlinkcolor,%
    urlcolor=\customlinkcolor%
]{hyperref}

% Removes vertical spacing around aligned environments
\usepackage{etoolbox}
\newcommand{\zerodisplayskips}{%
  \setlength{\abovedisplayskip}{2pt}%
  \setlength{\belowdisplayskip}{2pt}%
  \setlength{\abovedisplayshortskip}{0pt}%
  \setlength{\belowdisplayshortskip}{0pt}}
\appto{\normalsize}{\zerodisplayskips}
\appto{\small}{\zerodisplayskips}
\appto{\footnotesize}{\zerodisplayskips}

% Removes spacing around enumerate/itemize environments
\usepackage{enumitem}
\usepackage{setspace}
\setlist[enumerate,1]{leftmargin=1cm}
\setlist[enumerate,2]{leftmargin=2cm}
\setlist[itemize,1]{leftmargin=0.5cm}
\setlist[itemize,2]{leftmargin=2cm}
\setlist{nosep} % or \setlist{noitemsep} to leave space around whole list

% Section headings
\newcommand{\sectionheader}{Lecture~\thesection:~}

% Course info
\newcommand{\theinstructor}{Thomas Brazelton}
\newcommand{\thecoursetitle}{Galois groups in enumerative geometry}
\newcommand{\thetitle}{\thecoursetitle}
\newcommand{\theterm}{Spring 2025}

% Title page
\title{\MakeUppercase{\thetitle} \\ \theterm}
\author{\theinstructor}
%\date{\theterm}

% Header
\usepackage{fancyhdr}
\pagestyle{fancy}
\fancyhf{}
\fancyhead[L]{\small\itshape\thecoursetitle}
\fancyhead[R]{\small\itshape\theterm}
\setlength{\headheight}{12.0pt}
\setlength{\footskip}{13.0pt}

% Last hacky commands
\newcommand{\todo}{\color{red}\text{todo:}\, \color{black}}
\let\minus\smallsetminus
\renewcommand{\labelitemi}{$\triangleright$}
\let\emptyset\varnothing


% Pushout, pullback
\providecommand{\po}{\arrow[ul,phantom,"\ulcorner" very near start]}
\providecommand{\pb}{\arrow[dr,phantom,"\lrcorner" very near start]}

% Overset to and from
\providecommand{\xto}[1]{\xrightarrow{#1}}
\providecommand{\from}{\leftarrow}
\providecommand{\xfrom}[1]{\overset{#1}{\leftarrow}}

% Backwards verion of mapsto
\providecommand{\mapsfrom}{\mathrel{\reflectbox{\ensuremath{\mapsto}}}}
\providecommand{\longmapsfrom}{\mathrel{\reflectbox{\ensuremath{\longmapsto}}}}

% Hook arrows
\providecommand{\hookto}{\xhookrightarrow{}}
\providecommand{\xhookto}[1]{\overset{#1}{\hookrightarrow}}
\providecommand{\hookfrom}{\xhookleftarrow{}}
\providecommand{\xhookfrom}[1]{\xhookleftarrow{#1}}

% Two-headed arrows
\providecommand{\tto}{\twoheadrightarrow}
\providecommand{\xtto}[1]{\overset{#1}{\twoheadrightarrow}}
\providecommand{\ffrom}{\twoheadleftarrow}
\providecommand{\xffrom}[1]{\overset{#1}{\ffrom}}

% For superimposing in order to get closed and open immersion arrows
\makeatletter
\providecommand{\superimpose}[2]{%
  {\ooalign{$#1\@firstoftwo#2$\cr\hfil$#1\@secondoftwo#2$\hfil\cr}}}
\makeatother
\providecommand{\smallslash}{\mbox{\tiny/}}

% Closed and open hook arrows
\providecommand{\clhookto}{\mathrel{\raisebox{0.1em}{$\mathrel{\mathpalette\superimpose{{\hspace{0.1cm}\vspace{0.1em}\smallslash}{\hookrightarrow}}}$}}}
\providecommand{\xclhook}[1]{\overset{#1}{\clhook}}
\providecommand{\clhookfrom}{\mathrel{\raisebox{0.1em}{$\mathrel{\mathpalette\superimpose{{\hspace{0.1cm}\vspace{0.1em}\smallslash}{\hookleftarrow}}}$}}}
\providecommand{\ohookto}{\mathrel{\raisebox{0.03em}{$\mathrel{\mathpalette\superimpose{{\hspace{0.1cm}\vspace{0.03em}\mbox{\small$\circ$}}{\hookrightarrow}}}$}}}
\providecommand{\ohookfrom}{\mathrel{\raisebox{0.03em}{$\mathrel{\mathpalette\superimpose{{\hspace{0.1cm}\vspace{0.03em}\mbox{\small$\circ$}}{\hookleftarrow}}}$}}}

% Arrows with tails
\providecommand{\cofto}{\rightarrowtail}
\providecommand{\coffrom}{\leftarrowtail}
\providecommand{\xcofto}[1]{\overset{#1}{\cofto}}
\providecommand{\xcoffrom}[1]{\overset{#1}{\coffrom}}

% Dashed arrows
\providecommand{\dashto}{\dashrightarrow}
\providecommand{\dashfrom}{\dashleftarrow}

% better spacing colon for right adjoints
\newcommand\noloc{%
   \nobreak
   \mspace{6mu plus 1mu}
   {:}
   \nonscript\mkern-\thinmuskip
   \mathpunct{}
   \mspace{2mu}
}

% Squiggle arrows
\providecommand{\sqto}{\rightsquigarrow}
\providecommand{\sqfrom}{\mathrel{\reflectbox{\ensuremath{\sqto}}}}


%%%%%%%%%%%%%
% Text commands
\providecommand{\ab}{\mathrm{ab}}
\providecommand{\alg}{\mathrm{alg}}
\providecommand{\an}{\mathrm{an}}
\providecommand{\ann}{\mathrm{ann}}
\providecommand{\Aut}{\mathrm{Aut}}
\providecommand{\BG}{\mathrm{BG}}
\providecommand{\BGL}{\mathrm{BGL}}
\providecommand{\Bl}{\mathrm{Bl}}
\providecommand{\BO}{\mathrm{BO}}
\providecommand{\BP}{\mathrm{BP}}
\providecommand{\BSL}{\mathrm{BSL}}
\providecommand{\BSO}{\mathrm{BSO}}
\providecommand{\BSp}{\mathrm{BSp}}
\providecommand{\BSU}{\mathrm{BSU}}
\providecommand{\BU}{\mathrm{BU}}
\providecommand{\can}{\mathrm{can}}
\providecommand{\cd}{\mathrm{cd}}
\providecommand{\cdh}{\mathrm{cdh}}
\providecommand{\CH}{\mathrm{CH}}
\providecommand{\Ch}{\mathrm{Ch}}
\providecommand{\cl}{\mathrm{cl}}
\providecommand{\codim}{\mathrm{codim}}
\providecommand{\codom}{\mathrm{codom}}
\providecommand{\coeq}{\mathrm{coeq}}
\providecommand{\coev}{\mathrm{coev}}
\providecommand{\cof}{\mathrm{cof}}
\providecommand{\cofib}{\mathrm{cofib}}
\providecommand{\coker}{\mathrm{coker}}
\providecommand{\colim}{\mathrm{colim}}
\providecommand{\cone}{\mathrm{cone}}
\providecommand{\conj}{\mathrm{conj}}
\providecommand{\const}{\mathrm{const}}
\providecommand{\cyc}{\mathrm{cyc}}
\providecommand{\diag}{\mathrm{diag}}
\providecommand{\dg}{\mathrm{dg}}
\providecommand{\Disc}{\mathrm{Disc}}
\providecommand{\disc}{\mathrm{disc}}
\providecommand{\dual}{\mathrm{dual}}
\providecommand{\eff}{\mathrm{eff}}
\providecommand{\EKL}{\mathrm{EKL}}
\providecommand{\End}{\mathrm{End}}
\providecommand{\eq}{\mathrm{eq}}
\providecommand{\ess}{\mathrm{ess}}
\providecommand{\et}{\mathrm{et}}
\providecommand{\Et}{\mathrm{Et}}
\providecommand{\EU}{\mathrm{EU}}
\providecommand{\ev}{\mathrm{ev}}
\providecommand{\Ex}{\mathrm{Ex}}
\providecommand{\ex}{\mathrm{ex}}
\providecommand{\Exc}{\mathrm{Exc}}
\providecommand{\Ext}{\mathrm{Ext}}
\providecommand{\fib}{\mathrm{fib}}
\providecommand{\Fix}{\mathrm{Fix}}
\providecommand{\fppf}{\mathrm{fppf}}
\providecommand{\fpqc}{\mathrm{fpqc}}
\providecommand{\Frac}{\mathrm{Frac}}
\providecommand{\Frob}{\mathrm{Frob}}
\providecommand{\Fun}{\mathrm{Fun}}
\providecommand{\Gal}{\mathrm{Gal}}
\providecommand{\gen}{\mathrm{gen}}
\providecommand{\GL}{\mathrm{GL}}
\providecommand{\gp}{\mathrm{gp}}
\providecommand{\Gr}{\mathrm{Gr}}
\providecommand{\gr}{\mathrm{gr}}
\providecommand{\GW}{\mathrm{GW}}
\providecommand{\Her}{\mathrm{Her}}
\providecommand{\Ho}{\mathrm{Ho}}
\providecommand{\hocofib}{\mathrm{hocofib}}
\providecommand{\hocolim}{\mathrm{hocolim}}
\providecommand{\hofib}{\mathrm{hofib}}
\providecommand{\holim}{\mathrm{holim}}
\providecommand{\Hom}{\mathrm{Hom}}
\providecommand{\htp}{\mathrm{htp}}
\providecommand{\id}{\mathrm{id}}
\providecommand{\Idem}{\mathrm{Idem}}
\providecommand{\im}{\mathrm{im}}
\providecommand{\incl}{\mathrm{incl}}
\providecommand{\Ind}{\mathrm{Ind}}
\providecommand{\ind}{\mathrm{ind}}
\providecommand{\inj}{\mathrm{inj}}
\providecommand{\Inn}{\mathrm{Inn}}
\providecommand{\inv}{\mathrm{inv}}
\providecommand{\iso}{\mathrm{iso}}
\providecommand{\Jac}{\mathrm{Jac}}
\providecommand{\KGL}{\mathrm{KGL}}
\providecommand{\kgl}{\mathrm{kgl}}
\providecommand{\KH}{\mathrm{KH}}
\providecommand{\KO}{\mathrm{KO}}
\providecommand{\ko}{\mathrm{ko}}
\providecommand{\KQ}{\mathrm{KQ}}
\providecommand{\kq}{\mathrm{kq}}
\providecommand{\KR}{\mathrm{KR}}
\providecommand{\KSp}{\mathrm{KSp}}
\providecommand{\KU}{\mathrm{KU}}
\providecommand{\ku}{\mathrm{ku}}
\providecommand{\Lan}{\mathrm{Lan}}
\providecommand{\Map}{\mathrm{Map}}
\providecommand{\map}{\mathrm{map}}
\providecommand{\MGL}{\mathrm{MGL}}
\providecommand{\MO}{\mathrm{MO}}
\providecommand{\Mor}{\mathrm{Mor}}
\providecommand{\mor}{\mathrm{mor}}
\providecommand{\mot}{\mathrm{mot}}
\providecommand{\MSL}{\mathrm{MSL}}
\providecommand{\MSLc}{\mathrm{MSL}^{\mathrm{c}}}
\providecommand{\MSO}{\mathrm{MSO}}
\providecommand{\MSp}{\mathrm{MSp}}
\providecommand{\MSU}{\mathrm{MSU}}
\providecommand{\MU}{\mathrm{MU}}
\providecommand{\mult}{\mathrm{mult}}
\providecommand{\Nis}{\mathrm{Nis}}
\providecommand{\ob}{\mathrm{ob}}
\providecommand{\obj}{\mathrm{obj}}
\providecommand{\op}{\mathrm{op}}
\providecommand{\Orb}{\mathrm{Orb}}
\providecommand{\ord}{\mathrm{ord}}
\providecommand{\Out}{\mathrm{Out}}
\providecommand{\perf}{\mathrm{perf}}
\providecommand{\Perm}{\mathrm{Perm}}
\providecommand{\PGL}{\mathrm{PGL}}
\providecommand{\Pic}{\mathrm{Pic}}
\providecommand{\pr}{\mathrm{pr}}
\providecommand{\pre}{\mathrm{pre}}
\providecommand{\Prin}{\mathrm{Prin}}
\providecommand{\Proj}{\mathrm{Proj}}
\providecommand{\proj}{\mathrm{proj}}
\providecommand{\PSL}{\mathrm{PSL}}
\providecommand{\quot}{\mathrm{quot}}
\providecommand{\Ran}{\mathrm{Ran}}
\providecommand{\rank}{\mathrm{rank}}
\providecommand{\Res}{\mathrm{Res}}
\providecommand{\RO}{\mathrm{RO}}
\providecommand{\sep}{\mathrm{sep}}
\providecommand{\sgn}{\mathrm{sgn}}
\providecommand{\SH}{\mathrm{SH}}
\providecommand{\sig}{\mathrm{sig}}
\providecommand{\Sing}{\mathrm{Sing}}
\providecommand{\SL}{\mathrm{SL}}
\providecommand{\SO}{\mathrm{SO}}
\providecommand{\soc}{\mathrm{soc}}
\providecommand{\Sp}{\mathrm{Sp}}
\providecommand{\Span}{\mathrm{Span}}
\providecommand{\Spec}{\mathrm{Spec}\hspace{0.1em}}
\providecommand{\Spin}{\mathrm{Spin}}
\providecommand{\spn}{\mathrm{spn}}
\providecommand{\Sq}{\mathrm{Sq}}
\providecommand{\st}{\mathrm{st}}
\providecommand{\Stab}{\mathrm{Stab}}
\providecommand{\SU}{\mathrm{SU}}
\providecommand{\supp}{\mathrm{supp}}
\providecommand{\Syl}{\mathrm{Syl}}
\providecommand{\syl}{\mathrm{syl}}
\providecommand{\Sym}{\mathrm{Sym}}
\providecommand{\syn}{\mathrm{syn}}
\providecommand{\SYT}{\mathrm{SYT}}
\providecommand{\TC}{\mathrm{TC}}
\providecommand{\td}{\mathrm{td}}
\providecommand{\Th}{\mathrm{Th}}
\providecommand{\THH}{\mathrm{THH}}
\providecommand{\Tor}{\mathrm{Tor}}
\providecommand{\TP}{\mathrm{TP}}
\providecommand{\TR}{\mathrm{TR}}
\providecommand{\Tr}{\mathrm{Tr}}
\providecommand{\tr}{\mathrm{tr}}
\providecommand{\univ}{\mathrm{univ}}
\providecommand{\veff}{\mathrm{veff}}
\providecommand{\vol}{\mathrm{vol}}
\providecommand{\Wel}{\mathrm{Wel}}
\providecommand{\Wr}{\mathrm{Wr}}
\providecommand{\Zar}{\mathrm{Zar}}

% Special text commands
\providecommand{\et}{\text{\'{e}t}}
\renewcommand{\Im}{\mathrm{Im}}
\renewcommand{\Re}{\mathrm{Re}}
\providecommand{\Spec}{\text{Spec}\hspace{0.1em}}
\providecommand{\spn}{\text{span}}

% Blackboard letters
\providecommand{\A}{\mathbb{A}}
\providecommand{\C}{\mathbb{C}}
\providecommand{\F}{\mathbb{F}}
\providecommand{\G}{\mathbb{G}}
\providecommand{\H}{\mathbb{H}}
\providecommand{\N}{\mathbb{N}}
\providecommand{\P}{\mathbb{P}}
\providecommand{\Q}{\mathbb{Q}}
\providecommand{\R}{\mathbb{R}}
\providecommand{\Z}{\mathbb{Z}}

% Categories
\providecommand{\Ab}{\mathrm{Ab}}
\providecommand{\Alg}{\mathrm{Alg}}
\providecommand{\Ani}{\mathrm{Ani}}
\providecommand{\Bimod}{\mathrm{Bimod}}
\providecommand{\CAlg}{\mathrm{CAlg}}
\providecommand{\Cat}{\mathrm{Cat}}
\providecommand{\CDGA}{\mathrm{CDGA}}
\providecommand{\CG}{\mathrm{CG}}
\providecommand{\CGWH}{\mathrm{CGWH}}
\providecommand{\Ch}{\mathrm{Ch}}
\providecommand{\CMon}{\mathrm{CMon}}
\providecommand{\coAlg}{\mathrm{coAlg}}
\providecommand{\Coh}{\mathrm{Coh}}
\providecommand{\CommRing}{\mathrm{CommRing}}
\providecommand{\ConjSub}{\mathrm{ConjSub}}
\providecommand{\coMod}{\mathrm{coMod}}
\providecommand{\Cor}{\mathrm{Cor}}
\providecommand{\Corr}{\mathrm{Corr}}
\providecommand{\CoSh}{\mathrm{CoSh}}
\providecommand{\CRing}{\mathrm{CRing}}
\providecommand{\CW}{\mathrm{CW}}
\providecommand{\Field}{\mathrm{Field}}
\providecommand{\Fin}{\mathrm{Fin}}
\providecommand{\FinSet}{\mathrm{FinSet}}
\providecommand{\Gpd}{\mathrm{Gpd}}
\providecommand{\Grp}{\mathrm{Grp}}
\providecommand{\Grpd}{\mathrm{Grpd}}
\providecommand{\Grph}{\mathrm{Grph}}
\providecommand{\Kan}{\mathrm{Kan}}
\providecommand{\Kar}{\mathrm{Kar}}
\providecommand{\LMod}{\mathrm{LMod}}
\providecommand{\Mfld}{\mathrm{Mfld}}
\providecommand{\Mod}{\mathrm{Mod}}
\providecommand{\NAlg}{\mathrm{NAlg}}
\providecommand{\Ouv}{\mathrm{Ouv}}
\providecommand{\Perf}{\mathrm{Perf}}
\providecommand{\Poset}{\mathrm{Poset}}
\providecommand{\Pr}{\mathrm{Pr}}
\providecommand{\Pre}{\mathrm{Pre}}
\providecommand{\PSh}{\mathrm{PSh}}
\providecommand{\PShv}{\mathrm{PShv}}
\providecommand{\qCat}{\mathrm{qCat}}
\providecommand{\QCoh}{\mathrm{QCoh}}
\providecommand{\Rep}{\mathrm{Rep}}
\providecommand{\Ring}{\mathrm{Ring}}
\providecommand{\RMod}{\mathrm{RMod}}
\providecommand{\sAb}{\mathrm{sAb}}
\providecommand{\Sch}{\mathrm{Sch}}
\providecommand{\Set}{\mathrm{Set}}
\providecommand{\SH}{\mathrm{SH}}
\providecommand{\Sh}{\mathrm{Sh}}
\providecommand{\Shv}{\mathrm{Shv}}
\providecommand{\Sm}{\mathrm{Sm}}
\providecommand{\Sp}{\mathrm{Sp}}
\providecommand{\Spectra}{\mathrm{Spectra}}
\providecommand{\Spc}{\mathrm{Spc}}
\providecommand{\sPre}{\mathrm{sPre}}
\providecommand{\Spt}{\mathrm{Spt}}
\providecommand{\sSet}{\mathrm{sSet}}
\providecommand{\sShv}{\mathrm{sShv}}
\providecommand{\Stack}{\mathrm{Stack}}
\providecommand{\Sub}{\mathrm{Sub}}
\providecommand{\Top}{\mathrm{Top}}
\providecommand{\Tors}{\mathrm{Tors}}
\providecommand{\Var}{\mathrm{Var}}
\providecommand{\Vect}{\mathrm{Vect}}

%%%%%%%%%%%%
% category_theory
% For blackboard bold number and delta categories
\RequirePackage{bbm}
\providecommand{\onecat}{\mathbbm{1}}
\providecommand{\twocat}{\mathbbm{2}}

% Blackboard delta
\RequirePackage{pict2e,picture}

\makeatletter
\DeclareRobustCommand{\DDelta}{{\mathpalette\bb@Delta\relax}}
\newcommand{\bb@Delta}[2]{%
  \begingroup
  \sbox\z@{$\m@th#1\Delta$}%
  \dimendef\Dht=6 \dimendef\Dwd=8
  \setlength{\Dwd}{\wd\z@}%
  \setlength{\Dht}{\ht\z@}%
  \begin{picture}(\Dwd,\Dht)
  \put(0,0){$\m@th#1\Delta$}
  \put(.42\Dwd,.7\Dht){\line(10,-26){.25\Dht}}
  \end{picture}%
  \endgroup
}

% Heart (for e.g. t-structures)
\usepackage{graphicx}
\newcommand{\heart}{\ensuremath\heartsuit}

% Other
\providecommand{\HZ}{\mathrm{H}\mathbb{Z}}
\providecommand{\Gm}{\mathbb{G}_m}
% Spaces
\providecommand{\CP}{{\mathbb{C}\text{P}}}
\providecommand{\HP}{{\mathbb{H}\text{P}}}
\providecommand{\RP}{{\mathbb{R}\text{P}}}

\renewcommand{\O}{\mathcal{O}}
\renewcommand{\P}{\mathbb{P}}


\usepackage{cleveref}
\let\fullref\autoref
%
\def\makeautorefname#1#2{\expandafter\def\csname#1autorefname\endcsname{#2}}
%  
\makeautorefname{eqn}{Equation}%
\makeautorefname{sec}{Section}%
\makeautorefname{subsec}{Subsection}%
\makeautorefname{footnote}{footnote}%
\makeautorefname{item}{item}%
\makeautorefname{figure}{Figure}%
\makeautorefname{table}{Table}%
\makeautorefname{wraptab}{wraptable}%
\makeautorefname{part}{Part}%
\makeautorefname{app}{Appendix}%
\makeautorefname{cla}{claim}%
\makeautorefname{ans}{answer}%
\makeautorefname{assump}{assumption}%
\makeautorefname{conj}{conjecture}%
\makeautorefname{cor}{corollary}%
\makeautorefname{cex}{counterexample}%
\makeautorefname{cexs}{counterexamples}%
\makeautorefname{dig}{digression}%
\makeautorefname{disc}{discussion}%
\makeautorefname{def}{definition}%
\makeautorefname{ex}{example}%
\makeautorefname{exs}{examples}%
\makeautorefname{fac}{fact}%
\makeautorefname{goal}{goal}%
\makeautorefname{intu}{intuition}%
\makeautorefname{lem}{lemma}%
\makeautorefname{meta}{metathm}%
\makeautorefname{motiv}{motivation}%
\makeautorefname{nota}{notation}%
\makeautorefname{note}{note}%
\makeautorefname{warn}{warning}%
\makeautorefname{prop}{proposition}%
\makeautorefname{ques}{question}%
\makeautorefname{rmk}{remark}%
\makeautorefname{set}{setup}%
\makeautorefname{strat}{strategy}%
\makeautorefname{term}{terminology}%
\makeautorefname{thm}{theorem}%
\makeautorefname{upsh}{upshot}%
%
%                  *** End of hyperref stuff ***

\theoremstyle{definition}
\newtheorem{theorem}{Theorem}[section]
\numberwithin{theorem}{section} % important bit
\newtheorem{answer}[theorem]{Answer}
\newtheorem{assumption}[theorem]{Assumption}
\newtheorem{claim}[theorem]{Claim}
\newtheorem{conjecture}[theorem]{Conjecture}
\newtheorem{corollary}[theorem]{Corollary}
\newtheorem{counterexample}[theorem]{Counterexample}
\newtheorem{definition}[theorem]{Definition}
\newtheorem{digression}[theorem]{Digression}
\newtheorem{discussion}[theorem]{Discussion}
\newtheorem{example}[theorem]{Example}
\newtheorem{examples}[theorem]{Examples}
\newtheorem{exercise}[theorem]{Exercise}
\newtheorem{fact}[theorem]{Fact}
\newtheorem{goal}[theorem]{Goal}
\newtheorem{idea}[theorem]{Idea}
\newtheorem{intuition}[theorem]{Intuition}
\newtheorem{lemma}[theorem]{Lemma}
\newtheorem{metathm}[theorem]{Meta-theorem}
\newtheorem{motivation}[theorem]{Motivation}
\newtheorem{notation}[theorem]{Notation}
\newtheorem{note}[theorem]{Note}
\newtheorem{proposition}[theorem]{Proposition}
\newtheorem{question}[theorem]{Question}
\newtheorem{remark}[theorem]{Remark}
\newtheorem{setup}[theorem]{Setup}
\newtheorem{strategy}[theorem]{Strategy}
\newtheorem{terminology}[theorem]{Terminology}
\newtheorem{upshot}[theorem]{Upshot}
\newtheorem{warning}[theorem]{Warning}

%%%% hack to get fullref working correctly
\makeatletter
\let\c@corollary=\c@theorem
\let\c@proposition=\c@theorem
\let\c@lemma=\c@theorem
\let\c@assumption=\c@theorem
\let\c@conjecture=\c@theorem
\let\c@definition=\c@theorem
\let\c@example=\c@theorem
\let\c@remark=\c@theorem
\let\c@notation=\c@theorem
\let\c@equation\c@theorem
\let\c@strategy\c@theorem
\makeatother

\renewcommand*{\subsectionautorefname}{Subsection}
\renewcommand*{\sectionautorefname}{Section}


\usepackage{epigraph}

% For editing purposes, disable on commit
%\usepackage{showkeys}

\DeclareFieldFormat{postnote}{#1}
\DeclareFieldFormat{multipostnote}{#1}

\def\theshiftamount{2}

\let\del\partial
\let\til\widetilde
\let\nsubgp\trianglelefteq
\let\smashprod\wedge

% get rid of fancyhdr errors
\setlength{\footskip}{13.6pt}
\setlength{\parskip}{0.5em}


% Custom quote block command
\setlength{\epigraphwidth}{0.8\textwidth}
\newcommand{\quoteblock}[2]{\epigraph{\itshape#1}{#2}}

\providecommand{\Conf}{\mathrm{Conf}}
\providecommand{\Mon}{\mathrm{Mon}}

\fancyfoot[C]{\thepage}
\begin{document}

\begin{abstract} \href{https://github.com/tbrazel/galois-notes}{https://github.com/tbrazel/galois-notes}
\end{abstract}

\maketitle

% \setcounter{tocdepth}{1}
% \tableofcontents{}


\setcounter{section}{-1}
\section{About}

Some notes from a mini-seminar run in Spring 2025 on Galois groups of enumerative problems, following Joe Harris' 1979 paper of the same name \cite{Harris-Galois}. These notes aren't intended to be a definitive reference, but are perhaps more narrow than the survey paper \cite{SottileYahl}, for instance. Our goals are to fill out details and examples to better understand how to carry out computations in monodromy and enumerative geometry.

\section{Talk 1: Thomas Brazelton, 3/10}

\begin{question} Given an enumerative problem, to what field extension do we need to pass to in order to find all its solutions?
\end{question}

This question has its roots in 19th century algebraic geometry, and was crystallized by Jordan in Chapter III of his treatise on Galois theory:


\quoteblock{%
Les solutions des problèmes dont il s'agit sont en général assez nombreuses, mais liées les unes aux autres par certaines relations géometriques. De ces relations on déduit immédiatement l'existence d'une fonction entière $\phi(x_0,x_1,\ldots)$ dont le group contient celui de l'équation $X$. Réciproquement, si l'on était certain de connaître \emph{toutes} les relations géométriques que présente la question proposée (ou du moins celles dont les autres dérivent), le group de l'équation $X$ contiendrait toutes les substitutions du group de $\phi(x_0,x_1,\ldots)$. Mais une semblable certitude est difficile à obtenir, malgré le soin apporté par d'habiles géomètres à l'étude de ces problèmes. Il ne serait donc pas impossible que les équations auxquelles ces problèmes donnent naissance eussent parfois une forme plus particulière encore que celle que nous allons trouver, en nous appuyant sur les résultats obtenus par nos prédécesseurs}{%
Camille Jordan, 1870, \cite[pp.301-302]{Jordan}}

\begin{question} Are these sorts of questions \textit{solvable}? Meaning solvable in radicals?
\end{question}


The six subsections of Chapter III in Jordan's book are dedicated to particular problems whose Galois group he is interested in studying. For instance:
\begin{itemize}
    \item[III.I] Given a planar cubic curve, we know by work of Hesse that it has the property that the line passing through any two of its flexes pass through the cubic at a third inflection point. As there are nine flexes, each flex lies on four of these lines, for a total of 12 lines.\footnote{I'd love to include a nice picture here, but unfortunately by a result of Klein, a smooth real planar cubic will have at most three of its nine flexes defined over $\R$ \cite{Ronga-Klein}.} \emph{Can we obtain a formula for the flexes in terms of the coefficients of the cubic?}

    \item[III.II] Given a quartic curve, can we find a cubic curve so that among the 12 points of intersection, they are in three sets of four colinear points? This follows work of Clebsch, who asked \textit{contact problems} of a similar flavor (see \cite[(429)]{Jordan}).

    \item[III.III] Again following work of Clebsch, given a quartic surface with a double conic, we can find five cones whose edges are bitangent to the surface. There are 16 lines on the quartic surface, each of which intersect the doubled conic at a single point. Can we solve for the lines given the quartic surface or the cones?
    
    \item[III.IV] Kummer showed there exist quartic surfaces with 16 singular points, which lie in groups of six on singular tangent planes, each of which intersect other tangent planes at these six points. Can we solve for these points (or these planes) given the Kummer surface?\footnote{This configuration admits a really nice contemporary description in terms of $\theta$-characteristics. We should come back and fill out the details here.}

    \item[III.V] Following Steiner,\footnote{%
    %
    The Steiner reference is his 1857 paper \textit{Über die Flächen dritten Grades} (On cubic surfaces) \cite{Steiner-cubic-surfaces}, published in Crelle's Journal, which was called Borchardt's Journal during Borchardt's tenure as editor (1856--1880). It's interesting to me that Jordan references work of Steiner, rather than work of Cayley and Salmon, from 1849 and 1847, respectively.} every cubic surface has 27 lines. These lines lie on 45 tritangents, and two tritangent planes always intersect at some line, not necessarily a line on the cubic surface though. If two tritangent planes $a_1b_1c_1$ and $a_2b_2c_2$ don't meet at a line on a cubic surface, then there exists another tritangent plane $a_3b_3c_3$ for which $a_1a_2a_3$, $b_1b_2b_3$, and $c_1c_2c_3$ form tritangents. These are called \textit{trihedral pairs}, and there are 120 of them (see e.g. \cite[p.~112]{Hunt}). Can we solve for the equations of the lines given the equation of the cubic surface? Can we incorporate the constraints imposed by the trihedral pairs?

    \item[III.VI] Again following Clebsch, fix a curve $C$ of order $n$ and $\frac{n(n-3)}{2}$ points on it. Can we determine the curves of order $n-3$ intersecting $C$? For example if $n=4$, can we determine all the 28 bitangents to $C$?
\end{itemize}

The idea of a \textit{Galois group} of an enumerative problem is a setting in which we can capture these kinds of algebraic questions. The right way to phrase these kinds of questions is in the language of generically finite maps of varieties.

\subsection{Setup and goal}

Let $X$ and $Y$ be irreducible algebraic varieties of the same dimension over $\C$, let $\pi \colon Y \to X$ have degree $d>0$. This induces a map on function fields $K(X) \to K(Y)$ which is a finite field extension.\footnote{If $X$ and $Y$ are affine, it's clear this function field extension is finite. In the general case, we can reduce to the affine case by looking at the generic points (since they're irreducible), see e.g. \cite[02NW]{Stacks}.} and necessarily separable since we are in characteristic zero.

We're going to pick some nice $p \in X$ (in the region over which $\pi$ is finite), and look at its fibers $\pi^{-1}(p) = \left\{ q_1, \ldots, q_d \right\}$. We'll define two ways to permute the fibers -- one coming from Galois theory and one coming from monodromy, and we'll demonstrate that these are equal, following \cite[\S1]{Harris-Galois}.

\subsection{The Galois group, formally}

By the primitive element theorem, $K(Y)$ is generated over $K(X)$ by some rational function $f \in K(Y)$ which satisfies a minimal polynomial relation:
\begin{align*}
    P(f) = f^d + g_1 f^{d-1} + \ldots + g_{d-1}f + g_d = 0,
\end{align*}
where $g_1, \ldots, g_d\in K(X)$.

Recall that $\O_X$ is the sheaf of holomorphic functions. We obtain the sheaf $\mathcal{K}_X$ of meromorphic functions as the quotient ring. We get an injection of sheaves $\O_X \to \mathcal{K}_X$. In the general scheme-theoretic setup, this need not be a field, however since our $X$ is particularly nice (integral, Noetherian,..) it will be. So we can look at the germs of meromorphic functions around $p$, which forms a \textit{field} $\mathcal{K}_{X,p}$. The covering map induces an isomorphism of fields at each $q_\alpha$:
\begin{align*}
    \pi_\alpha := \pi_\ast \colon K_{Y,q_\alpha} \xto{\sim} K_{X,p}.
\end{align*}
Let's define $\phi$ to be the inclusion of fields obtained by restricting global meromorphic functions around $p$:
\begin{align*}
    \phi \colon K(X) \hookto \mathcal{K}_{X,p},
\end{align*}
and $\phi_\alpha$ to be the composite
\begin{align*}
    K(Y) \to \mathcal{K}_{Y,q_\alpha} \xto{\pi_\alpha} \mathcal{K}_{X,p}.
\end{align*}
%
We can sit everything inside $\mathcal{K}_{X,p}$, so let's fix some notation:
\begin{enumerate}
    \item $K$ is the image of $K(X)$ in $\mathcal{K}_{X,p}$ -- it is the restriction of global meromorphic functions on $X$ to a neighborhood of $p$
    \item $L$ is the subfield of $\mathcal{K}_{X,p}$ generated by all the images of the $\phi_\alpha$'s. --- this is meromorphic functions around $p$, which are coming from meromorphic functions on $Y$ that have been restricted to some neighborhood of some $q_\alpha$
\end{enumerate}
Finally we take our elements $f \in K(Y)$ and $g_i \in K(X)$ and sit them inside the larger field:
\begin{align*}
    \phi \colon K(X) &\to K \subseteq\mathcal{K}_{X,p} \\
    g_i &\mapsto \til{g}_i.
\end{align*}
and
\begin{align*}
    \phi_\alpha \colon K(Y) &\to \mathcal{K}_{X,p} \\
    f &\mapsto \til{f}_\alpha.
\end{align*}
%
Observe that all the $\til{f}_\alpha$'s are distinct! This is because in order for $f$ to generate $K(Y)$ over $K(X)$, it must take different values at each $q_\alpha$.

%
Moreover, each of the $\til{f}_\alpha$'s satisfy the image of the polynomial relation $P(f) = 0$ in $L$:
\begin{align*}
    \til{P}(\til{f}_\alpha) &= \til{f}_\alpha^d + \til{g}_1 \til{f}_\alpha^{d-1} + \ldots + \til{g}_d = 0.
\end{align*}
%
Let's think about $\til{P}$ as a polynomial in $L[t]$. It is a degree $d$ polynomial with $d$ distinct roots, given by the $\til{f}_\alpha$'s.

We claim then that $L$ is the normalization of $K(Y)/K(X)$, (which is identically $K_\alpha/K$. This is because the minimal polynomial $P(t)$ splits in $L$, and $L$ is the minimal field over which this occurs. So we have argued:

\begin{proposition} $\Gal(L/K)$ is Galois.
\end{proposition}

The action of the Galois group permutes all the $\til{f}_\alpha$'s, which permutes the indices $\alpha$, giving us an inclusion
\begin{align*}
    \Gal(L/K) \hookto \Sigma_d.
\end{align*}
The image of this is the \textit{Galois group} of the enumerative problem.

\subsection{The monodromy group}

Since $\pi \colon Y \to X$ is an branched cover, it satisfies homotopy lifting away from the branched points. That is, any path in $X$ which doesn't pass through the branch locus can be lifted to a path in $Y$ after a starting point has been chosen.

To that end, pick some $U \subseteq X$ Zariski open containing our $p$ and $V = \pi^{-1}(U)$ so that $\pi \colon V \to U$ is an unbranched cover. Then we obtain an inclusion
\begin{align*}
    \pi_1(U,p) \to \Sigma_d,
\end{align*}
given by the action of the deck group. The image of this is called the \textit{monodromy group} of our enumerative problem.

\subsection{Monodromy and analytic continuation}

Let $X$ be a Riemann surface, and consider a path $\gamma \colon [0,1] \to X$, and let's take two holomorphic germs $f\in \O_{X,\gamma(0)}$ and $g \in \O_{X,\gamma(1)}$. We say $g$ is the \textit{analytic continuation of $f$ along $\gamma$} if there is a finite sequence of open sets $U_i$ along the image of $\gamma$\footnote{We may assume finite since the image of $\gamma$ is compact.} and functions $f_i \in \O(U_i)$ so that $f_1 = f$ and $f_n = g$.

\begin{theorem} If $\pi \colon Y \to X$ is an unbranched cover of a Riemann surface $X$, and $\gamma \in \pi_1(X,x)$ is some loop, then for any $f \in \O_{X,x}$, any choice of lift $\widetilde{\gamma} \colon [0,1] \to Y$, and any germ $g\in \O_{Y,\widetilde{\gamma}(0)}$ with $\pi_\ast g = f$, we have that analytic continuation of $g$ along $\widetilde{\gamma}$ exists, and the resulting germ $\til{g}$ also satisfies $\pi_\ast \til{g} = f$.
\end{theorem}

\subsection{The main result}

We are going to prove the following:

\begin{proposition} The Galois group $G$ equals the monodromy group $M$ in our setup.
\end{proposition}

The first step is to argue that $M \subseteq G$. That is, any permutation coming from monodromy can be realized by an automorphism of $L$ over $K$. This follows via analytic continuation!

If we pick some $\gamma \in \pi_1(U,p)$, then any lift of $\gamma$ to $V$ will send some $\til{f}_\alpha$ to some $\til{f}_\beta$. In particular since $K = \im(K(X) \hookto \mathcal{K}_{X,p})$ is fixed under analytic continuation, and since any element in $L$ is polynomial in germs at the $q_\alpha$'s, analytic continuation induces a field automorphism of $L/K$ permuting the $\til{f}_\alpha$'s. That is, this permutation lies in the Galois group.

For the other direction, we want to see that the containment $M \subseteq G$ is not strict. Indeed if it were, then the $M$-fixed subfield $L^M$ would not be equal to $K$. So it suffices to argue that everything in $L$ fixed by the monodromy action is actually in $K$. As we have seen, the monodromy action occurs via analytic continuation.

So take some $h \in L \subseteq \mathcal{K}_{X,p}$, and suppose $h$ is fixed under analytic continuation along any lift of an element in $\pi_1(U,p)$. We want to argue that $h$ is the restriction of a global meromorphic function on $X$ to a neighborhood of $p$. We'll define a candidate one -- define $\til{h}$ on $U$ by picking, for every $p' \in U$, an arc from $p$ to $p'$ and analytically continuing $h$ along it. This is well-defined precisely because $h$ is fixed under the monodromy action, so we obtain a well-defined value in a neighborhood of $p'$, independent of the path we chose.

We now know that $h$ extends to a meromorphic function $\til{h}$ on $U$. We want to see that this extends to all of $X$. In order to do this, we exploit that $h \in L$. So $h$ can be written as some polynomial in $\til{h}_\alpha$'s, where $\til{h}_\alpha$ is a meromorphic function on all of $Y$ restricted to a neighborhood of $q_\alpha$. None of these have essential singularities, and this is unchanged when taking a polynomial in them. Hence $\til{h}$ has no essential singularities, and therefore extends to a meromorphic function on $X$ whose germ at $p$ is $h$. Thus $L^M = K$, and we are done.

\subsection{An example we know and love}

Consider the incidence variety of roots of a univariate polynomial
\begin{align*}
    Y = \left\{ (f,t) \colon f(t) = 0 \right\} \subseteq \P^d \times \P^1.
\end{align*}
Then $Y \to \P^d$ is a projective bundle. We want to argue its Galois group is $S_n$. We need to show it is $2$-transitive\footnote{Recall a group $G$ is $2$\emph{-transitive} if it acts transitively on the set $G \times G \minus \Delta$ of ordered pairs of elements in $G$.}
and it contains a single transposition.

\begin{proposition} If $Y \to X$ is generically finite, then the Galois group of $Y \to X$ is 2-transitive if and only if $Y \times_X Y \minus \Delta$ is irreducible.\footnote{There is also a version for $k$-transitivity.} 
\end{proposition}

In our case, we can argue that $Y \times_{\P^d} Y\minus\Delta$ is irreducible.

To see it has a transposition, we find a polynomial with a single double root and all other roots distinct. Then the map $Y \to \P^d$ is locally irreducible near the fiber where the two sheets merge to form a double root, and monodromy around it yields a single transposition. This is phrased and proven precisely in a lemma on \cite[p.~698]{Harris-Galois}, which we will get to in a later talk.

\textbf{An ahistorical application}: The Galois group of solving for a root of a degree $n$ polynomial is $\Sigma_n$, and hence not solvable by radicals for $n\ge 5$.

\section{Flexes and bitangents}

\subsection{Historical background}

The study of high degree plane curves (degrees three or greater) dates back to Newton, but one of the first major results in this are came from Pl\"{u}cker (\textit{System der analytischen Geometrie}, page 264).

\begin{theorem}[Plucker, 1835] A general degree $d$ planar curve has $3d(d-2)$ inflection points.
\end{theorem}
Pl\"{u}cker proved this by equation-bashing. We provide a slightly differeent proof, due to Hesse but likely predating him (see \cite[p.~169]{Gray-worlds}).

\begin{proof}[Hesse's proof]
Recall that the radius of curvature of a planar curve $F(x,y,z)=0$ at a point $(x_0,y_0,z_0)$ is the reciprocal of $\det(HF)_{|(x_0,y_0,z_0)}$, where $HF$ is the Hessian matrix of $F$. Note that the equation $\det HF=0$ is homogeneous of degree $3(d-2)$, and note further that a flex on a curve is precisely a point with infinite curvature radius. Hence we can count inflection points on $F$ via the intersection with its Hessian curve, hence by B\'{e}zout's theorem we have $3d(d-2)$ flexes.
\end{proof}

The so-called \textit{Pl\"ucker formulas} are highly related, and come from the same text. Pl\"{u}cker computed what is now called the \textit{class} of the curve $C$, namely the number of tangent lines to $C$ through another point $p$ on the plane. Note that under pole-polar duality, this is also the degree of the \textit{dual curve}. He observed by direct computation that the class of a general curve of degree $d$ is $d(d-1)$.

In this direct algebraic approach, any line through $p$ passing through two points of $C$ (counted with multiplicity) qualifies as a tangent. Presuming that $C$ is smooth (and if we take $C$ to be general this is the case) then this is true on the nose, however even if $C$ is mildly singular this fails. Pl\"{u}cker observed that the quantity $d(d-1)$ overcounts the honest class of the curve by two for every double point of $C$ and by three for every cusp of $C$. This leads us to the formula
\begin{align*}
    d^\ast = d(d-1) - 2\delta - 3\kappa
\end{align*}
where $\delta$ is the number of nodes of $C$ and $\kappa$ is the number of cusps. This is the \textit{Pl\"ucker formula}, and it resolves the so-called ``duality paradox'' which plagued the study of pole-polar duality since its invention. A nice exposition to these results is in Coolidge's treatise \cite[Chapter~VI]{Coolidge}.

\begin{terminology} A reduced irreducible planar curve $C$ is said to be a \textit{Pl\"ucker curve} if it falls under the scope of the Pl\"{u}cker formula -- explicitly, if the only singularities of $C$ and $C^\ast$ are cusps and simple nodes. 
\end{terminology}


\begin{remark} \,
\begin{enumerate}
    \item More general Pl\"{u}cker formulae hold in which singularities are allowed to be much more badly behaved.
    \item Klein proved analogues of the Pl\"{u}cker formulae for real curves, treating split and non-split nodes differently. Leveraging his formula we are able to prove new results and give contemporary proofs of classical results known to Pl\"{u}cker, such as the fact that at most three of the nine flexes on a real cubic are real.
    \item Pl\"{u}cker's arguments were fairly nonrigorous, and it is ahistorical to attribute to him rigorous arguments that appear to be easy applications of his formula (for instance the computation that there are 28 bitangents to a planar quartic, attributed either to Hesse in 1848 or Jacobi in 1850).
\end{enumerate}
\end{remark}

Under duality, the Pl\"{u}cker formulae allow us to relate the numbers of bitangents and the numbers of flexes, since a bitangent to $C$ is a node on $C^\ast$, and a flex on $C$ is a cusp on $C^\ast$. If $C$ is a Pl\"{u}cker curve of degree $d$, we then have the relationship
\begin{align*}
    d = d^\ast(d^\ast-1) - 2b - 2f.
\end{align*}
Combining this with the number $3d(d-2)$ of flexes on $C$, we get
\begin{align*}
    d = d^\ast(d^\ast-1) - 2b - 2\cdot 3d(d-2).
\end{align*}
Let's now suppose that $C$ is suitably general, so that it is nonsingular and hence $d^\ast = d(d-1)$ with no correcion term. We then get
\begin{align*}
    d = d(d-1)\left( d(d-1)-1 \right) - 2b - 3\cdot 3d(d-2).
\end{align*}
Solving for $b$ we obtain
\begin{equation}\label{eqn:bitangents-in-terms-of-degree}
\begin{aligned}
    b = \frac{1}{2}d(d-2)(d^2-9).
\end{aligned}
\end{equation}
This is the number of bitangents on a general smooth curve of degree $d$ as estimated by Pl\"{u}cker and proven by Jacobi.

\subsection{Monodromy of flexes: the setup}

Let $W_d := \mathbb{P}(H^0(\O_{\P^2}(d))) \cong \P^{\binom{d}{2}-1}$ be the complete linear system of degree $d$ plane curves, and let $I_0$ be the locus of points on lines
\begin{align*}
    I_0 := \left\{ (p,\ell) \colon p\in \ell \right\} \subseteq \P^2 \times (\P^2)^\ast.
\end{align*}
We let
\begin{align*}
    I_d := \left\{ (C,p,\ell) : m_p(C\cdot \ell)\ge 3 \right\} \subseteq W_d \times I_0
\end{align*}
be the incidence variety of curves equipped with a flex. We obtain projection maps
\[ \begin{tikzcd}
     & I_d\ar[dl,"\pi" above left]\ar[dr,"\eta" above right] & \\
    W_d &  & I_0.
\end{tikzcd} \]
%
\begin{proposition} If $d \ge 3$, then $\pi$ is generically finite of degree $3d(d-2)$.
\end{proposition}
\begin{proof} We can argue that the general curve of degree $d$ has only flexes and bitangents (it doesn't admit a hyperflex, a tritangent, or a flex bitangent). Hence over this open locus, $\pi$ has a well-defined degree. Note we restrict to $d\ge 3$, since a conic has no flexes. Obtaining the degree is exactly the Pl\"{u}cker formula, which as we have seen can be derived in multiple different ways.
\end{proof}


We want to try to find the monodromy group of $\pi$, and the following fact will be helpful.

\begin{proposition}\label{prop:Id-irred-flexes}
The incidence variety $I_d$ is irreducible.
\end{proposition}
\begin{proof} We can consider the projection $\eta \colon I_d \to I_0$. Since $I_0$ is irreducible, we will be able to conclude (by \autoref{prop:irred-criterion}) that $I_d$ is irreducible if $\eta$ has equidimensional fibers. Indeed we can check that the fiber of $\eta$ over any point $(p,\ell) \in I_0$ is exactly those degree $d$ plane curves with a flex line $\ell$ at the point $p$. This is always a codimension three linear subspace of $W_d$.
\end{proof}

We now want to argue that the monodromy group $\Mon(\pi)$ acts transitively on the fibers of $\pi$. By covering space theory, it will suffice to argue that, by restricting to the locus $U \subseteq W_d$ over which $\pi$ is unramified, the fiber $\pi^{-1}(U)$ is connected.

\begin{proposition}\label{prop:monodromy-flexes-transitive} 
The monodromy group acts transitively on the fibers of $\pi$.
\end{proposition}
\begin{proof} todo
\end{proof}


\begin{proposition} The monodromy group is $2$-transitive on the fibers of $\pi$.
\end{proposition}
\begin{proof} todo
\end{proof}


To argue that the monodromy group is full symmetric, it suffices to exhibit a transposition. Here we use this key lemma:

\begin{lemma}[{\cite[p.~698]{Harris-Galois}}] Let $\pi \colon Y \to X$ be holomorphic of degree $n$, and suppose there exists some $p\in X$ so that $\pi^{-1}(p) = \left\{ q_1, \ldots, q_{n-1} \right\}$ has $(n-1)$ distinct points, so that $\pi$ is simple at $q_1, \ldots, q_{n-2}$, and $\pi$ has a double point at $q_{n-1}$. Suppose further that $Y$ is locally irreducible at $q_{n-1}$. Then the monodromy group of $\pi$ contains a simple transposition, obtained by taking a small loop around $p$.
\end{lemma}
\begin{proof} todo
\end{proof}


So we'd like to locate a planar curve with $3d(d-2) - 2$ simple flexes, and exactly one \textit{hyperflex}. Note at the hyperflex the tangent line will meet to order $\ge 4$, and in particular this would violate B\'{e}zout's theorem if $d \le 3$.

\begin{theorem} For $d\ge 4$, the monodromy group of $I_d \xto{\pi} W_d$ is the full symmetric group $\Sigma_{3d(d-2)}$.
\end{theorem}
\begin{proof} Fix a point $(p_0,\ell_0) \in I_0$, and consider the linear system of degree $d$ curves with a hyperflex at $(p_0,\ell_0)$:
\begin{align*}
    W'' = \left\{ C \in W \colon m_{p_0}(C\cdot \ell_0) \ge 4 \right\} \subseteq W.
\end{align*}
A general $C \in W''$ is smooth at $p_0$ with $\ell_0$ a simple hyperflex (the multiplicity is exactly four). How do we know that the general element in $W''$ has only simple flexes at the other flex points though? Consider the incidence correspondence
\begin{align*}
    I'' = \left\{ (C,p,\ell) \colon m_p(C\cdot \ell) \ge3,\ p\ne p_0,\ \ell\ne \ell_0 \right\} \subseteq W'' \times I_0.
\end{align*}
The fibers of $\eta \colon I'' \to I_0$ are linear of codimension three in $W''$, therefore $I''$ is irreducible of dimension $\dim I'' = \dim W''$ (todo: why?). Then...
\end{proof}


\subsection{Monodromy of flexes on plane cubics}

todo


\section{Monodromy of bitangents}

A classical problem is to compute the monodromy group of the 28 bitangents to a smooth planar quartic. As mentioned, this was a particular application for Galois theory envisioned by Jordan \cite[III.VI]{Jordan}.

\subsection{History of bitangents to quartics}

As discussed earlier, the expression (\autoref{eqn:bitangents-in-terms-of-degree}) for the number of bitangents to a smooth curve of degree $d$, in terms of $d$, was known to Pl\"{u}cker in the 1830's. Despite this, Pl\"{u}cker's arguments are considered from a modern perspective (and likely also at the time) as sketchy at best. The so-called Pl\"{u}cker formulas were not rigorously proven by Pl\"{u}cker, and although he did sketch an accurate resolution of the pole-polar duality paradox, it did not constitute a proof.

The expression in \autoref{eqn:bitangents-in-terms-of-degree} was proven by Jacobi \cite{Jacobi1850}, an application of which is the computation that when $d=4$, a smooth planar quartic has 28 bitangents. This was extended shortly afterwards by Hesse, a student of Jacobi, who showed that a \textit{general} quartic has 28 bitangents \cite{Hesse1855}. Hesse's work was immensely difficult, and he likened it in a letter to Jacobi to Newton's work discovering the law of gravity \cite[p.~165]{Gray-worlds}.

For a very classical textbook treatment of the theory of bitangents to quartics, section 12 (\S95-\S105) of Weber's 1896 book on algebra is dedicated to this \cite[\S95-\S105]{Weber1896}. A contemporary treatment can be found in \cite[\S6.1]{Dolgachev}.

Jordan, inspired by work of Clebsch, initially asked to what extent equations for bitangents can be solved for in terms of the equations for a quartic, which is precisely the question of what the Galois group is. It's unclear to me who first solved for the Galois group of bitangents. We can write it in a number of different ways:
\begin{align*}
    O_6^-(\Z/2) \cong \Sp_6(\Z/2) \cong W(E_7)/\{\pm 1\}.
\end{align*}
The first two describe the same group (like even beyond group isomorphism they are describing the same quadratic forms in mildly different language). The latter group passes through the theory of Lie groups. Some references for this connection:
\begin{itemize}
    \item See \cite[Theorem~9]{DolgachevOrtland} for the derivation of this group in terms of the theory of Cayley octads (I can add the story of Cayley octads to these notes if people like)
    \item Manivel \cite{Manivel2006} discusses this, and mentions in the intro to the paper that the Galois groups for bitangents and lines to cubic surfaces were well-known at the beginning of the 1900's. He references a connection to Lie groups that passes through the theory of Del Pezzo surfaces of degrees 3 and 2. The point of Manivel's paper is cutting out this passage through Del Pezzos and drawing a direct connection to the Lie groups. I'd like to understand both of these stories.
    \item In \cite[p.~367]{MillerBlichfeldtDickson}, the Galois group of 28 bitangents is spelled out explicitly. The $7$ in $E_7$ comes in here through the theory of Aronhold sets --- that is, given an Aronhold set, we can solve for the remaining bitangents. What is this connection, and how does it relate to the theory of Cayley octads?
\end{itemize}


\subsection{Computing the monodromy of bitangents to plane curves}




... to show the monodromy is transitive, we have to show (by covering space theory) that the covering space is connected. To do this, we consider the projection
\begin{align*}
    \eta \colon J_d \to J_0 \cong \Conf_2(\P^2).
\end{align*}
The fiber of $\eta$ over a point in $J_0$ is a linear subspace in $J_d$. So $\eta$ is a fiber bundle, hence $J_d$ is irreducible, hence connected. Then the monodromy is transitive.

\begin{proposition} The monodromy acts 2-transitively on the fiber.
\end{proposition}
\begin{proof} It suffices to argue that the stabilizer of a bitangent, as a subgroup of monodromy, acts on the remaining points in the fiber transitively....
\end{proof}


To exhibit a simple transposition, we want to find a curve with one fewer than the generic number of bitangents. It will be a curve with a simple flex bitangent. This will exist for $d \ge 5$ (because we can't have a flex bitangent for a degree four curve by B\'{e}zout).


\subsection{Monodromy of bitangents to quartics}

Given a bitangent $\overline{p_1 p_2}$ to a generic quartic $C$, note that $2p_1 + 2p_2 \sim K_C$ is the canonical class, i.e. a hyperplane section of the quartic. Conversely, since $C$ is ``nice'' any divisor equivalent to $K_C$ which has the form $2p_1 + 2p_2$ is cut out by a line. Here nice means projectively normal, i.e. canonically embedded by a complete linear series, and for every $d$, we have a surjection
\begin{align*}
    H^0(\P^2, \O_{\P^2}(d)) \tto H^0(C,\O_C(d)).
\end{align*}
Projectively normal is normal + this condition.

Hence we get a bijection between bitangents to $C$ and divisors $p_1 + p_2$ so that $2(p_1 + p_2)\sim K_C$.

If $k$ is any divisor with $2k\sim K_C$, then $2(p_1 + p_2 - k) = 0$. Hence $(p_1 + p_2) - k$ is a deg 0 divisor in $C$ and $k - (p_1 + p_2)$ has order 2.

So bitangents correspond to 2-torsion points in $\Pic^0$.

\textbf{Abel-Jacobi theorem} We have that
\begin{align*}
    \Pic^0(C) \xto{\sim} J(C),
\end{align*}
where $J(C)$ is the Jacobian
\begin{align*}
    J(C) = H^0(C,\Omega^1_C)^\ast / H_1(C;\Z).
\end{align*}
%
Explicitly, if we fix a basepoint $p_0 \in C$, we send $p - p_0$ to $\omega \mapsto \int_{p_0}^p \omega$. A priori the target depends on the path we pick between $p_0$ and $p$, however we've modded out by $H_1(C;\Z)$ so it doesn't matter.

Bitangents correspond to some 2-torsion points in $J(C) = \C^3 / \Z^6$.\footnote{Here $g = 3$ so $H^0(C,\Omega^1_C) \cong \C^3$.} Here we're looking at a torus, so we have a nice structure to work with to find 2-torsion stuff.

[pic -- in $\C/\Z$ there are four 2-torsion points]

We know the points of order 2 in $J(C)$ form a 6D vector space $\mathbb{F}_2^{6}$. This has 64 elements, so not all of them are bitangents.

How does the monodromy group act on $J(C)$? We claim it preserves the affine structure (this follows from the following: if we take four bitangents, with the property that the four lines sum to zero in the jacobian, then the eight points of tangency lie on a common conic. this geometric intersection property is preserved by monodromy ), so we get that
\begin{align*}
    \Mon(\pi) \le \text{AGL}_6(\mathbb{F}_2).
\end{align*}
%
It turns out there are further restrictions.

We have a skew-symmetric non-degenerate pairing
\begin{align*}
    \til{Q} \colon H_1(C;\Z) \times H_1(C,\Z) \to \Z,
\end{align*}
given by intersection with signs. This induces a form on half of the lattice
\begin{align*}
    4 \til{Q} \colon \frac{1}{2}\Lambda \times \frac{1}{2} \Lambda &\to \Z \\
    (v,w) &\mapsto \til{Q}(2v,2w).
\end{align*}
This is ``manually imposing'' the intersection form on the more dense lattice.

This in turn induces a form on the quotient (where $V = \frac{1}{2}\Lambda / \Lambda$ is our 6D vector space over $\mathbb{F}_2$ as before):
\begin{align*}
    Q \colon V \times V \to \Z/2.
\end{align*}
It is valued in $2\Z$, since if we input any two things from $\Lambda$ we get something even.

This is a \textit{strictly} skew-symmetric bilinear form.

This form is preserved by $\Mon(\pi)$, hence
\begin{align*}
    \Mon(\pi) \le \text{AO}_6(\mathbb{F}_2).
\end{align*}
%

Associated to any symmetric bilinear form, we will get two quadratic refinements $q^+$ and $q_-$. These are helpful in determining the \textit{effective} semicanonical divisors. That is, we can find quadratic forms satisfying $q(v) = 0$ if and only if $v$ represents a bitangent. The monodromy group will preserve this quadratic form, so we ultimately get the monodromy group is a subgroup of the \textit{Steiner group} $H$, which is $O_6(\Z/2)$.

Can show that $\Stab_{\Mon(\pi)}(k_0) = O_6^{-}(\Z/2)$. Sidhanth will explain!




\appendix
\section{Algebraic geometry terms and references}

Here are some terms and results, laid out in the order we need to use them throughout the notes.

\begin{definition}[{\cite[p.~91]{Hartshorne}}]
\label{def:generically-finite}
If $f \colon X \to Y$ is a morphism of varieties with $Y$ irreducible, we say it is \textit{generically finite} if $f^{-1}(\eta)$ is a finite set, where $\eta$ is the generic point in $Y$.
\end{definition}

\begin{remark} If $f$ is locally of finite type and qcqs, being generically finite admits some equivalent conditions (see e.g. \cite[02NW]{Stacks}).
\end{remark}

\begin{proposition}\label{prop:irred-criterion}
Let $f\colon X\to Y$ is a map with irreducible equidimensional fibers, with $Y$ irreducible. Suppose that either $X$ is equidimensional or $f$ is proper. Then $X$ is irreducible.
\end{proposition}
\begin{proof} A proof can be found in \url{https://public.websites.umich.edu/~mmustata/Note1_09.pdf}
\end{proof}




\printbibliography
\end{document}
