\documentclass[11pt]{amsart}
\usepackage[page,toc,titletoc,title]{appendix}

% Bibliography stuff
\usepackage[doi=false,isbn=false,url=false,style=alphabetic]{biblatex}
\bibliography{citationsgalois.bib}


% Packages
\usepackage{amsmath,amssymb,amsthm,amsfonts,thmtools}
\usepackage[shortlabels]{enumitem}[cleveref]
\usepackage{amsfonts}
\usepackage[margin=1in]{geometry}
\usepackage{float}

% Avoid footnote patch error
\usepackage[final,nopatch=footnote]{microtype}

\RequirePackage{color}
\RequirePackage{tikz}
\RequirePackage{tikz-cd}

% For arrows
\RequirePackage{mathtools}

% For script letters
\RequirePackage{mathrsfs}

% For boxes around cheatsheets
\usepackage{mdframed}
\mdfdefinestyle{cheatsheet}{%
    linecolor=black,
    outerlinewidth=2pt,
    roundcorner=20pt,
    innertopmargin=4pt,
    innerbottommargin=4pt,
    innerrightmargin=40pt,
    innerleftmargin=40pt,
    leftmargin = 100pt,
    rightmargin = 100pt
    backgroundcolor=gray!50!white}

% Parindent/parskip
% \setlength{\parindent}{0pt}

% Custom color names
\usepackage{xcolor}
\definecolor{darkgreen}{rgb}{0,0.30,0}
\definecolor{darkred}{rgb}{0.75,0,0}
\definecolor{darkblue}{rgb}{0,0,0.6} 
\definecolor{custompurple}{RGB}{62, 34, 127}


% Citation colors
\def\customcitecolor{darkred}
\def\customlinkcolor{darkred}

% Hyperref settings
\usepackage[%
    colorlinks,
    citecolor=\customcitecolor,%
    linkcolor=\customlinkcolor,%
    urlcolor=\customlinkcolor%
]{hyperref}

% Removes vertical spacing around aligned environments
\usepackage{etoolbox}
\newcommand{\zerodisplayskips}{%
  \setlength{\abovedisplayskip}{2pt}%
  \setlength{\belowdisplayskip}{2pt}%
  \setlength{\abovedisplayshortskip}{0pt}%
  \setlength{\belowdisplayshortskip}{0pt}}
\appto{\normalsize}{\zerodisplayskips}
\appto{\small}{\zerodisplayskips}
\appto{\footnotesize}{\zerodisplayskips}

% Removes spacing around enumerate/itemize environments
\usepackage{enumitem}
\usepackage{setspace}
\setlist[enumerate,1]{leftmargin=1cm}
\setlist[enumerate,2]{leftmargin=2cm}
\setlist[itemize,1]{leftmargin=0.5cm}
\setlist[itemize,2]{leftmargin=2cm}
\setlist{nosep} % or \setlist{noitemsep} to leave space around whole list

% Section headings
\newcommand{\sectionheader}{Lecture~\thesection:~}

% Course info
\newcommand{\theinstructor}{Thomas Brazelton}
\newcommand{\thecoursetitle}{Galois groups in enumerative geometry}
\newcommand{\thetitle}{\thecoursetitle}
\newcommand{\theterm}{Spring 2025}

% Title page
\title{\MakeUppercase{\thetitle} \\ \theterm}
\author{\theinstructor}
%\date{\theterm}

% Header
\usepackage{fancyhdr}
\pagestyle{fancy}
\fancyhf{}
\fancyhead[L]{\small\itshape\thecoursetitle}
\fancyhead[R]{\small\itshape\theterm}
\setlength{\headheight}{12.0pt}
\setlength{\footskip}{13.0pt}

% Last hacky commands
\newcommand{\todo}{\color{red}\text{todo:}\, \color{black}}
\let\minus\smallsetminus
\renewcommand{\labelitemi}{$\triangleright$}
\let\emptyset\varnothing


% Pushout, pullback
\providecommand{\po}{\arrow[ul,phantom,"\ulcorner" very near start]}
\providecommand{\pb}{\arrow[dr,phantom,"\lrcorner" very near start]}

% Overset to and from
\providecommand{\xto}[1]{\xrightarrow{#1}}
\providecommand{\from}{\leftarrow}
\providecommand{\xfrom}[1]{\overset{#1}{\leftarrow}}

% Backwards verion of mapsto
\providecommand{\mapsfrom}{\mathrel{\reflectbox{\ensuremath{\mapsto}}}}
\providecommand{\longmapsfrom}{\mathrel{\reflectbox{\ensuremath{\longmapsto}}}}

% Hook arrows
\providecommand{\hookto}{\xhookrightarrow{}}
\providecommand{\xhookto}[1]{\overset{#1}{\hookrightarrow}}
\providecommand{\hookfrom}{\xhookleftarrow{}}
\providecommand{\xhookfrom}[1]{\xhookleftarrow{#1}}

% Two-headed arrows
\providecommand{\tto}{\twoheadrightarrow}
\providecommand{\xtto}[1]{\overset{#1}{\twoheadrightarrow}}
\providecommand{\ffrom}{\twoheadleftarrow}
\providecommand{\xffrom}[1]{\overset{#1}{\ffrom}}

% For superimposing in order to get closed and open immersion arrows
\makeatletter
\providecommand{\superimpose}[2]{%
  {\ooalign{$#1\@firstoftwo#2$\cr\hfil$#1\@secondoftwo#2$\hfil\cr}}}
\makeatother
\providecommand{\smallslash}{\mbox{\tiny/}}

% Closed and open hook arrows
\providecommand{\clhookto}{\mathrel{\raisebox{0.1em}{$\mathrel{\mathpalette\superimpose{{\hspace{0.1cm}\vspace{0.1em}\smallslash}{\hookrightarrow}}}$}}}
\providecommand{\xclhook}[1]{\overset{#1}{\clhook}}
\providecommand{\clhookfrom}{\mathrel{\raisebox{0.1em}{$\mathrel{\mathpalette\superimpose{{\hspace{0.1cm}\vspace{0.1em}\smallslash}{\hookleftarrow}}}$}}}
\providecommand{\ohookto}{\mathrel{\raisebox{0.03em}{$\mathrel{\mathpalette\superimpose{{\hspace{0.1cm}\vspace{0.03em}\mbox{\small$\circ$}}{\hookrightarrow}}}$}}}
\providecommand{\ohookfrom}{\mathrel{\raisebox{0.03em}{$\mathrel{\mathpalette\superimpose{{\hspace{0.1cm}\vspace{0.03em}\mbox{\small$\circ$}}{\hookleftarrow}}}$}}}

% Arrows with tails
\providecommand{\cofto}{\rightarrowtail}
\providecommand{\coffrom}{\leftarrowtail}
\providecommand{\xcofto}[1]{\overset{#1}{\cofto}}
\providecommand{\xcoffrom}[1]{\overset{#1}{\coffrom}}

% Dashed arrows
\providecommand{\dashto}{\dashrightarrow}
\providecommand{\dashfrom}{\dashleftarrow}

% better spacing colon for right adjoints
\newcommand\noloc{%
   \nobreak
   \mspace{6mu plus 1mu}
   {:}
   \nonscript\mkern-\thinmuskip
   \mathpunct{}
   \mspace{2mu}
}

% Squiggle arrows
\providecommand{\sqto}{\rightsquigarrow}
\providecommand{\sqfrom}{\mathrel{\reflectbox{\ensuremath{\sqto}}}}


%%%%%%%%%%%%%
% Text commands
\providecommand{\ab}{\mathrm{ab}}
\providecommand{\alg}{\mathrm{alg}}
\providecommand{\an}{\mathrm{an}}
\providecommand{\ann}{\mathrm{ann}}
\providecommand{\Aut}{\mathrm{Aut}}
\providecommand{\BG}{\mathrm{BG}}
\providecommand{\BGL}{\mathrm{BGL}}
\providecommand{\Bl}{\mathrm{Bl}}
\providecommand{\BO}{\mathrm{BO}}
\providecommand{\BP}{\mathrm{BP}}
\providecommand{\BSL}{\mathrm{BSL}}
\providecommand{\BSO}{\mathrm{BSO}}
\providecommand{\BSp}{\mathrm{BSp}}
\providecommand{\BSU}{\mathrm{BSU}}
\providecommand{\BU}{\mathrm{BU}}
\providecommand{\can}{\mathrm{can}}
\providecommand{\cd}{\mathrm{cd}}
\providecommand{\cdh}{\mathrm{cdh}}
\providecommand{\CH}{\mathrm{CH}}
\providecommand{\Ch}{\mathrm{Ch}}
\providecommand{\cl}{\mathrm{cl}}
\providecommand{\codim}{\mathrm{codim}}
\providecommand{\codom}{\mathrm{codom}}
\providecommand{\coeq}{\mathrm{coeq}}
\providecommand{\coev}{\mathrm{coev}}
\providecommand{\cof}{\mathrm{cof}}
\providecommand{\cofib}{\mathrm{cofib}}
\providecommand{\coker}{\mathrm{coker}}
\providecommand{\colim}{\mathrm{colim}}
\providecommand{\cone}{\mathrm{cone}}
\providecommand{\conj}{\mathrm{conj}}
\providecommand{\const}{\mathrm{const}}
\providecommand{\cyc}{\mathrm{cyc}}
\providecommand{\diag}{\mathrm{diag}}
\providecommand{\dg}{\mathrm{dg}}
\providecommand{\Disc}{\mathrm{Disc}}
\providecommand{\disc}{\mathrm{disc}}
\providecommand{\dual}{\mathrm{dual}}
\providecommand{\eff}{\mathrm{eff}}
\providecommand{\EKL}{\mathrm{EKL}}
\providecommand{\End}{\mathrm{End}}
\providecommand{\eq}{\mathrm{eq}}
\providecommand{\ess}{\mathrm{ess}}
\providecommand{\et}{\mathrm{et}}
\providecommand{\Et}{\mathrm{Et}}
\providecommand{\EU}{\mathrm{EU}}
\providecommand{\ev}{\mathrm{ev}}
\providecommand{\Ex}{\mathrm{Ex}}
\providecommand{\ex}{\mathrm{ex}}
\providecommand{\Exc}{\mathrm{Exc}}
\providecommand{\Ext}{\mathrm{Ext}}
\providecommand{\fib}{\mathrm{fib}}
\providecommand{\Fix}{\mathrm{Fix}}
\providecommand{\fppf}{\mathrm{fppf}}
\providecommand{\fpqc}{\mathrm{fpqc}}
\providecommand{\Frac}{\mathrm{Frac}}
\providecommand{\Frob}{\mathrm{Frob}}
\providecommand{\Fun}{\mathrm{Fun}}
\providecommand{\Gal}{\mathrm{Gal}}
\providecommand{\gen}{\mathrm{gen}}
\providecommand{\GL}{\mathrm{GL}}
\providecommand{\gp}{\mathrm{gp}}
\providecommand{\Gr}{\mathrm{Gr}}
\providecommand{\gr}{\mathrm{gr}}
\providecommand{\GW}{\mathrm{GW}}
\providecommand{\Her}{\mathrm{Her}}
\providecommand{\Ho}{\mathrm{Ho}}
\providecommand{\hocofib}{\mathrm{hocofib}}
\providecommand{\hocolim}{\mathrm{hocolim}}
\providecommand{\hofib}{\mathrm{hofib}}
\providecommand{\holim}{\mathrm{holim}}
\providecommand{\Hom}{\mathrm{Hom}}
\providecommand{\htp}{\mathrm{htp}}
\providecommand{\id}{\mathrm{id}}
\providecommand{\Idem}{\mathrm{Idem}}
\providecommand{\im}{\mathrm{im}}
\providecommand{\incl}{\mathrm{incl}}
\providecommand{\Ind}{\mathrm{Ind}}
\providecommand{\ind}{\mathrm{ind}}
\providecommand{\inj}{\mathrm{inj}}
\providecommand{\Inn}{\mathrm{Inn}}
\providecommand{\inv}{\mathrm{inv}}
\providecommand{\iso}{\mathrm{iso}}
\providecommand{\Jac}{\mathrm{Jac}}
\providecommand{\KGL}{\mathrm{KGL}}
\providecommand{\kgl}{\mathrm{kgl}}
\providecommand{\KH}{\mathrm{KH}}
\providecommand{\KO}{\mathrm{KO}}
\providecommand{\ko}{\mathrm{ko}}
\providecommand{\KQ}{\mathrm{KQ}}
\providecommand{\kq}{\mathrm{kq}}
\providecommand{\KR}{\mathrm{KR}}
\providecommand{\KSp}{\mathrm{KSp}}
\providecommand{\KU}{\mathrm{KU}}
\providecommand{\ku}{\mathrm{ku}}
\providecommand{\Lan}{\mathrm{Lan}}
\providecommand{\Map}{\mathrm{Map}}
\providecommand{\map}{\mathrm{map}}
\providecommand{\MGL}{\mathrm{MGL}}
\providecommand{\MO}{\mathrm{MO}}
\providecommand{\Mor}{\mathrm{Mor}}
\providecommand{\mor}{\mathrm{mor}}
\providecommand{\mot}{\mathrm{mot}}
\providecommand{\MSL}{\mathrm{MSL}}
\providecommand{\MSLc}{\mathrm{MSL}^{\mathrm{c}}}
\providecommand{\MSO}{\mathrm{MSO}}
\providecommand{\MSp}{\mathrm{MSp}}
\providecommand{\MSU}{\mathrm{MSU}}
\providecommand{\MU}{\mathrm{MU}}
\providecommand{\mult}{\mathrm{mult}}
\providecommand{\Nis}{\mathrm{Nis}}
\providecommand{\ob}{\mathrm{ob}}
\providecommand{\obj}{\mathrm{obj}}
\providecommand{\op}{\mathrm{op}}
\providecommand{\Orb}{\mathrm{Orb}}
\providecommand{\ord}{\mathrm{ord}}
\providecommand{\Out}{\mathrm{Out}}
\providecommand{\perf}{\mathrm{perf}}
\providecommand{\Perm}{\mathrm{Perm}}
\providecommand{\PGL}{\mathrm{PGL}}
\providecommand{\Pic}{\mathrm{Pic}}
\providecommand{\pr}{\mathrm{pr}}
\providecommand{\pre}{\mathrm{pre}}
\providecommand{\Prin}{\mathrm{Prin}}
\providecommand{\Proj}{\mathrm{Proj}}
\providecommand{\proj}{\mathrm{proj}}
\providecommand{\PSL}{\mathrm{PSL}}
\providecommand{\quot}{\mathrm{quot}}
\providecommand{\Ran}{\mathrm{Ran}}
\providecommand{\rank}{\mathrm{rank}}
\providecommand{\Res}{\mathrm{Res}}
\providecommand{\RO}{\mathrm{RO}}
\providecommand{\sep}{\mathrm{sep}}
\providecommand{\sgn}{\mathrm{sgn}}
\providecommand{\SH}{\mathrm{SH}}
\providecommand{\sig}{\mathrm{sig}}
\providecommand{\Sing}{\mathrm{Sing}}
\providecommand{\SL}{\mathrm{SL}}
\providecommand{\SO}{\mathrm{SO}}
\providecommand{\soc}{\mathrm{soc}}
\providecommand{\Sp}{\mathrm{Sp}}
\providecommand{\Span}{\mathrm{Span}}
\providecommand{\Spec}{\mathrm{Spec}\hspace{0.1em}}
\providecommand{\Spin}{\mathrm{Spin}}
\providecommand{\spn}{\mathrm{spn}}
\providecommand{\Sq}{\mathrm{Sq}}
\providecommand{\st}{\mathrm{st}}
\providecommand{\Stab}{\mathrm{Stab}}
\providecommand{\SU}{\mathrm{SU}}
\providecommand{\supp}{\mathrm{supp}}
\providecommand{\Syl}{\mathrm{Syl}}
\providecommand{\syl}{\mathrm{syl}}
\providecommand{\Sym}{\mathrm{Sym}}
\providecommand{\syn}{\mathrm{syn}}
\providecommand{\SYT}{\mathrm{SYT}}
\providecommand{\TC}{\mathrm{TC}}
\providecommand{\td}{\mathrm{td}}
\providecommand{\Th}{\mathrm{Th}}
\providecommand{\THH}{\mathrm{THH}}
\providecommand{\Tor}{\mathrm{Tor}}
\providecommand{\TP}{\mathrm{TP}}
\providecommand{\TR}{\mathrm{TR}}
\providecommand{\Tr}{\mathrm{Tr}}
\providecommand{\tr}{\mathrm{tr}}
\providecommand{\univ}{\mathrm{univ}}
\providecommand{\veff}{\mathrm{veff}}
\providecommand{\vol}{\mathrm{vol}}
\providecommand{\Wel}{\mathrm{Wel}}
\providecommand{\Wr}{\mathrm{Wr}}
\providecommand{\Zar}{\mathrm{Zar}}

% Special text commands
\providecommand{\et}{\text{\'{e}t}}
\renewcommand{\Im}{\mathrm{Im}}
\renewcommand{\Re}{\mathrm{Re}}
\providecommand{\Spec}{\text{Spec}\hspace{0.1em}}
\providecommand{\spn}{\text{span}}

% Blackboard letters
\providecommand{\A}{\mathbb{A}}
\providecommand{\C}{\mathbb{C}}
\providecommand{\F}{\mathbb{F}}
\providecommand{\G}{\mathbb{G}}
\providecommand{\H}{\mathbb{H}}
\providecommand{\N}{\mathbb{N}}
\providecommand{\P}{\mathbb{P}}
\providecommand{\Q}{\mathbb{Q}}
\providecommand{\R}{\mathbb{R}}
\providecommand{\Z}{\mathbb{Z}}

% Categories
\providecommand{\Ab}{\mathrm{Ab}}
\providecommand{\Alg}{\mathrm{Alg}}
\providecommand{\Ani}{\mathrm{Ani}}
\providecommand{\Bimod}{\mathrm{Bimod}}
\providecommand{\CAlg}{\mathrm{CAlg}}
\providecommand{\Cat}{\mathrm{Cat}}
\providecommand{\CDGA}{\mathrm{CDGA}}
\providecommand{\CG}{\mathrm{CG}}
\providecommand{\CGWH}{\mathrm{CGWH}}
\providecommand{\Ch}{\mathrm{Ch}}
\providecommand{\CMon}{\mathrm{CMon}}
\providecommand{\coAlg}{\mathrm{coAlg}}
\providecommand{\Coh}{\mathrm{Coh}}
\providecommand{\CommRing}{\mathrm{CommRing}}
\providecommand{\ConjSub}{\mathrm{ConjSub}}
\providecommand{\coMod}{\mathrm{coMod}}
\providecommand{\Cor}{\mathrm{Cor}}
\providecommand{\Corr}{\mathrm{Corr}}
\providecommand{\CoSh}{\mathrm{CoSh}}
\providecommand{\CRing}{\mathrm{CRing}}
\providecommand{\CW}{\mathrm{CW}}
\providecommand{\Field}{\mathrm{Field}}
\providecommand{\Fin}{\mathrm{Fin}}
\providecommand{\FinSet}{\mathrm{FinSet}}
\providecommand{\Gpd}{\mathrm{Gpd}}
\providecommand{\Grp}{\mathrm{Grp}}
\providecommand{\Grpd}{\mathrm{Grpd}}
\providecommand{\Grph}{\mathrm{Grph}}
\providecommand{\Kan}{\mathrm{Kan}}
\providecommand{\Kar}{\mathrm{Kar}}
\providecommand{\LMod}{\mathrm{LMod}}
\providecommand{\Mfld}{\mathrm{Mfld}}
\providecommand{\Mod}{\mathrm{Mod}}
\providecommand{\NAlg}{\mathrm{NAlg}}
\providecommand{\Ouv}{\mathrm{Ouv}}
\providecommand{\Perf}{\mathrm{Perf}}
\providecommand{\Poset}{\mathrm{Poset}}
\providecommand{\Pr}{\mathrm{Pr}}
\providecommand{\Pre}{\mathrm{Pre}}
\providecommand{\PSh}{\mathrm{PSh}}
\providecommand{\PShv}{\mathrm{PShv}}
\providecommand{\qCat}{\mathrm{qCat}}
\providecommand{\QCoh}{\mathrm{QCoh}}
\providecommand{\Rep}{\mathrm{Rep}}
\providecommand{\Ring}{\mathrm{Ring}}
\providecommand{\RMod}{\mathrm{RMod}}
\providecommand{\sAb}{\mathrm{sAb}}
\providecommand{\Sch}{\mathrm{Sch}}
\providecommand{\Set}{\mathrm{Set}}
\providecommand{\SH}{\mathrm{SH}}
\providecommand{\Sh}{\mathrm{Sh}}
\providecommand{\Shv}{\mathrm{Shv}}
\providecommand{\Sm}{\mathrm{Sm}}
\providecommand{\Sp}{\mathrm{Sp}}
\providecommand{\Spectra}{\mathrm{Spectra}}
\providecommand{\Spc}{\mathrm{Spc}}
\providecommand{\sPre}{\mathrm{sPre}}
\providecommand{\Spt}{\mathrm{Spt}}
\providecommand{\sSet}{\mathrm{sSet}}
\providecommand{\sShv}{\mathrm{sShv}}
\providecommand{\Stack}{\mathrm{Stack}}
\providecommand{\Sub}{\mathrm{Sub}}
\providecommand{\Top}{\mathrm{Top}}
\providecommand{\Tors}{\mathrm{Tors}}
\providecommand{\Var}{\mathrm{Var}}
\providecommand{\Vect}{\mathrm{Vect}}

%%%%%%%%%%%%
% category_theory
% For blackboard bold number and delta categories
\RequirePackage{bbm}
\providecommand{\onecat}{\mathbbm{1}}
\providecommand{\twocat}{\mathbbm{2}}

% Blackboard delta
\RequirePackage{pict2e,picture}

\makeatletter
\DeclareRobustCommand{\DDelta}{{\mathpalette\bb@Delta\relax}}
\newcommand{\bb@Delta}[2]{%
  \begingroup
  \sbox\z@{$\m@th#1\Delta$}%
  \dimendef\Dht=6 \dimendef\Dwd=8
  \setlength{\Dwd}{\wd\z@}%
  \setlength{\Dht}{\ht\z@}%
  \begin{picture}(\Dwd,\Dht)
  \put(0,0){$\m@th#1\Delta$}
  \put(.42\Dwd,.7\Dht){\line(10,-26){.25\Dht}}
  \end{picture}%
  \endgroup
}

% Heart (for e.g. t-structures)
\usepackage{graphicx}
\newcommand{\heart}{\ensuremath\heartsuit}

% Other
\providecommand{\HZ}{\mathrm{H}\mathbb{Z}}
\providecommand{\Gm}{\mathbb{G}_m}
% Spaces
\providecommand{\CP}{{\mathbb{C}\text{P}}}
\providecommand{\HP}{{\mathbb{H}\text{P}}}
\providecommand{\RP}{{\mathbb{R}\text{P}}}

\renewcommand{\O}{\mathcal{O}}
\renewcommand{\P}{\mathbb{P}}


\usepackage{cleveref}
\let\fullref\autoref
%
\def\makeautorefname#1#2{\expandafter\def\csname#1autorefname\endcsname{#2}}
%  
\makeautorefname{eqn}{Equation}%
\makeautorefname{sec}{Section}%
\makeautorefname{subsec}{Subsection}%
\makeautorefname{footnote}{footnote}%
\makeautorefname{item}{item}%
\makeautorefname{figure}{Figure}%
\makeautorefname{table}{Table}%
\makeautorefname{wraptab}{wraptable}%
\makeautorefname{part}{Part}%
\makeautorefname{app}{Appendix}%
\makeautorefname{cla}{claim}%
\makeautorefname{ans}{answer}%
\makeautorefname{assump}{assumption}%
\makeautorefname{conj}{conjecture}%
\makeautorefname{cor}{corollary}%
\makeautorefname{cex}{counterexample}%
\makeautorefname{cexs}{counterexamples}%
\makeautorefname{dig}{digression}%
\makeautorefname{disc}{discussion}%
\makeautorefname{def}{definition}%
\makeautorefname{ex}{example}%
\makeautorefname{exs}{examples}%
\makeautorefname{fac}{fact}%
\makeautorefname{goal}{goal}%
\makeautorefname{intu}{intuition}%
\makeautorefname{lem}{lemma}%
\makeautorefname{meta}{metathm}%
\makeautorefname{motiv}{motivation}%
\makeautorefname{nota}{notation}%
\makeautorefname{note}{note}%
\makeautorefname{warn}{warning}%
\makeautorefname{prop}{proposition}%
\makeautorefname{ques}{question}%
\makeautorefname{rmk}{remark}%
\makeautorefname{set}{setup}%
\makeautorefname{strat}{strategy}%
\makeautorefname{term}{terminology}%
\makeautorefname{thm}{theorem}%
\makeautorefname{upsh}{upshot}%
%
%                  *** End of hyperref stuff ***

\theoremstyle{definition}
\newtheorem{theorem}{Theorem}[section]
\numberwithin{theorem}{section} % important bit
\newtheorem{answer}[theorem]{Answer}
\newtheorem{assumption}[theorem]{Assumption}
\newtheorem{claim}[theorem]{Claim}
\newtheorem{conjecture}[theorem]{Conjecture}
\newtheorem{corollary}[theorem]{Corollary}
\newtheorem{counterexample}[theorem]{Counterexample}
\newtheorem{definition}[theorem]{Definition}
\newtheorem{digression}[theorem]{Digression}
\newtheorem{discussion}[theorem]{Discussion}
\newtheorem{example}[theorem]{Example}
\newtheorem{examples}[theorem]{Examples}
\newtheorem{exercise}[theorem]{Exercise}
\newtheorem{fact}[theorem]{Fact}
\newtheorem{goal}[theorem]{Goal}
\newtheorem{idea}[theorem]{Idea}
\newtheorem{intuition}[theorem]{Intuition}
\newtheorem{lemma}[theorem]{Lemma}
\newtheorem{metathm}[theorem]{Meta-theorem}
\newtheorem{motivation}[theorem]{Motivation}
\newtheorem{notation}[theorem]{Notation}
\newtheorem{note}[theorem]{Note}
\newtheorem{proposition}[theorem]{Proposition}
\newtheorem{question}[theorem]{Question}
\newtheorem{remark}[theorem]{Remark}
\newtheorem{setup}[theorem]{Setup}
\newtheorem{strategy}[theorem]{Strategy}
\newtheorem{terminology}[theorem]{Terminology}
\newtheorem{upshot}[theorem]{Upshot}
\newtheorem{warning}[theorem]{Warning}

%%%% hack to get fullref working correctly
\makeatletter
\let\c@corollary=\c@theorem
\let\c@proposition=\c@theorem
\let\c@lemma=\c@theorem
\let\c@assumption=\c@theorem
\let\c@conjecture=\c@theorem
\let\c@definition=\c@theorem
\let\c@example=\c@theorem
\let\c@remark=\c@theorem
\let\c@notation=\c@theorem
\let\c@equation\c@theorem
\let\c@strategy\c@theorem
\makeatother

\renewcommand*{\subsectionautorefname}{Subsection}
\renewcommand*{\sectionautorefname}{Section}


\usepackage{epigraph}

% For editing purposes, disable on commit
%\usepackage{showkeys}

\DeclareFieldFormat{postnote}{#1}
\DeclareFieldFormat{multipostnote}{#1}

\def\theshiftamount{2}
\let\del\partial
\let\til\widetilde
\let\nsubgp\trianglelefteq
\let\smashprod\wedge

% get rid of fancyhdr errors
\setlength{\footskip}{13.6pt}
\setlength{\parskip}{0.5em}


% Custom quote block command
\setlength{\epigraphwidth}{0.8\textwidth}

\providecommand{\ASL}{\mathrm{ASL}}
\providecommand{\Bitan}{\mathrm{Bitan}}
\providecommand{\Conf}{\mathrm{Conf}}
\providecommand{\Flex}{\mathrm{Flex}}
\providecommand{\Flextan}{\mathrm{Flextan}}
\providecommand{\HypFlex}{\mathrm{HypFlex}}
\providecommand{\Mon}{\mathrm{Mon}}
\providecommand{\NS}{\mathrm{NS}}
\providecommand{\res}{\mathrm{res}}
\providecommand{\Tan}{\mathrm{Tan}}
\providecommand{\UConf}{\mathrm{UConf}}

\fancyfoot[C]{\thepage}
\begin{document}

\begin{abstract} \href{https://github.com/tbrazel/galois-notes}{https://github.com/tbrazel/galois-notes}
\end{abstract}

\maketitle

% \setcounter{tocdepth}{1}
% \tableofcontents{}


\setcounter{section}{-1}
\section{About}

Some notes from a mini-seminar run in Spring 2025 on Galois groups of enumerative problems, following Joe Harris' 1979 paper of the same name \cite{Harris-Galois}. These notes aren't intended to be a definitive reference, but are perhaps more narrow than the survey paper \cite{SottileYahl}, for instance. Our goals are to fill out details and examples to better understand how to carry out computations in monodromy and enumerative geometry.

\subsection{Notation}

We modify some notation slightly (e.g. $\Gr(k,n)$ instead of $G(k,n)$ for the Grassmannian of affine $k$-planes in affine $n$-space), but for the most part follow closely with the notation in \cite{Harris-Galois}. Some exceptions include using $\Flex_{(p_0,\ell_0)}$ and $\HypFlex_{(p_0,\ell_0)}$ instead of $W'$ and $W''$ for the locus of degree $d$ curves with a fixed flex or hyperflex in \Cref{sec:flexes}

\section{Solvability}

A foundational question in algebra is the following.

\begin{question}\label{solvable-bivariate-polynomial} 
Given a polynomial equation $F(u,z)=0$, can we express $u$ as a function in terms of $z$?
\end{question}

Locally we know the answer to be yes via the inverse function theorem, but we are interested in the shape of the equation -- did we need square roots? Cube roots? Rational functions? A basic example comes in the case where $F(u,z) = z-f(u)$. Then we are asking how to go from $z = f(u)$ to an expression for $u$ in terms of $z$. This is generally done via Galois theory. If $K$ is our base field, we consider the extension
\begin{align*}
    K(z) \hookto \frac{K(z)[u]}{(z-f(u))}.
\end{align*}
We are interested in whether $u$ can be expressed in terms of $z$, and via a Hilbert irreducibility argument, this is tantamount to asking whether $u$ can be expressed in terms of elements in the base field $K$ in the extension
\begin{align*}
    K \hookto \frac{K[u]}{f(u)-\zeta},
\end{align*}
obtained by specializing $z$ to some value $\zeta\in K$. This question we know to be approachable from the perspective of Galois theory.

\begin{theorem}[Galois] If $L :=K[u]/f(u)$ is a separable extension of $K$, then $u$ is solvable in radicals if and only the Galois group $\Gal(L/ K)$ is a solvable group.
\end{theorem}

\subsection{From Galois groups of fields to enumerative problems}

In the early decades of Galois theory, mathematicians were interested in generalizing \Cref{solvable-bivariate-polynomial} away from the setting of univariate expressions $z = f(u)$, and towards more general polynomials $F(u,z) = 0$. A pillar in this direction is Hermite's work on algebraic functions \cite{Hermite51}, which could be considered the genesis of the alignment between Galois groups and monodromy groups. We recount his story in the notation of that paper but in contemporary language.

Consider $Y = V(F(u,z))$ as a subvariety of $\P^1 \times \P^1$, and let $\pi \colon Y \to \P^1$ be the projection onto the $z$-coordinate. Over a point $\zeta\in \P^1$, the fibers $\pi^{-1}(\zeta)$ are precisely the roots of the univariate polynomial $F(u,\zeta)=0$. Assuming that $F$ is homogeneous of degree $m$, we have that $\pi$ is generically $m$-to-$1$.

The projection $\pi$ has some ramification locus, however, and we denote this by
\begin{align*}
    \left\{ z_0, z_1, \ldots, z_{\mu-1} \right\} \subseteq \P^1.
\end{align*}
These are precisely the values of $z$ for which $F(u,z_i)$ has repeated roots. Pick some point $P \in \P^1$, not in the ramification locus, and let
\begin{align*}
    \pi^{-1}(P) = \left\{ u_0, u_1, \ldots, u_{m-1} \right\}.
\end{align*}
Note that any class in $\pi_1(\P^1\minus \{z_0, \ldots, z_{\mu-1}\},P)$ induces a permutation on the roots $\left\{ u_0, \ldots, u_{m-1} \right\}$. This is called the \textit{monodromy action}. Hermite argues the following.\footnote{Hermite \textit{proves} this in his paper, but it's a matter of opinion whether the two paragraphs of proof hold up to modern standards of rigor.}

\begin{theorem}[{\cite{Hermite51}}]
A function in the roots $u$ is a rational function in $z$ if and only if it is invariant under the permutations obtained from lifting paths in $\pi_1(\P^1\minus \{z_0, \ldots, z_{\mu-1}\},P)$ (cf.~\cite{Harris-Galois}).
\end{theorem}
We will discuss a modern proof of this in a much more general setting following Harris (\Cref{prop:gal-equals-mon}). In the years following Hermite, it became clear that these questions could be approached in the setting where we might have more variables and more equations. This leads us to the study of Galois groups of a \textit{system} of equations. A natural source for such systems of equations came from the burgeoning field of enumerative algebraic geometry, and Jordan was the first to envision how Galois theory would be used here.

\subsection{Galois groups in enumerative geometry, a la Jordan}

\begin{question} Given an enumerative problem, to what field extension do we need to pass to in order to find all its solutions?
\end{question}

This question has its roots in 19th century algebraic geometry, and was crystallized by Jordan in Chapter III of his treatise on Galois theory:


\quoteblock{%
Les solutions des problèmes dont il s'agit sont en général assez nombreuses, mais liées les unes aux autres par certaines relations géometriques. De ces relations on déduit immédiatement l'existence d'une fonction entière $\phi(x_0,x_1,\ldots)$ dont le group contient celui de l'équation $X$. Réciproquement, si l'on était certain de connaître \emph{toutes} les relations géométriques que présente la question proposée (ou du moins celles dont les autres dérivent), le group de l'équation $X$ contiendrait toutes les substitutions du group de $\phi(x_0,x_1,\ldots)$. Mais une semblable certitude est difficile à obtenir, malgré le soin apporté par d'habiles géomètres à l'étude de ces problèmes. Il ne serait donc pas impossible que les équations auxquelles ces problèmes donnent naissance eussent parfois une forme plus particulière encore que celle que nous allons trouver, en nous appuyant sur les résultats obtenus par nos prédécesseurs}{%
Camille Jordan, 1870, \cite[pp.301-302]{Jordan}}

\begin{question} Are these sorts of questions \textit{solvable}? Meaning solvable in radicals?
\end{question}


The six subsections of Chapter III in Jordan's book are dedicated to particular problems whose Galois group he is interested in studying. For instance:
\begin{itemize}
    \item[III.I] (\Cref{sec:flexes}) Given a planar cubic curve, we know by work of Hesse that it has the property that the line passing through any two of its flexes pass through the cubic at a third inflection point. As there are nine flexes, each flex lies on four of these lines, for a total of 12 lines.\footnote{I'd love to include a nice picture here, but unfortunately by a result of Klein, a smooth real planar cubic will have at most three of its nine flexes defined over $\R$ \cite{Ronga-Klein}.} \emph{Can we obtain a formula for the flexes in terms of the coefficients of the cubic?}

    \item[III.II] Given a quartic curve, can we find a cubic curve so that among the 12 points of intersection, they are in three sets of four colinear points? This follows work of Clebsch, who asked \textit{contact problems} of a similar flavor (see \cite[(429)]{Jordan}).

    \item[III.III] Again following work of Clebsch, given a quartic surface with a double conic, we can find five cones whose edges are bitangent to the surface. There are 16 lines on the quartic surface, each of which intersect the doubled conic at a single point. Can we solve for the lines given the quartic surface or the cones?
    
    \item[III.IV] Kummer showed there exist quartic surfaces with 16 singular points, which lie in groups of six on singular tangent planes, each of which intersect other tangent planes at these six points. Can we solve for these points (or these planes) given the Kummer surface?\footnote{This configuration admits a really nice contemporary description in terms of $\theta$-characteristics. We should come back and fill out the details here.}

    \item[III.V] Following Steiner,\footnote{%
    %
    The Steiner reference is his 1857 paper \textit{Über die Flächen dritten Grades} (On cubic surfaces) \cite{Steiner-cubic-surfaces}, published in Crelle's Journal, which was called Borchardt's Journal during Borchardt's tenure as editor (1856--1880). It's interesting to me that Jordan references work of Steiner, rather than work of Cayley and Salmon, from 1849 and 1847, respectively.} every smooth cubic surface has 27 lines. These lines lie on 45 tritangents, and two tritangent planes always intersect at some line, not necessarily a line on the cubic surface though. If two tritangent planes $a_1b_1c_1$ and $a_2b_2c_2$ don't meet at a line on a cubic surface, then there exists another tritangent plane $a_3b_3c_3$ for which $a_1a_2a_3$, $b_1b_2b_3$, and $c_1c_2c_3$ form tritangents. These are called \textit{trihedral pairs}, and there are 120 of them (see e.g. \cite[p.~112]{Hunt}). Can we solve for the equations of the lines given the equation of the cubic surface? Can we incorporate the constraints imposed by the trihedral pairs?

    \item[III.VI] Again following Clebsch, fix a curve $C$ of order $n$ and $\frac{n(n-3)}{2}$ points on it. Can we determine the curves of order $n-3$ intersecting $C$? For example if $n=4$, can we determine all the 28 bitangents to $C$?
\end{itemize}

With this in mind, let's see how to generalize Hermite's result and give a contemporary proof.

\subsection{Setup and goal}

Let $X$ and $Y$ be irreducible algebraic varieties of the same dimension over $\C$, let $\pi \colon Y \to X$ have degree $d>0$. This induces a map on function fields $K(X) \to K(Y)$ which is a finite field extension.\footnote{If $X$ and $Y$ are affine, it's clear this function field extension is finite. In the general case, we can reduce to the affine case by looking at the generic points (since they're irreducible), see e.g. \cite[02NW]{Stacks}.} and necessarily separable since we are in characteristic zero.

We're going to pick some nice $p \in X$ (in the region over which $\pi$ is finite), and look at its fibers $\pi^{-1}(p) = \left\{ q_1, \ldots, q_d \right\}$. We'll define two ways to permute the fibers -- one coming from Galois theory and one coming from monodromy, and we'll demonstrate that these are equal, following \cite[\S1]{Harris-Galois}.

\subsection{The Galois group, formally}

By the primitive element theorem, $K(Y)$ is generated over $K(X)$ by some rational function $f \in K(Y)$ which satisfies a minimal polynomial relation:
\begin{align*}
    P(f) = f^d + g_1 f^{d-1} + \ldots + g_{d-1}f + g_d = 0,
\end{align*}
where $g_1, \ldots, g_d\in K(X)$.

Recall that $\O_X$ is the sheaf of holomorphic functions. We obtain the sheaf $\mathcal{K}_X$ of meromorphic functions as the quotient ring. We get an injection of sheaves $\O_X \to \mathcal{K}_X$. In the general scheme-theoretic setup, this need not be a field, however since our $X$ is particularly nice (integral, Noetherian,..) it will be. So we can look at the germs of meromorphic functions around $p$, which forms a \textit{field} $\mathcal{K}_{X,p}$. The covering map induces an isomorphism of fields at each $q_\alpha$:
\begin{align*}
    \pi_\alpha := \pi_\ast \colon K_{Y,q_\alpha} \xto{\sim} K_{X,p}.
\end{align*}
Let's define $\phi$ to be the inclusion of fields obtained by restricting global meromorphic functions around $p$:
\begin{align*}
    \phi \colon K(X) \hookto \mathcal{K}_{X,p},
\end{align*}
and $\phi_\alpha$ to be the composite
\begin{align*}
    K(Y) \to \mathcal{K}_{Y,q_\alpha} \xto{\pi_\alpha} \mathcal{K}_{X,p}.
\end{align*}
%
We can sit everything inside $\mathcal{K}_{X,p}$, so let's fix some notation:
\begin{enumerate}
    \item $K$ is the image of $K(X)$ in $\mathcal{K}_{X,p}$ -- it is the restriction of global meromorphic functions on $X$ to a neighborhood of $p$
    \item $L$ is the subfield of $\mathcal{K}_{X,p}$ generated by all the images of the $\phi_\alpha$'s. --- this is meromorphic functions around $p$, which are coming from meromorphic functions on $Y$ that have been restricted to some neighborhood of some $q_\alpha$
\end{enumerate}
Finally we take our elements $f \in K(Y)$ and $g_i \in K(X)$ and sit them inside the larger field:
\begin{align*}
    \phi \colon K(X) &\to K \subseteq\mathcal{K}_{X,p} \\
    g_i &\mapsto \til{g}_i.
\end{align*}
and
\begin{align*}
    \phi_\alpha \colon K(Y) &\to \mathcal{K}_{X,p} \\
    f &\mapsto \til{f}_\alpha.
\end{align*}
%
Observe that all the $\til{f}_\alpha$'s are distinct! This is because in order for $f$ to generate $K(Y)$ over $K(X)$, it must take different values at each $q_\alpha$.

%
Moreover, each of the $\til{f}_\alpha$'s satisfy the image of the polynomial relation $P(f) = 0$ in $L$:
\begin{align*}
    \til{P}(\til{f}_\alpha) &= \til{f}_\alpha^d + \til{g}_1 \til{f}_\alpha^{d-1} + \ldots + \til{g}_d = 0.
\end{align*}
%
Let's think about $\til{P}$ as a polynomial in $L[t]$. It is a degree $d$ polynomial with $d$ distinct roots, given by the $\til{f}_\alpha$'s.

We claim then that $L$ is the normalization of $K(Y)/K(X)$, (which is identically $K_\alpha/K$. This is because the minimal polynomial $P(t)$ splits in $L$, and $L$ is the minimal field over which this occurs. So we have argued:

\begin{proposition} $\Gal(L/K)$ is Galois.
\end{proposition}

The action of the Galois group permutes all the $\til{f}_\alpha$'s, which permutes the indices $\alpha$, giving us an inclusion
\begin{align*}
    \Gal(L/K) \hookto \Sigma_d.
\end{align*}
The image of this is the \textit{Galois group} of the enumerative problem.

\subsection{The monodromy group}

Since $\pi \colon Y \to X$ is an branched cover, it satisfies homotopy lifting away from the branched points. That is, any path in $X$ which doesn't pass through the branch locus can be lifted to a path in $Y$ after a starting point has been chosen.

To that end, pick some $U \subseteq X$ Zariski open containing our $p$ and $V = \pi^{-1}(U)$ so that $\pi \colon V \to U$ is an unbranched cover. Then we obtain an inclusion
\begin{align*}
    \pi_1(U,p) \to \Sigma_d,
\end{align*}
given by the action of the deck group. The image of this is called the \textit{monodromy group} of our enumerative problem.

\subsection{Monodromy and analytic continuation}

Let $X$ be a Riemann surface, and consider a path $\gamma \colon [0,1] \to X$, and let's take two holomorphic germs $f\in \O_{X,\gamma(0)}$ and $g \in \O_{X,\gamma(1)}$. We say $g$ is the \textit{analytic continuation of $f$ along $\gamma$} if there is a finite sequence of open sets $U_i$ along the image of $\gamma$\footnote{We may assume finite since the image of $\gamma$ is compact.} and functions $f_i \in \O(U_i)$ so that $f_1 = f$ and $f_n = g$.

\begin{theorem} If $\pi \colon Y \to X$ is an unbranched cover of a Riemann surface $X$, and $\gamma \in \pi_1(X,x)$ is some loop, then for any $f \in \O_{X,x}$, any choice of lift $\widetilde{\gamma} \colon [0,1] \to Y$, and any germ $g\in \O_{Y,\widetilde{\gamma}(0)}$ with $\pi_\ast g = f$, we have that analytic continuation of $g$ along $\widetilde{\gamma}$ exists, and the resulting germ $\til{g}$ also satisfies $\pi_\ast \til{g} = f$.
\end{theorem}

\subsection{The main result}

We are going to prove the following:

\begin{proposition}\label{prop:gal-equals-mon}
The Galois group $G$ equals the monodromy group $M$ in our setup.
\end{proposition}

The first step is to argue that $M \subseteq G$. That is, any permutation coming from monodromy can be realized by an automorphism of $L$ over $K$. This follows via analytic continuation!

If we pick some $\gamma \in \pi_1(U,p)$, then any lift of $\gamma$ to $V$ will send some $\til{f}_\alpha$ to some $\til{f}_\beta$. In particular since $K = \im(K(X) \hookto \mathcal{K}_{X,p})$ is fixed under analytic continuation, and since any element in $L$ is polynomial in germs at the $q_\alpha$'s, analytic continuation induces a field automorphism of $L/K$ permuting the $\til{f}_\alpha$'s. That is, this permutation lies in the Galois group.

For the other direction, we want to see that the containment $M \subseteq G$ is not strict. Indeed if it were, then the $M$-fixed subfield $L^M$ would not be equal to $K$. So it suffices to argue that everything in $L$ fixed by the monodromy action is actually in $K$. As we have seen, the monodromy action occurs via analytic continuation.

So take some $h \in L \subseteq \mathcal{K}_{X,p}$, and suppose $h$ is fixed under analytic continuation along any lift of an element in $\pi_1(U,p)$. We want to argue that $h$ is the restriction of a global meromorphic function on $X$ to a neighborhood of $p$. We'll define a candidate one -- define $\til{h}$ on $U$ by picking, for every $p' \in U$, an arc from $p$ to $p'$ and analytically continuing $h$ along it. This is well-defined precisely because $h$ is fixed under the monodromy action, so we obtain a well-defined value in a neighborhood of $p'$, independent of the path we chose.

We now know that $h$ extends to a meromorphic function $\til{h}$ on $U$. We want to see that this extends to all of $X$. In order to do this, we exploit that $h \in L$. So $h$ can be written as some polynomial in $\til{h}_\alpha$'s, where $\til{h}_\alpha$ is a meromorphic function on all of $Y$ restricted to a neighborhood of $q_\alpha$. None of these have essential singularities, and this is unchanged when taking a polynomial in them. Hence $\til{h}$ has no essential singularities, and therefore extends to a meromorphic function on $X$ whose germ at $p$ is $h$. Thus $L^M = K$, and we are done.

\subsection{Galois groups over the rationals}

A priori this tells us nothing about Galois groups of enumerative problems over $\Q$, or field extensions of solutions, or anything like that. We can translate information happening over this complex function field to statements about rationals via \textit{Hilbert irreducibility}. As a jumping off point, let's talk about monic univariate polynomials and their Galois groups (which is the origin of this story).

\begin{example} Let $f(x) = x^n + a_1 x^{n-1} + \ldots + a_n$ be the generic monic univariate polynomial of degree $n$, and let's think about it as living in $\C(a_1, \ldots, a_n)[x]$. Then there exist \textit{rational numbers} $a_1, \ldots, a_n$ for which $f(x) = x^n + a_1 x^{n-1} + \ldots + a_n$ is irreducible in $\Q[x]$. In fact the set of points $(a_1, \ldots, a_n) \in \A^n_{\Q}$ for which this property holds is Zariski dense.
\end{example}

\begin{corollary} If $G$ is the Galois group of $\C(a_1, \ldots, a_n)[x] / f$ then $G$ is the Galois group of any specialization of $f$ to $\Q[x]$ which is irreducible. In particular, we can compute in the generic case that $G = S_n$.
\end{corollary}

This admits a generalization to polynomials in arbitrarily many variables, so we get an application which we will refer to as Hilbert irreducibility in these notes.

\begin{slogan}[Hilbert irreducibility] Let $X$ be a complex moduli space of geometric objects (e.g. cubic surfaces), and $\pi \colon Y \to X$ a generically finite map. Then the Galois group of $\pi$ is isomorphic to the Galois group of a general object in $X$ defined over the rationals.
\end{slogan}

As a particular case to illustrate what we mean, we will see that the Galois group of lines on complex cubic surfaces is $W(E_6)$. That implies that if $X$ is any randomly chosen cubic surface defined over $\Q$, the field of definition $K/\Q$ of the lines on $X$ has Galois group $\Gal(K/\Q) \cong W(E_6)$.

\subsection{An example we know and love}

Consider the incidence variety of roots of a univariate polynomial
\begin{align*}
    Y = \left\{ (f,t) \colon f(t) = 0 \right\} \subseteq \P^d \times \P^1.
\end{align*}
Then $Y \to \P^d$ is a projective bundle, whose Galois group is the full symmetric group. There are many ways to prove this, we highlight two:
\begin{enumerate}
    \item We can argue algebraically that the Galois group of the generic polynomial of degree $d$ is the full symmetric group $\Sigma_d$.
    \item We can argue that the monodromy of the cover is symmetric, e.g. by showing it is 2-transitive and contains a transposition.
\end{enumerate}

In either of these approaches, we obtain the Abel--Ruffini theorem as an immediate application via Hilbert irreducibility.

\begin{theorem} If $f(x) \in \Q[x]$ is a polynomial of degree $\ge 5$, there is no general formula in radicals for the roots of $f$ in terms of the coefficients of $f$.
\end{theorem}

\section{Flexes and bitangents}\label{sec:flexes}

\subsection{Historical background}

The study of high degree plane curves (degrees three or greater) dates back to Newton, but one of the first major results in this are came from Pl\"{u}cker (\textit{System der analytischen Geometrie}, page 264).

\begin{theorem}[Pl\"ucker, 1835] A general degree $d$ planar curve has $3d(d-2)$ inflection points.
\end{theorem}
Pl\"{u}cker proved this by equation-bashing. We provide a slightly different proof, due to Hesse but likely predating him (see \cite[p.~169]{Gray-worlds}).

\begin{proof}[Hesse's proof]
Recall that the radius of curvature of a planar curve $F(x,y,z)=0$ at a point $(x_0,y_0,z_0)$ is the reciprocal of $\det(HF)_{|(x_0,y_0,z_0)}$, where $HF$ is the Hessian matrix of $F$. Note that the equation $\det HF=0$ is homogeneous of degree $3(d-2)$, and note further that a flex on a curve is precisely a point with infinite curvature radius. Hence we can count inflection points on $F$ via the intersection with its Hessian curve, hence by B\'{e}zout's theorem we have $3d(d-2)$ flexes.
\end{proof}

The so-called \textit{Pl\"ucker formulas} are highly related, and come from the same text. Pl\"{u}cker computed what is now called the \textit{class} of the curve $C$, namely the number of tangent lines to $C$ through another point $p$ on the plane. Note that under pole-polar duality, this is also the degree of the \textit{dual curve}. He observed by direct computation that the class of a general curve of degree $d$ is $d(d-1)$.

In this direct algebraic approach, any line through $p$ passing through two points of $C$ (counted with multiplicity) qualifies as a tangent. Presuming that $C$ is smooth (and if we take $C$ to be general this is the case) then this is true on the nose, however even if $C$ is mildly singular this fails. Pl\"{u}cker observed that the quantity $d(d-1)$ overcounts the honest class of the curve by two for every double point of $C$ and by three for every cusp of $C$. This leads us to the formula
\begin{align*}
    d^\ast = d(d-1) - 2\delta - 3\kappa
\end{align*}
where $\delta$ is the number of nodes of $C$ and $\kappa$ is the number of cusps. This is the \textit{Pl\"ucker formula}, and it resolves the so-called ``duality paradox'' which plagued the study of pole-polar duality since its invention. A nice exposition to these results is in Coolidge's treatise \cite[Chapter~VI]{Coolidge}.

\begin{terminology} A reduced irreducible planar curve $C$ is said to be a \textit{Pl\"ucker curve} if it falls under the scope of the Pl\"{u}cker formula -- explicitly, if the only singularities of $C$ and $C^\ast$ are cusps and simple nodes. 
\end{terminology}


\begin{remark} \,
\begin{enumerate}
    \item More general Pl\"{u}cker formulae hold in which singularities are allowed to be much more badly behaved.
    \item Klein proved analogues of the Pl\"{u}cker formulae for real curves, treating split and non-split nodes differently. Leveraging his formula we are able to prove new results and give contemporary proofs of classical results known to Pl\"{u}cker, such as the fact that at most three of the nine flexes on a real cubic are real.
    \item Pl\"{u}cker's arguments were fairly nonrigorous, and it is ahistorical to attribute to him rigorous arguments that appear to be easy applications of his formula (for instance the computation that there are 28 bitangents to a planar quartic, attributed either to Hesse in 1848 or Jacobi in 1850).
\end{enumerate}
\end{remark}

Under duality, the Pl\"{u}cker formulae allow us to relate the numbers of bitangents and the numbers of flexes, since a bitangent to $C$ is a node on $C^\ast$, and a flex on $C$ is a cusp on $C^\ast$. If $C$ is a Pl\"{u}cker curve of degree $d$, we then have the relationship
\begin{align*}
    d = d^\ast(d^\ast-1) - 2b - 2f.
\end{align*}
Combining this with the number $3d(d-2)$ of flexes on $C$, we get
\begin{align*}
    d = d^\ast(d^\ast-1) - 2b - 2\cdot 3d(d-2).
\end{align*}
Let's now suppose that $C$ is suitably general, so that it is nonsingular and hence $d^\ast = d(d-1)$ with no correction term. We then get
\begin{align*}
    d = d(d-1)\left( d(d-1)-1 \right) - 2b - 3\cdot 3d(d-2).
\end{align*}
Solving for $b$ we obtain
\begin{equation}\label{eqn:bitangents-in-terms-of-degree}
\begin{aligned}
    b = \frac{1}{2}d(d-2)(d^2-9).
\end{aligned}
\end{equation}
This is the number of bitangents on a general smooth curve of degree $d$ as estimated by Pl\"{u}cker and proven by Jacobi.

\subsection{The Pl\"ucker formulas via Riemann-Hurwitz} TODO

\subsection{Monodromy of flexes: the setup}

Let $W_d := \mathbb{P}(H^0(\O_{\P^2}(d))) \cong \P^{\binom{d}{2}-1}$ be the complete linear system of degree $d$ plane curves, and let $I_0$ be the locus of points on lines
\begin{align*}
    I_0 := \left\{ (p,\ell) \colon p\in \ell \right\} \subseteq \P^2 \times (\P^2)^\ast.
\end{align*}
We let
\begin{align*}
    I_d := \left\{ (C,p,\ell) : m_p(C\cdot \ell)\ge 3 \right\} \subseteq W_d \times I_0
\end{align*}
be the incidence variety of curves equipped with a flex. We obtain projection maps
\[ \begin{tikzcd}
     & I_d\ar[dl,"\pi" above left]\ar[dr,"\eta" above right] & \\
    W_d &  & I_0.
\end{tikzcd} \]
%
\begin{proposition} If $d \ge 3$, then $\pi$ is generically finite of degree $3d(d-2)$.
\end{proposition}
\begin{proof} We can argue that the general curve of degree $d$ has only flexes and bitangents (it doesn't admit a hyperflex, a tritangent, or a flex bitangent). Hence over this open locus, $\pi$ has a well-defined degree. Note we restrict to $d\ge 3$, since a conic has no flexes. Obtaining the degree is exactly the Pl\"{u}cker formula, which as we have seen can be derived in multiple different ways.
\end{proof}


We want to try to find the monodromy group of $\pi$, and the following fact will be helpful.

\begin{proposition}\label{prop:Id-irred-flexes}
The incidence variety $I_d$ is irreducible.
\end{proposition}
\begin{proof} We can consider the projection $\eta \colon I_d \to I_0$. Since $I_0$ is irreducible, we will be able to conclude (by \Cref{prop:irred-criterion}) that $I_d$ is irreducible if $\eta$ has equidimensional fibers. Indeed we can check that the fiber of $\eta$ over any point $(p,\ell) \in I_0$ is exactly those degree $d$ plane curves with a flex line $\ell$ at the point $p$. This is always a codimension three linear subspace of $W_d$.
\end{proof}

Leveraging irreducibility of $I_d$, we can now argue that $\Mon(\pi)$ acts transitively on the fibers of $\pi$.

\begin{proposition}\label{prop:monodromy-flexes-transitive} 
The monodromy group $\Mon(\pi)$ acts transitively on the fibers of $\pi$ over any unbranched point in $W_d$.
\end{proposition}
\begin{proof} Let $U \subseteq W_d$ be a Zariski open set over which $\pi$ is unbranched, and hence a topological covering space. By covering space theory, the monodromy group will be transitive if and only if $\pi^{-1}(U)$ is connected. This is true because $\pi^{-1}(U) \subseteq I_d$ is a Zariski open subset of an irreducible variety (\Cref{prop:Id-irred-flexes}), hence topologically connected.\footnote{%
If not, we could write $\pi^{-1}(U)$ as the union of two disjoint Zariski open subsets, but this is a contradiction, because irreducibility of $I_d$ means any two nonempty opens necessarily intersect.} 
\end{proof}


\begin{proposition} The monodromy group $\Mon(\pi)$ is $2$-transitive.
\end{proposition}
\begin{proof} Let $(p_0,\ell_0) \in I_0$ be a fixed point and flex on some fixed curve $C_0$, and let $\Stab_{(p_0,\ell_0)} \le \Mon(\pi)$ be its stabilizer in the monodromy group. We let
\begin{align*}
    \Flex_{(p_0,\ell_0)} := \left\{ C\in W_d \mid m_{p_0}(C\cdot \ell_0)\ge 3 \right\},
\end{align*}
and consider the incidence variety 
\begin{align*}
    I' := \{(C,p,\ell)\in \Flex_{(p_0,\ell_0)} \times I_0 \mid m_p(C\cdot \ell) \ge 3,\ p\ne p_0,\ \ell\ne \ell_0\} \subseteq \Flex_{(p_0,\ell_0)} \times I_0.
\end{align*}
It suffices to argue that $I'$ is irreducible, since then if $U$ is as in the proof of \Cref{prop:monodromy-flexes-transitive}, we will have that $\pi^{-1}(U) \cap I'$ is connected. Again we use $\eta$, and we get a map
\begin{align*}
    \eta \colon I' \to I_0.
\end{align*}
This surjects onto the Zariski open $\left\{ (p,\ell) \mid p\ne p_0,\ \ell\ne \ell_0 \right\} \subseteq \P^2 \times (\P^2)^\ast$, which is a Zariski open subset of the irreducible variety $I_0$, and hence itself irreducible. The fibers of $\eta$ are linear spaces of constant codimension $3$, hence again by \Cref{prop:irred-criterion}, we conclude that $I'$ is irreducible.
\end{proof}


To argue that the monodromy group is full symmetric, it suffices to exhibit a transposition. Here we use this key lemma:

\begin{lemma}[{\cite[p.~698]{Harris-Galois}}] 
\label{lem:transposition-exists}
Let $\pi \colon Y \to X$ be holomorphic of degree $n$, and suppose there exists some $p\in X$ so that $\pi^{-1}(p) = \left\{ q_1, \ldots, q_{n-1} \right\}$ has $(n-1)$ distinct points, so that $\pi$ is simple at $q_1, \ldots, q_{n-2}$, and $\pi$ has a double point at $q_{n-1}$. Suppose further that $Y$ is locally irreducible at $q_{n-1}$. Then the monodromy group of $\pi$ contains a simple transposition, obtained by taking a small loop around $p$.
\end{lemma}

So we'd like to locate a planar curve with $3d(d-2) - 2$ simple flexes, and exactly one \textit{hyperflex}. Note at the hyperflex the tangent line will meet to order $\ge 4$, and in particular this would violate B\'{e}zout's theorem if $d \le 3$.

\begin{theorem} For $d\ge 4$, the monodromy group of $I_d \xto{\pi} W_d$ is the full symmetric group $\Sigma_{3d(d-2)}$.
\end{theorem}
\begin{proof} We want to apply \autoref{lem:transposition-exists}, so we need to show that a curve with exactly $3d(d-2)-1$ flexes exists, with a simple hyperflex at exactly one of these flexes, and no other interesting behaviors (no singularities or anything). We moreover need to argue that $I_d$ is locally irreducible at this tuple.

To exhibit such a curve, we start with a fixed point $(p_0,\ell_0) \in I_0$, and consider the linear system of degree $d$ curves with a (possibly higher order) hyperflex at $(p_0,\ell_0)$:
\begin{align*}
    \HypFlex_{(p_0,\ell_0)} := \left\{ C\in W_d \colon m_{p_0}(C\cdot \ell_0) \ge 4 \right\} \subseteq W_d.
\end{align*}
We claim that the generic $C \in \HypFlex_{(p_0,\ell_0)}$ is smooth.\footnote{Zhang says this follows from Bertini's theorem.} We claim that the generic degree $d$ curve with a hyperflex at $(p_0,\ell_0)$ has simple flexes at all other points. To see this, we first set up the incidence variety
\begin{align*}
    I'' := \left\{ (C,p,\ell) \in \HypFlex_{(p_0,\ell_0)} \times I_0 \mid m_p(C\cdot \ell)\ge 3,\ p\ne p_0,\ \ell\ne \ell_0  \right\} \subseteq \HypFlex_{(p_0,\ell_0)} \times I_0.
\end{align*}
Note that $I''$ is irreducible (via analogous arguments to the ones we've used to show $I_d$ and $I'$ are irreducible previously). The dimension of $I''$ is the same as the dimension of $\HypFlex_{(p_0,\ell_0)}$.

We claim now that the general $(C,p,\ell) \in I''$ has the property that $p$ is a simple flex of $C$. Indeed the locus in $I''$ of curves with multiple hyperflexes is closed, hence is either equal to $I''$ or of strictly smaller dimension. We claim we cannot have equality -- TODO how does this argument wrap?
\end{proof}

\subsection{Flexes on a plane cubic: geometric properties}

The following observation is classical, and its origins are hard to trace, but it was well-known at the time of Jordan (see \cite[p.~302]{Jordan}). The configuration of lines and flexes along a planar cubic is called the \textit{Hesse arrangement} (see \cite[p.~118]{Dolgachev}).

\begin{proposition} Let $C$ be a smooth planar cubic, and consider its nine inflection points.
\begin{enumerate}
    \item These nine points lie on twelve lines
    \item The twelve lines meet four at a time at any given flex.
\end{enumerate}
This configuration is illustrated in \Cref{fig: A2 F3}.
\end{proposition}
\begin{proof} This can be seen by endowing $C$ with an abelian structure as an elliptic curve, after which the flexes of $C$ can be identified with the $3$-torsion points in the elliptic curve structure.
\end{proof}

Jordan noted (in modern language) that any monodromy of the flexes over the moduli of cubics will preserve the Hesse arrangement, in particular the resulting permutation of the nine flexes will preserve the colinearity properties they satisfy. He proved the following result.

\begin{theorem}[{\cite[Theorem~425]{Jordan}}] Let $f(x)$ be an irreducible polynomial of degree nine, and let $\psi(x,y)$ be a rational function of two variables, which is symmetric in the variables. Suppose that $f$ has the property that, given any two of its roots $a$ and $b$, we can define a third root by the formula $c = \psi(a,b)$, which further satisfies:
\begin{align*}
    b = \psi(a,c) \text{ and } a = \psi(b,c).
\end{align*}
Then the Galois group of $f$ is contained in the affine special linear group $\ASL(2,\F_3)$.
\end{theorem}

The immediate application of this is of course:

\begin{corollary} The monodromy of flexes on smooth plane cubics is contained in $\ASL(2,\F_3)$.
\end{corollary}

Jordan didn't argue this was an equality, although Dickson and Weber did shortly thereafter (modulo a technicality about which field we are working over -- a technicality which is explained by Hilbert irreducibility and the fact that nine is odd. We should explain further). We will prove this is an equality following \cite[\S~II.2]{Harris-Galois}.

\begin{question} Can we discuss Jordan's theorem in the language of resolvent degree?
\end{question}


\begin{note} This group $\ASL(2,\F_3)$ acts on the cubic via projective transformations, and hence forms a subgroup of $\PGL_3(\C)$ often called the \textit{Hesse group} \cite[\S3.1.4]{Dolgachev}.
\end{note}

\begin{note} This group $\ASL(2,\F_3)$ has GAP ID \texttt{[216,153]}.
\end{note}

\begin{question} Does this group (as so many other groups in this field) admit any exceptional isomorphisms which hint at other ways to visualize the monodoromy?
\end{question}


\subsection{Monodromy of flexes on plane cubics}

\paragraph{\textbf{The Jacobian and Abel's Theorem}}

Let $C$ be a smooth curve of genus $g$ over $\C$. We have $H_1(C;\Z) \cong \Z^{2g}$ and by Serre duality, $H^0(C, \omega_C) \cong H^1(C, \O_C) \cong \C^g$. There is a natural inclusion $i: H_1(C;\Z) \to H^0(C, \omega_C)^\vee$ sending the path $(p\to q)$ to the functional $\int_p^q$. The \textit{Jacobian} of $C$ is $$J(C):= \frac{H^0(C, \omega_C)^\vee}{H_1(C; \Z)} \cong \frac{\C^g}{\Z^{2g}}.$$
If we choose a base point $p_0\in C$, then for all $d\geq 1$, the \textit{Abel-Jacobi map} $\mu_d: C^d\to J(C)$ sends a $d$-uple $(q_1, \dots, q_d)$ to the functional $\left[\sum_{i = 1}^d \int_{p_0}^{q_i}\right]$.
\begin{theorem}[Abel's Theorem \textcolor{red}{cite}]
    Two divisors $D_1$ and $D_2$ of degree $d$ on $C$ are linearly equivalent if and only if their images under the Abel-Jacobi map $\mu_d$ are equal. Moreover, the Abel-Jacobi map factors as $$C^d\to \operatorname{Pic}_d(C)\overset{-d\cdot p_0}{\longrightarrow} \operatorname{Pic}_0(C) \overset{\cong}{\to} J(C).$$
\end{theorem}
\begin{corollary} \label{cor: n torsion in J(C)}
    The number of $n$-torsion points in $\operatorname{Pic}_0(C)$ is $n^{2g}$.
\end{corollary}

\begin{proof}
    The group $\operatorname{Pic}_0(C)$ is isomorphic to the Jacobian $J(C)\cong (S^1)^{2g}$. The subgroup of $n$-torsion points is then $(\Z/n)^{2g} \subset (S^1)^{2g}$ and has order $n^{2g}$.
\end{proof}

\begin{proposition}
    For an elliptic curve $(E, o)$, the Abel-Jacobi map $\mu_1: E \to J(E)$ is an isomorphism. 
\end{proposition}

\paragraph{\textbf{Flexes of a Plane Cubic and Monodromy}}

Let $(E,o)$ be an elliptic curve. The complete linear series $|3o|$ embeds $E$ as a smooth plane cubic curve. The divisor $3o$ is then linearly equivalent to the intersection of $E$ with a general line $\ell$. 

If $p\in E$ is a flex point with tangent line $\ell$, then $3p\sim 3o$, which means $[3(p-o)]= 0$ in $\operatorname{Pic}_0(C)$. By Corollary \ref{cor: n torsion in J(C)}, there are $3^2 = 9$ points in $\operatorname{Pic}_0(C)$ that are $3$-torsion, so $E$ has $9$ flexes. 

Recall that $E\cong J(E) \cong \C/ \Lambda$ where the lattice $\Lambda = H_1(E; \Z)$. If $z\in \C$ is such that $z (\mod \Lambda)$ is a flex, then $z-o \in \frac{1}{3}\Lambda$. Therefore, the flexes of $E$ can be obtained as the cokernel $\operatorname{coker}\left(\Lambda \overset{\cdot 3}{\longrightarrow} \Lambda\right) \cong \A^2_{\mathbb{F}_3}$.
Upon choosing an origin for $\A^2_{\mathbb{F}_3}$, the intersection form on $\Lambda = H_1(E; \Z)$ descends to a bilinear form $\alpha: \mathbb{F}_3^2 \times \mathbb{F}_3^2 \to \mathbb{F}_3$. 

\begin{proposition}
    The monodromy group $\Mon(\pi)$ preserves the structure of the affine space $\A^2_{\mathbb{F}_3}$. Upon choosing an origin $O \in \A^2_{\mathbb{F}_3}$, the stabilizer subgroup $\operatorname{Stab}_O$ preserves the $\mathbb{F}_3$-bilinear form $\alpha$. 
\end{proposition}

\begin{proof}
    Todo.
\end{proof}

\begin{figure}
    

\tikzset{every picture/.style={line width=0.75pt}} %set default line width to 0.75pt        

\begin{tikzpicture}[x=0.75pt,y=0.75pt,yscale=-1,xscale=1]
%uncomment if require: \path (0,236); %set diagram left start at 0, and has height of 236

%Shape: Circle [id:dp29149172681843316] 
\draw   (295.5,60) .. controls (295.5,57.51) and (297.51,55.5) .. (300,55.5) .. controls (302.49,55.5) and (304.5,57.51) .. (304.5,60) .. controls (304.5,62.49) and (302.49,64.5) .. (300,64.5) .. controls (297.51,64.5) and (295.5,62.49) .. (295.5,60) -- cycle ;
%Shape: Circle [id:dp9808173276719914] 
\draw   (345.5,60) .. controls (345.5,57.51) and (347.51,55.5) .. (350,55.5) .. controls (352.49,55.5) and (354.5,57.51) .. (354.5,60) .. controls (354.5,62.49) and (352.49,64.5) .. (350,64.5) .. controls (347.51,64.5) and (345.5,62.49) .. (345.5,60) -- cycle ;
%Shape: Circle [id:dp7653584666158506] 
\draw   (395.5,60) .. controls (395.5,57.51) and (397.51,55.5) .. (400,55.5) .. controls (402.49,55.5) and (404.5,57.51) .. (404.5,60) .. controls (404.5,62.49) and (402.49,64.5) .. (400,64.5) .. controls (397.51,64.5) and (395.5,62.49) .. (395.5,60) -- cycle ;
%Shape: Circle [id:dp048804822483997645] 
\draw   (395.5,110) .. controls (395.5,107.51) and (397.51,105.5) .. (400,105.5) .. controls (402.49,105.5) and (404.5,107.51) .. (404.5,110) .. controls (404.5,112.49) and (402.49,114.5) .. (400,114.5) .. controls (397.51,114.5) and (395.5,112.49) .. (395.5,110) -- cycle ;
%Shape: Circle [id:dp4011922592976429] 
\draw   (345.5,110) .. controls (345.5,107.51) and (347.51,105.5) .. (350,105.5) .. controls (352.49,105.5) and (354.5,107.51) .. (354.5,110) .. controls (354.5,112.49) and (352.49,114.5) .. (350,114.5) .. controls (347.51,114.5) and (345.5,112.49) .. (345.5,110) -- cycle ;
%Shape: Circle [id:dp47663344954512565] 
\draw   (295.5,110) .. controls (295.5,107.51) and (297.51,105.5) .. (300,105.5) .. controls (302.49,105.5) and (304.5,107.51) .. (304.5,110) .. controls (304.5,112.49) and (302.49,114.5) .. (300,114.5) .. controls (297.51,114.5) and (295.5,112.49) .. (295.5,110) -- cycle ;
%Shape: Circle [id:dp027829738163854145] 
\draw   (295.5,160) .. controls (295.5,157.51) and (297.51,155.5) .. (300,155.5) .. controls (302.49,155.5) and (304.5,157.51) .. (304.5,160) .. controls (304.5,162.49) and (302.49,164.5) .. (300,164.5) .. controls (297.51,164.5) and (295.5,162.49) .. (295.5,160) -- cycle ;
%Shape: Circle [id:dp06589455695352875] 
\draw   (345.5,160) .. controls (345.5,157.51) and (347.51,155.5) .. (350,155.5) .. controls (352.49,155.5) and (354.5,157.51) .. (354.5,160) .. controls (354.5,162.49) and (352.49,164.5) .. (350,164.5) .. controls (347.51,164.5) and (345.5,162.49) .. (345.5,160) -- cycle ;
%Shape: Circle [id:dp20194510942366017] 
\draw   (395.5,160) .. controls (395.5,157.51) and (397.51,155.5) .. (400,155.5) .. controls (402.49,155.5) and (404.5,157.51) .. (404.5,160) .. controls (404.5,162.49) and (402.49,164.5) .. (400,164.5) .. controls (397.51,164.5) and (395.5,162.49) .. (395.5,160) -- cycle ;
%Straight Lines [id:da4726082385512418] 
\draw  [dash pattern={on 0.84pt off 2.51pt}]  (304.5,60) -- (345.5,60) ;
%Straight Lines [id:da9339583028826415] 
\draw  [dash pattern={on 0.84pt off 2.51pt}]  (354.5,60) -- (395.5,60) ;
%Straight Lines [id:da753918387431791] 
\draw  [dash pattern={on 0.84pt off 2.51pt}]  (304.5,110) -- (345.5,110) ;
%Straight Lines [id:da4693302220823379] 
\draw  [dash pattern={on 0.84pt off 2.51pt}]  (354.5,110) -- (395.5,110) ;
%Straight Lines [id:da9317365682118544] 
\draw  [dash pattern={on 0.84pt off 2.51pt}]  (304.5,160) -- (345.5,160) ;
%Straight Lines [id:da6404048128863478] 
\draw  [dash pattern={on 0.84pt off 2.51pt}]  (354.5,160) -- (395.5,160) ;
%Straight Lines [id:da5757561221477003] 
\draw  [dash pattern={on 0.84pt off 2.51pt}]  (300,64.5) -- (300,105.5) ;
%Straight Lines [id:da5870716236628886] 
\draw  [dash pattern={on 0.84pt off 2.51pt}]  (300,114.5) -- (300,155.5) ;
%Straight Lines [id:da685666987947436] 
\draw  [dash pattern={on 0.84pt off 2.51pt}]  (350,64.5) -- (350,105.5) ;
%Straight Lines [id:da9301291379483692] 
\draw  [dash pattern={on 0.84pt off 2.51pt}]  (350,114.5) -- (350,155.5) ;
%Straight Lines [id:da6692278327982961] 
\draw  [dash pattern={on 0.84pt off 2.51pt}]  (400,64.5) -- (400,105.5) ;
%Straight Lines [id:da10220446606530609] 
\draw  [dash pattern={on 0.84pt off 2.51pt}]  (400,114.5) -- (400,155.5) ;
%Straight Lines [id:da36731177860605024] 
\draw  [dash pattern={on 0.84pt off 2.51pt}]  (303,106.83) -- (347,62.83) ;
%Straight Lines [id:da4930805410953554] 
\draw  [dash pattern={on 0.84pt off 2.51pt}]  (303.5,157) -- (347.5,113) ;
%Straight Lines [id:da6401249829778788] 
\draw  [dash pattern={on 0.84pt off 2.51pt}]  (354,106.5) -- (398,62.5) ;
%Straight Lines [id:da21630549216753225] 
\draw  [dash pattern={on 0.84pt off 2.51pt}]  (353.5,157) -- (397.5,113) ;
%Straight Lines [id:da35063104508567067] 
\draw  [dash pattern={on 0.84pt off 2.51pt}]  (303.5,63) -- (347,106.5) ;
%Straight Lines [id:da8023105746802055] 
\draw  [dash pattern={on 0.84pt off 2.51pt}]  (354.25,62.75) -- (397.75,106.25) ;
%Straight Lines [id:da4855163525491868] 
\draw  [dash pattern={on 0.84pt off 2.51pt}]  (303.75,113.25) -- (347.25,156.75) ;
%Straight Lines [id:da3350558438730069] 
\draw  [dash pattern={on 0.84pt off 2.51pt}]  (353.75,113.25) -- (397.25,156.75) ;
%Straight Lines [id:da914773440303466] 
\draw  [dash pattern={on 0.84pt off 2.51pt}]  (404.5,155) -- (430,129.83) ;
%Straight Lines [id:da4581769528242361] 
\draw  [dash pattern={on 0.84pt off 2.51pt}]  (275,84.83) -- (297,62.5) ;
%Straight Lines [id:da5439048799611215] 
\draw  [dash pattern={on 0.84pt off 2.51pt}]  (303.5,164) -- (327.26,188.43) ;
%Straight Lines [id:da6809518926167577] 
\draw  [dash pattern={on 0.84pt off 2.51pt}]  (375,34.83) -- (397,56.5) ;
%Shape: Arc [id:dp8699401089393226] 
\draw  [draw opacity=0][dash pattern={on 0.84pt off 2.51pt}] (347.09,163.41) .. controls (347.09,163.41) and (347.09,163.41) .. (347.09,163.41) .. controls (326.4,184.1) and (292.86,183.91) .. (272.17,163) .. controls (251.48,142.08) and (251.48,108.36) .. (272.17,87.67) -- (309.63,125.54) -- cycle ; \draw  [dash pattern={on 0.84pt off 2.51pt}] (347.09,163.41) .. controls (347.09,163.41) and (347.09,163.41) .. (347.09,163.41) .. controls (326.4,184.1) and (292.86,183.91) .. (272.17,163) .. controls (251.48,142.08) and (251.48,108.36) .. (272.17,87.67) ;  
%Shape: Arc [id:dp5167926451569047] 
\draw  [draw opacity=0][dash pattern={on 0.84pt off 2.51pt}] (403,113.5) .. controls (403,113.5) and (403,113.5) .. (403,113.5) .. controls (423.69,134.19) and (423.5,167.73) .. (402.59,188.42) .. controls (381.67,209.11) and (347.95,209.11) .. (327.26,188.43) -- (365.13,150.96) -- cycle ; \draw  [dash pattern={on 0.84pt off 2.51pt}] (403,113.5) .. controls (403,113.5) and (403,113.5) .. (403,113.5) .. controls (423.69,134.19) and (423.5,167.73) .. (402.59,188.42) .. controls (381.67,209.11) and (347.95,209.11) .. (327.26,188.43) ;  
%Shape: Arc [id:dp6187993624010264] 
\draw  [draw opacity=0][dash pattern={on 0.84pt off 2.51pt}] (355.07,54.09) .. controls (355.07,54.09) and (355.07,54.09) .. (355.07,54.09) .. controls (375.76,33.4) and (409.31,33.59) .. (430,54.5) .. controls (450.69,75.42) and (450.69,109.14) .. (430,129.83) -- (392.54,91.96) -- cycle ; \draw  [dash pattern={on 0.84pt off 2.51pt}] (355.07,54.09) .. controls (355.07,54.09) and (355.07,54.09) .. (355.07,54.09) .. controls (375.76,33.4) and (409.31,33.59) .. (430,54.5) .. controls (450.69,75.42) and (450.69,109.14) .. (430,129.83) ;  
%Shape: Arc [id:dp16392203774247194] 
\draw  [draw opacity=0][dash pattern={on 0.84pt off 2.51pt}] (297,107.5) .. controls (297,107.5) and (297,107.5) .. (297,107.5) .. controls (276.31,86.81) and (276.5,53.27) .. (297.41,32.58) .. controls (318.33,11.89) and (352.05,11.89) .. (372.74,32.57) -- (334.87,70.04) -- cycle ; \draw  [dash pattern={on 0.84pt off 2.51pt}] (297,107.5) .. controls (297,107.5) and (297,107.5) .. (297,107.5) .. controls (276.31,86.81) and (276.5,53.27) .. (297.41,32.58) .. controls (318.33,11.89) and (352.05,11.89) .. (372.74,32.57) ;  




\end{tikzpicture}
    \caption{Lines on affine plane over $\mathbb{F}_3$.}
    \label{fig: A2 F3}
\end{figure}


\subsection{Solvability of flexes in radicals, in the style of Jordan}

Observe that $\ASL_2(\F_3)$ is a \textit{solvable group} -- this was first noticed by Jordan in \cite[III.III\S1]{Jordan}. An example subnormal series exhibiting solvability is:
\[ \begin{tikzcd}
    C_4\rar[hook] & Q_8\rar[hook]\dar[two heads] & \SL_2(\F_3)\dar[two heads]\rar[hook] & \ASL_2(\F_3)\dar[two heads]\\
     & C_2 & C_2 & \F_3^2.
\end{tikzcd} \]
Here for instance
\begin{align*}
    C_4 &= \left\langle \begin{pmatrix} 0 & 1 \\ 2 & 0 \end{pmatrix}  \right\rangle \le \left\langle \begin{pmatrix} 0 & 1 \\ 2 & 0 \end{pmatrix}, \begin{pmatrix} 2 & 2 \\ 2 & 1 \end{pmatrix}  \right\rangle = Q_8.
\end{align*}
In particular this solvability implies that the equation for a flex point on a cubic admits a formula in radicals in terms of the coefficients for the cubic.

\begin{question} Can we find this formula written out explicitly anywhere?
\end{question}

\subsection{Solving for flexes on a given cubic}

A textbook treatment is described on \cite[p.~333]{MillerBlichfeldtDickson}, in a section appropriately entitled \emph{Group $G$ of the Equation $X$ for the Abscissas}\footnote{This is terminology in some older math books: \textit{abscissa} means $x$-coordinate, and \textit{ordinate} means $y$-coordinate.} \emph{of the Points of Inflexion}. For a general cubic $f(x,y,z)=0$, we can restrict our attention to the affine patch $z=1$. We then set up the equation for $f(x,y,1)=\det Hf = 0$, and via elimination theory we can write $y$ as a rational function in $x$:
\begin{align*}
    y = \phi(x) \in \C(a_0,a_1, \ldots, a_9)(x),
\end{align*}
where the $a_i$'s are the coordinates of a general cubic. Finally we plug in $f(x,\phi(x),1) =0$ and clear denominators to obtain a degree nine polynomial in $x$, the roots of which are the $x$-coordinates of all the nine flexes on the cubic.


\section{Monodromy of bitangents}

A classical problem is to compute the monodromy group of the 28 bitangents to a smooth planar quartic. As mentioned, this was a particular application for Galois theory envisioned by Jordan \cite[III.VI]{Jordan}.

\subsection{History of bitangents to quartics}

As discussed earlier, the expression (\Cref{eqn:bitangents-in-terms-of-degree}) for the number of bitangents to a smooth curve of degree $d$, in terms of $d$, was known to Pl\"{u}cker in the 1830's. Despite this, Pl\"{u}cker's arguments are considered from a modern perspective (and likely also at the time) as sketchy at best. The so-called Pl\"{u}cker formulas were not rigorously proven by Pl\"{u}cker, and although he did sketch an accurate resolution of the pole-polar duality paradox, it did not constitute a proof.

The expression in \Cref{eqn:bitangents-in-terms-of-degree} was proven by Jacobi \cite{Jacobi1850}, an application of which is the computation that when $d=4$, a smooth planar quartic has 28 bitangents. This was extended shortly afterwards by Hesse, a student of Jacobi, who showed that a \textit{general} quartic has 28 bitangents \cite{Hesse1855}. Hesse's work was immensely difficult, and he likened it in a letter to Jacobi to Newton's work discovering the law of gravity \cite[p.~165]{Gray-worlds}.

For a very classical textbook treatment of the theory of bitangents to quartics, section 12 (\S95-\S105) of Weber's 1896 book on algebra is dedicated to this \cite[\S95-\S105]{Weber1896}. A contemporary treatment can be found in \cite[\S6.1]{Dolgachev}.

Jordan, inspired by work of Clebsch, initially asked to what extent equations for bitangents can be solved for in terms of the equations for a quartic, which is precisely the question of what the Galois group is. It's unclear to me who first solved for the Galois group of bitangents. We can write it in a number of different ways:
\begin{align*}
    O_6^-(\Z/2) \cong \Sp_6(\Z/2) \cong W(E_7)/\{\pm 1\}.
\end{align*}
The first two describe the same group (like even beyond group isomorphism they are describing the same quadratic forms in mildly different language). The latter group passes through the theory of Lie groups. Some references for this connection:
\begin{itemize}
    \item See \cite[Theorem~9]{DolgachevOrtland} for the derivation of this group in terms of the theory of Cayley octads
    \item Manivel \cite{Manivel2006} discusses this, and mentions in the intro to the paper that the Galois groups for bitangents and lines to cubic surfaces were well-known at the beginning of the 1900's. He references a connection to Lie groups that passes through the theory of Del Pezzo surfaces of degrees 3 and 2. The point of Manivel's paper is cutting out this passage through Del Pezzos and drawing a direct connection to the Lie groups. I'd like to understand both of these stories.
    \item In \cite[p.~367]{MillerBlichfeldtDickson}, the Galois group of 28 bitangents is spelled out explicitly. The $7$ in $E_7$ comes in here through the theory of Aronhold sets --- that is, given an Aronhold set, we can solve for the remaining bitangents. What is this connection, and how does it relate to the theory of Cayley octads?
\end{itemize}

\subsection{\texorpdfstring{$\theta$}{theta}-characteristics for quartics}

Recall that a $\theta$\textit{-characteristic} on a variety $X$ is a square root of the canonical bundle $\omega_X$. From this perspective it can equivalently be thought of as a divisor $D$ for which $2D \sim K_X$.

\begin{definition} We say a $\theta$-characteristic, represented by a line bundle $\mathcal{L}$ on $X$ is...
\begin{itemize}
    \item \textit{even} if $\dim H^0(X,\mathcal{L})$ is even
    \item \textit{odd} if $\dim H^0(X,\mathcal{L})$ is odd
\end{itemize}
it's entirely possible that a $\theta$-characteristic has no nonzero sections. So we also say a $\theta$-characteristic represented by $\mathcal{L}$ is...
\begin{itemize}
    \item \textit{effective} if $\dim H^0(X,\mathcal{L})>0$
    \item \textit{vanishing} or \textit{non-effective} if $\dim H^0(X,\mathcal{L}) = 0$.
\end{itemize}
\end{definition}

\begin{remark}[reference needed] If $C$ is a complex smooth projective curve of genus $g$, then there is a bijection between the effective $\theta$-characteristics on $C$ and $\Jac(C)[2]$, the $2$-torsion points in the Jacobian of $C$.
\end{remark}

\begin{example} An effective $\theta$-characteristic on a smooth canonical curve of genus $3$ is a bitangent.
\end{example}
\begin{proof} The canonical class $K$ is the hyperplane class for the canonical embedding $C \subseteq \P^2$, hence as a divisor it is in general the sum of four points. If $D$ is a $\theta$-characteristic, it has the property that $2D \sim K$, hence if $D$ is effective, it is a sum of two points $p$ and $q$ on $K$. We'd love to say that $2D$ is actually given by a line, since then it would be tangent to $C$ at both $p$ and $q$. This turns out to be true -- it is a feature of the linear series $|K_C|$ implied by $C$ being projectively normal.\footnote{Can we explain this more? We had a discussion about definitions during Michael's talk I'd love to clarify in the notes.}
\end{proof}

\begin{proposition} On a smooth canonical curve of genus $g=3$, a $\theta$-characteristic is odd if and only if it is effective.
\end{proposition}
\begin{proof} How do we prove this?
\end{proof}


\begin{example} If $C$ is a smooth quartic curve, it has $28$ odd (effective) $\theta$-characteristics, corresponding to the bitangents.
\end{example}

What do the even $\theta$-characteristics correspond to? It turns out they reveal quite a bit about the geometry of the quartic. We have the following maps and constructions, many of which are bijections:
\[ \begin{tikzcd}
    {\{\substack{\text{quartics and even } \\\theta\text{-characteristics }(C,\theta) }\}}\dar[<->,"\sim" right,"(b)" left]\rar[<->,"(a)"] & {\{\substack{\text{space sextics} \\ \text{in } \P^3}\}} & \\
    {\{\substack{\text{nets of quadrics}\\\text{in }\P^3}\}} \rar["(d)" below]\ar[ur,"(c)" below right]\dar["(f)" left] & {\{\substack{\text{symmetric determinantal}\\\text{representations of }C}\}}\uar["(e)" right] \\
    {\{\substack{\text{Cayley octads} \\ \text{in }\P^3}\}}
\end{tikzcd} \]

\begin{enumerate}[(a)]
\item \textit{Sextic space curve embeddings from even $\theta$-characteristics}: Given a tuple $(C,\theta)$, we can consider the very ample linear series $|3\theta| = |K_C+\theta|$. This has $h^0(C,3\theta) = 4$, hence defines an embedding $C \hookto \P^3$ which we can verify is of degree six.

To go back the other way, we consider the difference between the hyperplane class on $C$ embedded in $\P^3$ and its hyperplane class under its canonical embedding in the plane. The difference between these is an even $\theta$-characteristic.\footnote{\textbf{Question}: Is every space sextic a genus $3$ curve embedded via an even $\theta$-characteristic?}

\item \textit{Even $\theta$-characteristics and nets of quadrics}: This bijection is described in the proof of \cite[6.3]{GrossHarris-theta}.


\item \textit{Nets of quadrics to space sextics}: Given a net of quadrics, we can consider its subset of singular elements. Singular quadric surfaces are cones, and their vertices are points in $\P^3$. Varying over all singular quadrics in a net, the vertices sweep out a space curve of degree six. This process is called taking the \textit{discriminant} of the net of sextics.

\item \textit{Nets of quadrics and symmetric determinantal representations}: Any quadric in $\P^3$ can be represented by a symmetric $4 \times 4$ matrix, so given a quadric net we can pick any three quadrics to form a basis, and consider the quartic form produced by the following determinant:
\begin{align*}
    f = \det(xA + yB + zC).
\end{align*}
This is a symmetric determinantal representation of a quartic. To go the other way, we produce the quadrics $Q_A(y) = y^T A y$ and $Q_B$ and $Q_C$ arising from a symmetric determinantal representation of $f$ and look at the net they span.


\item \textit{Space sextics from symmetric determinantal representations}: Given a determinantal representation $(A,B,C)$ for the quartic $C$, we obtain an embedding $C \hookto \P^3$ via
\begin{align*}
    \phi_\theta \colon C &\to \P^3 \\
    [x:y:z] &\mapsto \ker(xA+yB+zC).
\end{align*}
%
\item \textit{Nets of quadrics to Cayley octads}: Given a net of quadric surfaces, its base locus is eight points in $\P^3$, which we call a \textit{Cayley octad}.
\end{enumerate}


\subsection{Cayley octads, formally}

\begin{definition}[{\cite[p.~20]{Cayley-octad}}]
Given eight points in $\P^3$ which are the base locus of a net of quadrics, we say they are a \textit{Cayley octad}.\footnote{These arise in Cayley's work via the study of \textit{octadic surfaces}, being quartic surfaces with eight nodes.}
\end{definition}


\begin{proposition}[{\cite[6.3.3]{Dolgachev}}]
\label{prop:colinearity-coplanarity-cayley-octads}
Any Cayley octad arising from a quartic $(C,\theta)$ has the property that no three of its eight points are colinear, and no four are coplanar.
\end{proposition}

Suppose we are handed eight points in $\P^3$ satisfying \Cref{prop:colinearity-coplanarity-cayley-octads}. We can ask whether an associated quartic and even $\theta$-characteristic can be uniquely recovered from this data, and the answer is yes. Moreover, we don't need all eight points to rebuild the quartic -- we only need \textit{seven of the eight points}.

\subsection{Duality via an even \texorpdfstring{$\theta$}{theta}-characteristic}

Recall we can send a curve to its \textit{dual} by sending a point on a curve to its tangent line. With an even $\theta$-characteristic in hand, we can reframe this in a slightly different way. Recall we have this embedding
\begin{align*}
    \phi_\theta \colon C \to \P^3
\end{align*}
which depended upon a choice of determinantal representation arising from $\theta$. Similarly, we have an associated net of quadrics $N_\theta$, and we can pick a basis for them, say
\begin{align*}
    \text{span}\{Q_1,Q_2,Q_3\} = N_\theta.
\end{align*}
These give rise to a rational map
\begin{align*}
    \psi_\theta \colon \P^3 &\dashto \P^2 \\
    y &\mapsto [Q_1(y) : Q_2(y) : Q_3(y)].
\end{align*}
%
This is defined everywhere except the base locus of the net (at the Cayley octad).

\begin{proposition} The image $\im(\gamma)$ of the composite of $\phi_\theta$ and $\psi_\theta$ is the dual curve $C^\ast$:
\[ \begin{tikzcd}
    C\ar[dr,"\gamma"]\dar["\phi" left] & \\
    \P^3\rar[dashed,"\psi" below] & \P^2.
\end{tikzcd} \]
\end{proposition}

This perspective makes the connection to bitangents slightly more clear. If we look at the eight points $\left\{ p_1, \ldots, p_8 \right\}$ forming the indeterminacy loci of $\psi$, the image of $\psi(\overline{p_i p_j})$ is a single point in the dual projective plane, yielding a bitangent to the original quartic. There are $\binom{8}{2}$ such lines, yielding $28$ bitangents.

See for instance \cite[\S2]{Sameera-theta} for a more careful treatment of this.







\subsection{Computing the monodromy of bitangents to plane curves}




... to show the monodromy is transitive, we have to show (by covering space theory) that the covering space is connected. To do this, we consider the projection
\begin{align*}
    \eta \colon J_d \to J_0 \cong \Conf_2(\P^2).
\end{align*}
The fiber of $\eta$ over a point in $J_0$ is a linear subspace in $J_d$. So $\eta$ is a fiber bundle, hence $J_d$ is irreducible, hence connected. Then the monodromy is transitive.

\begin{proposition} The monodromy acts 2-transitively on the fiber.
\end{proposition}
\begin{proof} It suffices to argue that the stabilizer of a bitangent, as a subgroup of monodromy, acts on the remaining points in the fiber transitively.... TODO
\end{proof}


To exhibit a simple transposition, we want to find a curve with one fewer than the generic number of bitangents. It will be a curve with a simple flex bitangent. This will exist for $d \ge 5$ (because we can't have a flex bitangent for a degree four curve by B\'{e}zout).


\subsection{Monodromy of bitangents to quartics}

Given a bitangent $\overline{p_1 p_2}$ to a generic quartic $C$, note that $2p_1 + 2p_2 \sim K_C$ is the canonical class, i.e. a hyperplane section of the quartic. Conversely, since $C$ is ``nice'' any divisor equivalent to $K_C$ which has the form $2p_1 + 2p_2$ is cut out by a line. Here nice means projectively normal, i.e. canonically embedded by a complete linear series, and for every $d$, we have a surjection
\begin{align*}
    H^0(\P^2, \O_{\P^2}(d)) \tto H^0(C,\O_C(d)).
\end{align*}
Projectively normal is normal + this condition.

Hence we get a bijection between bitangents to $C$ and divisors $p_1 + p_2$ so that $2(p_1 + p_2)\sim K_C$.

If $k$ is any divisor with $2k\sim K_C$, then $2(p_1 + p_2 - k) = 0$. Hence $(p_1 + p_2) - k$ is a deg 0 divisor in $C$ and $k - (p_1 + p_2)$ has order 2.

So bitangents correspond to 2-torsion points in $\Pic^0$.

\textbf{Abel-Jacobi theorem} We have that
\begin{align*}
    \Pic^0(C) \xto{\sim} J(C),
\end{align*}
where $J(C)$ is the Jacobian
\begin{align*}
    J(C) = H^0(C,\Omega^1_C)^\ast / H_1(C;\Z).
\end{align*}
%
Explicitly, if we fix a basepoint $p_0 \in C$, we send $p - p_0$ to $\omega \mapsto \int_{p_0}^p \omega$. A priori the target depends on the path we pick between $p_0$ and $p$, however we've modded out by $H_1(C;\Z)$ so it doesn't matter.

Bitangents correspond to some 2-torsion points in $J(C) = \C^3 / \Z^6$.\footnote{Here $g = 3$ so $H^0(C,\Omega^1_C) \cong \C^3$.} Here we're looking at a torus, so we have a nice structure to work with to find 2-torsion stuff.

[pic -- in $\C/\Z$ there are four 2-torsion points]

We know the points of order 2 in $J(C)$ form a 6D vector space $\mathbb{F}_2^{6}$. This has 64 elements, so not all of them are bitangents.

How does the monodromy group act on $J(C)$? We claim it preserves the affine structure (this follows from the following: if we take four bitangents, with the property that the four lines sum to zero in the jacobian, then the eight points of tangency lie on a common conic. this geometric intersection property is preserved by monodromy ), so we get that
\begin{align*}
    \Mon(\pi) \le \text{AGL}_6(\mathbb{F}_2).
\end{align*}
%
It turns out there are further restrictions.

We have a skew-symmetric non-degenerate pairing
\begin{align*}
    \til{Q} \colon H_1(C;\Z) \times H_1(C,\Z) \to \Z,
\end{align*}
given by intersection with signs. This induces a form on half of the lattice
\begin{align*}
    4 \til{Q} \colon \frac{1}{2}\Lambda \times \frac{1}{2} \Lambda &\to \Z \\
    (v,w) &\mapsto \til{Q}(2v,2w).
\end{align*}
This is ``manually imposing'' the intersection form on the more dense lattice.

This in turn induces a form on the quotient (where $V = \frac{1}{2}\Lambda / \Lambda$ is our 6D vector space over $\mathbb{F}_2$ as before):
\begin{align*}
    Q \colon V \times V \to \Z/2.
\end{align*}
It is valued in $2\Z$, since if we input any two things from $\Lambda$ we get something even.

This is a \textit{strictly} skew-symmetric bilinear form.

This form is preserved by $\Mon(\pi)$, hence
\begin{align*}
    \Mon(\pi) \le \text{AO}_6(\mathbb{F}_2).
\end{align*}
%

Associated to any symmetric bilinear form, we will get two quadratic refinements $q^+$ and $q_-$. These are helpful in determining the \textit{effective} semicanonical divisors. That is, we can find quadratic forms satisfying $q(v) = 0$ if and only if $v$ represents a bitangent. The monodromy group will preserve this quadratic form, so we ultimately get the monodromy group is a subgroup of the \textit{Steiner group} $H$, which is $O_6(\Z/2)$.

Can show that $\Stab_{\Mon(\pi)}(k_0) = O_6^{-}(\Z/2)$. Sidhanth will explain!



\subsection{Why \texorpdfstring{$W(E_7)$}{W(E7)}?}

Given a smooth planar quartic, we can take a cover of $\P^2$ branched along the quartic, and we obtain a degree two del Pezzo. This is of the form $\Bl_7(\P^2)$. A line on the del Pezzo is any curve intersecting the canonical divisor with order 1, and we have 56 such lines. The image of any of these in $\P^2$ is precisely one of our bitangents, and they map 2-to-1. From this perspective, $W(E_7)/\{\pm 1\}$ emerges as the Galois group analogously to how $W(E_6)$ emerges as the Galois group from $\Bl_6 \P^2$.

In fact, given seven points on the plane satisfying the closed condition that no three are colinear and no six lie on a conic, we obtain a nonsingular quartic together with a choice of even $\theta$-characteristic. In fact we have a birational isomorphism
\begin{align*}
    \UConf_7(\P^2)/\PGL_3 \overset{\sim}{\dashto} \mathcal{M}_3^\text{ev},
\end{align*}
where $\UConf$ is the moduli of unordered configurations, and $\mathcal{M}_3^\text{ev}$ is the moduli of genus three curves equipped with an even $\theta$-characteristic \cite[6.3.12]{Dolgachev}

\section{Lines on hypersurfaces}

\begin{question} How many lines lie on a general hypersurface $X_d \subseteq \P^n$ of degree $d$.
\end{question}

If $n=3$, $d=3$, then a smooth cubic surface has exactly 27 lines.

For $n=4$, $d=5$, it's not true that a \textit{smooth} quintic threefold has the same number of lines, but a general quintic threefold has 2875 lines.

The \textit{Fano variety} of lines parametrizes lines inside of a given projective variety $X \subseteq \P^n$.

\begin{question} Before constructing this, for which $n$ and $d$ do we expect the existence of lines on a degree $d$ hypersurface $X_d \subseteq \P^n$?
\end{question}

Consider the dimension of the space of all hypersurfaces
\begin{align*}
    N = \dim \left| \O_{\P^n}(d) \right| = \binom{n+d}{n}-1.
\end{align*}
We get an incidence correspondence\footnote{Here $\mathbb{G}(1,n)$ is the lines in $\P^n$, it is equivalently the affine Grassmannian $G(2,n+1)$.}
\begin{align*}
    \Phi_{n,d}:= \left\{ (X,L) \in \P^N \times \mathbb{G}(1,n) : L \subseteq X \right\}.
\end{align*}
We get two projections
\[ \begin{tikzcd}
     & \Phi_{n,d}\ar[dl,"\pi_1" above left]\ar[dr,"\pi_2" above right] & \\
    \mathbb{P}^N &  & \mathbb{G}.
\end{tikzcd} \]
%
\begin{proposition} The incidence variety $\Phi_{n,d}$ is smooth and irreducible of dimension $\dim(\Phi_{n,d}) = N + 2(n-1) - (d+1)$.
\end{proposition}
\begin{proof} Observe that $\pi_2$ is a projective bundle, since over any line $[L] \in \mathbb{G}$, its fiber is all those hypersurfaces containing $L$. This is the same as
\begin{align*}
    \pi_2^{-1}([L]) = \mathbb{P} H^0(\P^n, \mathcal{I}_L(d))^\vee.
\end{align*}
This comes from the ideal sheaf sequence for a line $L \subseteq \P^n$, tensoring with $\O(d)$, then taking the LES on cohomology:
\begin{align*}
    0 \to H^0(\mathcal{I}_L(d)) \to H^0(\O_{\P^n}(d)) \tto H^0(\O_L(d)) \to 0
\end{align*}
We can show that latter map is surjective (can extend any homogeneous polynomial in two variables to one in $n+1$ variables). Since $H^0(\O_L(d)) = H^0(\O_{\P^1}(d))$, we have that $H^0(\mathcal{I}_L(d))$ has constant dimension.\footnote{Jake notes: if we fix a line $L$, then $F_{|L}=0$ is a linear condition on $F$.} Therefore $\pi_2$ is a projective bundle. We get that
\begin{align*}
    \dim H^0(\mathcal{I}_L(d)) &= \dim H^0(\O_{\P^n}(d)) - H^0(\O_L(d)) \\
    &= N - (d+1).
\end{align*}
Then we add $2(n-1)$ back in for the dimension of the Grassmannian we are working over.
\end{proof}

The expected dimension of lines on a general $X_d \subseteq \P^n$ is then
\begin{align*}
    \dim \Phi - N &= 2(n-1) - (d+1).
\end{align*}
%
We expect to have finitely many lines when this is zero, i.e. when $d = 2n-3$.

Since we computed the dimension of $\Phi$, the equality above is an \textit{upper bound}.

\begin{theorem} If the expected dimension is $\ge0$ then every hypersurface contains lines (the fiber of $\pi_1$ is nonempty for every hypersurface $X\in \P^N$) and the dimension of the space of lines on $X$, denoted $F_1(X)$, is equal to the expected dimension, for a general $X$.
\end{theorem}

So if we're working with general hypersurfaces, then we have the expected dimension count.

\begin{conjecture}[Debarre-de Jong] If $X_d \subseteq \P^n$ is \textit{any} smooth hypersurface of degree $d\le n$, then $F_1(X)$ has dimension equal to the expected dimension $2n-3-d$.
\end{conjecture}

This is false in other degree ranges.

\begin{example} If $n=3$ and $d=4$, then $X_4 \subseteq \P^3$ is a smooth quartic K3, and a general one doesn't have any lines. By Noether--Lefschetz the very general quartic has Picard rank one. However we can write down smooth quartics which contain lines (e.g. the Fermat quartic), which would violate the conjecture above.
\end{example}

The main theorem we will want to prove is

\begin{theorem} The Galois group of lines on a general hypersurface $X \subseteq \P^n$ of degree $2n-3$, for $n\ge 4$, is the full symmetric group.
\end{theorem}

\subsection{Construction}

Let $L \subseteq X = \left\{ g=0 \right\} \subseteq \P^n$. Then, thinking of $g \in H^0(\O_{\P^d}(d))$, there is a restriction map to the line $L$:
\begin{align*}
    \res \colon H^0(\P^n(d)) \to H^0(\O_L(d)).
\end{align*}
To say $X$ contains the line is to say $g \mapsto 0$.

\textbf{Key idea}: If we look at the family of restrictions of sections to $L$, we can vary this as $L$ ranges over points in the Grassmannian, and we can realize this as a vector bundle over $\mathbb{G}(1,n)$.

We get that the images of $g$ under the restriction map become values of a section of this bundle.

More explicitly, let $V$ be a vector space of dimension $n+1$, where we are thinking of $\P^n = \mathbb{P}(V)$. There is a bijection between lines $L \subseteq \P^n$ and dimension two subspaces $\Lambda \subseteq V$. Tensoring $V$ with $\O_\mathbb{G}$, we get a trivial bundle
\begin{align*}
    V \otimes\O_\mathbb{G} \to \mathbb{G}.
\end{align*}
Inside of this we have a tautological rank two subbundle $\mathcal{S} \subseteq V \otimes \O_\mathbb{G}$, whose fiber $\mathbb{S}_{[L]}$ is $\Lambda$, where $L = \mathbb{P}(\Lambda)$ for $\Lambda \subseteq V$ a rank two subspace.

The fiber of $\mathcal{S}^\vee$ over $[L]$ is linear forms on $L$, which is $H^0(\O_L(1))$. What does it mean for a polynomial to restrict to zero? Dualizing the inclusion $\mathcal{S} \subseteq V \otimes \O_\mathbb{G}$, we get
\begin{align*}
    V^\vee \otimes \O_{\mathbb{G}} \to \mathcal{S}^\vee.
\end{align*}
Note that $V^\vee \otimes \O_\mathbb{G}$ is trivial, so it's recording \textit{constant sections}. Restricting this map to some $L$, we get that a linear form $\phi$ on $V$ is mapped to the restricted linear form $\phi_{|L}$.

We can symmetrize this to look at degree $d$ forms, and we get
\begin{align*}
    H^0(\O_{\P^n}(d)) = \Sym^d V^\vee \to \Sym^d(\mathcal{S}^\vee),
\end{align*}
sending $g \mapsto g_{|L}$. The constant section $g$ gets send to some $\sigma_g \in H^0(\Sym^d V^\vee)$, and $F_1(X)$ is the zero locus of this section $\sigma_g$.

\begin{proposition} The total Chern class of $\mathcal{S}^\vee$ is
\begin{align*}
    c(\mathcal{S}^\vee) = 1 + \sigma_1 + \sigma_{1,1}.
\end{align*}
\end{proposition}

We have $\sigma_g \in H^0(\Sym^d \mathcal{S}^\vee)$. Here $\mathcal{S}$ has rank two, so $\Sym^d(\mathcal{S}^\vee)$ has rank $d+1$. Its zero locus will then be
\begin{align*}
    \deg c_{d+1}(\Sym^d \mathcal{S}^\vee) = \#\{\text{lines on a general hypersurface } X_{2n-3} \subseteq \P^n\}
\end{align*}
%
We use the splitting principle and get
\begin{align*}
    d d! \sum_k \frac{(2k)!}{k! (k+1)!}\sum_{\substack{I \subseteq \left\{ 1, \ldots, n-2 \right\} \\ \#I = n-2-k}} \prod_{i\in I} \frac{(d-2i)^2}{i(d-i)}.
\end{align*}
This is \url{https://oeis.org/A027363}

\begin{remark} There is a short discussion in the paper of when $F_1(X)$ is singular. It turns out $F_1(X)$ is singular at $u \in F_1(X) \subseteq \mathbb{G}$ if and only if the corresponding line $\lambda_u \subseteq X$ has normal bundle $N$ with the property that $h^0(N) > 2n-3+d$. If we remember, this is the \textit{expected dimension}.

By Riemann--Roch, we write
\begin{align*}
    h^0(N) = 2n-3+d + h^1(N),
\end{align*}
so this is the same as saying $h^1(N)>0$.
\end{remark}

\begin{remark}[On the Galois group] \,
\begin{enumerate}
    \item We show it's transitive (follows from $\Phi$ being irreducible)
    \item We show it's 2-transitive (fix a line $L$ and show that $\Stab_L$ is transitive. Here is where we need $n\ge 4$)
    \item Existence of a transposition: we show that there exists an $F_1(X)$ with a point of multiplicity two (also need $n\ge4$)
\end{enumerate}
\end{remark}

\section{Lines on cubic surfaces}

\subsection{The topology of cubic surfaces}

Recall that a cubic surface $S \subseteq \P^3$ is the blowup of $\P^2$ at six general points, which we will denote by $S = \Bl_{p_1, \ldots, p_6}\P^2$. We denote by $E_i$ the exceptional divisor above $p_i$ for $i=1, \ldots, 6$. Let $H$ be the hyperplane class on $\P^2$.

As smooth manifolds, $S$ is diffeomorphic to $\P^2\# 6\overline{\P^2}$. Hence by Mayer--Vietoris, we get that the Neron--Severi group\footnote{The image of $c_1$.} of $S$ is 
\begin{align*}
    \NS(S) = H^2(S,\Z) = \Z\{H,E_1, \ldots, E_6\}.
\end{align*}
This is a rank 7 $\Z$-module, equipped with an intersection form given by the cup product. Adjunction tells us that the canonical class $K_S$\footnote{The canonical class pulls back under blowups --- recall that $K_{\P^2} = -3H$.} is given by
\begin{align*}
    K_S = -3H + E_1 + \ldots + E_6.
\end{align*}
Cubic surfaces are \textit{anticanonically} embedded in $\P^3$, given by taking global sections of the anticanonical bundle. This tells us that the intersection of $D \subseteq S$ and $-K_S$ records how (un)twisted $D$ is in $S$.

Finally, we note that we get a splitting of cohomology
\begin{align*}
    H^2(S,\Z) = \Z\{K_S\} \oplus \Z\{K_S\}^\perp.
\end{align*}
The orthogonal piece $\Z\{K_S\}^\perp$ is called the integral \textit{primitive cohomology} of $S$.\footnote{This borrows language from Hodge theory. Since cubic surfaces are anticanonically embedded, we have a canonical choice of K\"{a}hler class on $S$. The word ``primitive'' comes from the theory of highest weights.} We will denote by $V$ or $V_\Z$ the primitive cohomology of $S$.

\subsection{Counting lines on a cubic surface}

\begin{proposition} The exceptional curves $E \subseteq S$ are exactly the lines in $S$.
\end{proposition}
\begin{proof} Exceptional implies that the genus of these curves are zero, and the self intersection is $E\cdot E = -1$. By the degree-genus formula (or adjunction or whatever), we have
\begin{align*}
    g(E) = 1 + \frac{K_S\cdot E + E\cdot E}{2}.
\end{align*}
Solving, we get $K_S\cdot E = -1$. This is true if and only if $(-K_S)\cdot E =1$, thus $E$ is linearly embedded in $\P^3$.
\end{proof}

So counting lines on $S$ is the same as counting exceptional curves in homology. Let $E$ be an arbitrary element of the form
\begin{align*}
    E = \alpha H - \sum_{i=1}^6 c_i E_i.
\end{align*}
Since $c_i = E\cdot E_i$, and if we are assuming both $E$ and $E_i$ are lines, then either $c_i = 0$ or $1$ by B\'{e}zout's theorem. Moreover, we have that
\begin{align*}
    1 &= E\cdot (-K_S) = \left( \alpha H - \sum c_i E_i \right)\cdot (3H - E_1 - \ldots - E_6) \\
    &= 3\alpha - \sum c_i.
\end{align*}
We have two possible cases:
\begin{enumerate}
    \item If $\alpha = 1$ then exactly two $c_i$'s must be equal to 1. These lines are $L_{ij}$, which are the strict transform of the line $\overline{p_i p_j} \subseteq \P^2$.
    \item If $\alpha = 2$ then \textit{five} of the $c_i$'s must be equal to $1$. This line is denoted $C_i$, which is the strict transform of the conic passing through $\left\{ p_1, \ldots, p_6 \right\}\minus \left\{ p_i \right\}$.
\end{enumerate}
Altogether we have
\begin{align*}
    \underset{\text{exceptional lines}}{6} + \underset{\text{the }L_{ij}\text{'s}}{\binom{6}{2}} + 6 = 27.
\end{align*}
%
\begin{theorem} There are $27$ lines on $S$.
\end{theorem}

\subsection{The intersection form and its quadratic refinement}

Let $Q$ be the intersection form on mod $2$ cohomology, and $V$ the mod 2 primitive cohomology -- i.e. all classes which are $Q$-orthogonal to $K_S$ modulo $2$. The restricted form
\begin{align*}
    Q \colon V \times V \to \Z/2
\end{align*}
is skew-symmetric. That is, for all $D \in V_\Z$, we have that
\begin{align*}
    1 + \frac{K_S\cdot D + D\cdot D}{ 2 } = 1 + \frac{D\cdot D}{2} \in \Z.
\end{align*}
Hence $D \cdot D = Q(D,D) \equiv 0 \pmod{2}$.

Thus there is a \textit{quadratic refinement} $q$ of $Q$ given by
\begin{align*}
    q \colon V &\to \Z/2 \\
    \bar{D} &\mapsto \frac{D\cdot D}{2} \pmod{2}.
\end{align*}
%
We claim this \textit{remembers} the 27 lines. Explicitly, given an exceptional curve $E \subseteq S$, we have that
\begin{align*}
    (E + K_S)(-K_S) = 1 - 3 \equiv 0 \pmod{2}.
\end{align*}
This implies $E + K_S\in V$ by definition. Let's evaluate on $q$:
\begin{align*}
    q(E + K_S) &= \frac{(E + K_S)\cdot (E + K_S)}{2} \\
    &= \frac{1 - 3 + 2}{2} = 0.
\end{align*}
Thus, we get a well-defined map
\begin{align*}
\{\text{exceptional curve}\} &\to \{\text{roots of }q\} \\
    E &\mapsto q(E+K_S).
\end{align*}
Moreover this is injective, because all exceptional curves are distinct, and $-K_S$ is not exceptional. Observe that
\begin{align*}
    E = \alpha H - \sum c_i E_i \pmod{2}
\end{align*}
intersects to the form
\begin{align*}
    q(E) = \alpha^2 - \sum c_i^2.
\end{align*}
We can check that for $\alpha,c_i \in \{0,1\}$, the quantity $q(E)$ is a multiple of 4 exactly 28 times. One of these is trivial (when everything is zero).

Thus $q$ has $27+1$ zeros. Since $q$ has 27 distinct roots and 27 is odd, we call $q$ an \textit{odd} quadratic refinement.

\subsection{Automorphisms and monodromy}

\begin{proposition} Let $\Gamma$ be the set of lines in $S$. The incidence-preserving permutation group of $\Gamma$ is isomorphic to $\Aut(V,Q,q)$. This is abstractly isomorphic to the \textit{odd} orthogonal group $O^-(6,\Z/2)$. The minus sign is recording that $q$ is an \textit{odd} quadratic refinement of $Q$.
\end{proposition}
\begin{proof} Clearly $\Gamma$ spans $H^2(S,\Z)$, so any permutation of $V$ which preserves $Q$ extends linearly to all of $H^2(S,\Z)$. The anticanonical class can be written as
\begin{align*}
    -K_S = (H - E_1 - E_2) + (H - E_3 - E_4) + (H - E_5 - E_6),
\end{align*}
which is the sum of three mutually intersecting lines. Hence $-K_S$ is also fixed by the incidence-preserving permutation group of $\Gamma$. Since $-K_S$ is fixed, so is its orthogonal complement, i.e. primitive cohomology. Here $\sigma$ induces an action on $V \subseteq H^2(S,\Z/2)$ which preserves $Q$ and $q$. Moreover any induced permutation of the roots of $q$ can be used to rebuild the permutation $\sigma$ of $\Gamma$. Thus $\Aut(\Gamma) \cong O^-(6,\Z/2)$.
\end{proof}

Observe that six pairwise disjoint lines, together with $-K_S$, span all of $H^2(S,\Z)$. Hence an element of $O^-(6,\Z/2)$ is determined by what it does to six disjoint lines. Let
\begin{align*}
    \Omega = \left\{ \text{sets of }6\text{ disjoint lines} \subseteq S \right\}.
\end{align*}
We can directly verify that this group has order 72. This implies that
\begin{align*}
    |O^-(6,\Z/2)| = 6! \cdot |\Omega| = 51840.
\end{align*}
%
Let's call $\sigma\in O^-(6,\Z/2)$ \textit{elementary} if $\sigma(\gamma) = \gamma$ for \textit{some} $\gamma\in \Omega$. That is, for some choice of six disjoint lines, $\gamma$ acts by permuting them amongst themselves.

\begin{proposition} The group $O^-(6,\Z/2)$ is \textit{generated} by elementary elements.
\end{proposition}
\begin{proof} Let $G' \le O^-(6,\Z/2)$ be the subgroup generated by elementary elements, and let $\Omega' \subseteq \Omega$ be the orbit of $\left\{ E_1, \ldots, E_6 \right\}$ under $G'$. We can see that
\begin{align*}
    |G'| = 6! |\Omega'| = 720|\Omega'|
\end{align*}
and since $|G'|$ divides $51840$, we have that $|\Omega'|$ divides $72$. Notice that all $72$ families of disjoint $6$ lines are acted upon transitively by the permutations of $\left\{ E_1, \ldots, E_6 \right\}$. The point is that $\Omega'$ is a union of these subfamilies, and the only union having the right order is of size $72$, so $\Omega' = \Omega$.
\end{proof}

The monodromy group of 27 lines on cubic surfaces is clearly a subgroup $M \le O^-(6,\Z/2)$. This is exactly saying that monodromy proves the incidence structure of lines on a cubic surface. What turns out to be true is that this is equality.

\begin{theorem} $M$ contains all elementary automorphisms, hence $M = O^-(6,\Z/2)$.
\end{theorem}
\begin{proof} Given some $\sigma \in O^-(6,\Z/2)$ we want to produce a path in the moduli space inducing that permutation. Suppose $\sigma$ permutes $E_1, \ldots, E_6$ on $S$. Blowing down to $\P^2$, $\sigma$ defines a map permuting $p_1, \ldots, p_6$. Draw some paths from $p_i$ to $p_{\sigma(i)}$. Since being in general position is Zariski open in $(\P^2)^{\times6}$, we can take the paths $p_i(t)$ connecting $p_i$ to $p_{\sigma(i)}$ to also be in general position for all times $0\le t \le 1$. We get cubic surfaces
\begin{align*}
    S_t = \Bl_{p_1(t), \ldots, p_6(t)},
\end{align*}
still anticanonically embedded in $\P^3$. Moreover global sections 
\begin{align*}
    s_0(t), \ldots, s_3(t) \in H^0(S_t, -K_{S_t})
\end{align*}
are \textit{also} varying continuously. Thus the monodromy action given by this path in $(\P^2)^{\times6}-\Delta$ induces $\sigma$ again.
\end{proof}


\section{Galois groups to bitangents, part II}

Goal today:
\begin{enumerate}
    \item recap the moduli problem
    \item prove a lemma
    \item compute the stabilizer of a bitangent
    \item show that $M$ isn't solvable.
\end{enumerate}

\subsection{What we know}

By the Pl\"{u}ocker formula, we have 28 bitangents, so we have an irreducible cover
\begin{align*}
    I_4 \to W_4 = \mathbb{P} \left( H^0(\P^2, \O(4)) \right).
\end{align*}
Recall $J(C)\cong \C^3/\Lambda$, so we get a strictly skew-symmetric bilinear form
\begin{align*}
    Q \colon V \times V \to \mathbb{F}_2
\end{align*}
induced by intersection of cycles, hence preserved by $M$. For any $v\in V$, we have that $v + k_0$ is effective if and only if $q(v) = 0$ (here $k_0$ is a fixed root of $K_C$).

\emph{Conclusion}: We have that $M \subseteq H$, where $H$ is the Steiner subgroup of $\text{AO}_6(\Z/2)$. This Steiner subgroup is the one preserving the zeros of $q$.

\textbf{Fact}: $H \cong O_6(\Z/2)$.

We want to prove

\begin{theorem} We have $M \cong H$.
\end{theorem}

As a remark, since $I_4$ is irreducible, we have that $M$ acts transitively, so it is enough to show that $\Stab_M(\ell_0) = O_6^-(\Z/2)$ for any bitangent $\ell_0$. Note this was the Galois group for 27 lines!

\subsection{Relation to 27 lines}

Let $S \subseteq \P^3$ be an (anti-canonically embedded) smooth cubic surface, and $p \in S$ not lying on any line. Then we can project away from $p$, and extend this to a blowup
\[ \begin{tikzcd}
    \til{S}\rar\dar & \P^2\\
    S\ar[ur,dashed] & \\
\end{tikzcd} \]
then $\til{\pi}_p \colon \til{S} \to \P^2$ is a double cover ramified over a smooth quartic $C$.

\begin{lemma} ..
\end{lemma}
\begin{proof} Let $B \subseteq \P^2$ be the branch locus of $\til{\pi}_p$. To get its degree, we take $\ell$ to be a generic line on the plane, so we get $\overline{\ell_p} =: H_p \subseteq \P^3$ intersects $S$ in a smooth curve $C_\ell$ and $H$ is nowhere tangent to $S$. The strict transform of $C_\ell$ is
\begin{align*}
    \til{\pi}_p \colon \til{C}_\ell \to \ell,
\end{align*}
which is 2:1. We know that $\deg C_\ell = 3$ smooth in $H$, hence $g(C_\ell) = 1 = g(\til{C}_\ell)$. Hence $\tilde{\pi}_p$ has 4 ramification points all with multiplicity 1. So $B$ has degree 4.

To get smoothness of $B$, we take some $q\in \P^2$ and $\ell \subseteq H$ generic so that $q\in \ell$. Then $H_\ell$ meets $S$ transversely,\footnote{The generic plane passing through $\overline{pq}$ is not tangent to any point of $\overline{pq}\cap S$, plus Bertini.} so by the degree computation, $\ell\cap B$ is four points, so $B$ is non-singular.
\end{proof}

\begin{remark} We have also shown that if $\ell \subseteq \P^2$ so that $\overline{p \ell} \cap S = C_\ell$ is a smooth curve, then $\# \ell\cap B = 4$, hence $\ell$ is not a bitangent.
\end{remark}

\begin{proof}[Lemma 2] Let $Q\in \til{B}$, where $\til{B}$ is the set-theoretic image $\til{\pi}_p(B)$. Let $q\in B$ be $\til{\pi}_p Q$, and let $\ell_0 = T_q B$. Then $\overline{pQ} \subseteq T_Q S$ (since $q\in B$) and $T_Q \til{B} \subseteq T_Q S$ (since $\til{B} \subseteq S$). And $H_{\ell_0} = T_Q S$.

Upshot: $\ell \subseteq \P^2$ is tangent to $q\in B$ iff $H_\ell$ is tangent to $S$ at $Q$, iff $C_\ell = H_\ell\cap S$ is singular at $Q$ (here $q = \til{\pi}_p(Q)$).

Let $L_1, \ldots, L_{27}$ be the lines on $S$, and $\ell_1, \ldots, \ell_{27}$ their images in $\P^2$ under $\pi_p$. Since the lines $L_i$ don't contain $p$, we have that $\ell_i$ is a line. Moreover, $\ell_i \ne \ell_j$ for $i\ne j$. This is because otherwise we would have $H_{\ell_i} = H_{\ell_j}$, so a sum of three lines contains $p$...

By construction for all $i$, we have $H_{\ell_i} = L_i + C_i$, where $C_i$ is a smooth conic curve, and $C_i \cap L_i = \left\{ Q_i,R_i \right\}$. If we let $q_i = \pi_p(Q_i)$ and $r_i = \pi_p(Q_i)$, then these are points of tangency of $\ell_i$ with $B$. It might be, however, that $q_i = r_i$.

If $q_i \ne r_i$, then $\ell_i$ is a \textit{bitangent}. If $q_i = r_i$, then $C_i$ is tangent to $L_i$ at $Q_i = R_i$ and any line $L \ne \overline{PQ_i}$ passing through $P$ intersects $C_i$ and $L$ at two distinct points, hence $\ell_i$ meets $B$ only at $q_i$. This is a hypeflex of $B$.

Let $D = T_p S \cap S$, which is a cubic curve with a singularity at $p$, and $\ell_{28} = \pi_p(D)$ (same as taking $\til{\pi}_p$ of the exceptional divisor). The intersection doesn't split, it is just a singular irreducible cubic. If we get a node, then $\ell_{28}$ is a bitangent. If we get a cusp, then $\ell_{28}$ is a hyperflex. And actually the converse holds as well.
\end{proof}

\subsection{}

We identified the set of bitangents with a sub\textit{set} $\Gamma \subseteq V \cong \mathbb{F}_2^6$. The results of this previous section yield an identification
\begin{align*}
    V\cong P(S, \Z/2)
\end{align*}
between $V$ and the primitive mod two cohomology in $H^2(S,\Z/2)$. This isomorphism respects the quadrati forms $q$ on $V$ and the intersection form on $P(S,\Z/2)$.

\begin{lemma}  $\Stab_M(\ell_0) = O_6^-(\Z/2)$.
\end{lemma}
\begin{proof}  Take any path $\gamma$ in $\P^{19}$ which preserves a fixed point $p$ on every surface $S_t$ appearing on the path, and $T_p S_t$ so that $\ell_{28} = \ell_0$ for every $t$. This is a path on $\P^{14}$ which is the space of plane quartics. The monodromy action of $\gamma$ fixes $\ell_{28}=\ell_0$ and the action on the lines $L_1, \ldots, L_{27}$ corresponds to the monodromy action of $\gamma$ on the remaining bitangents $\ell_{1}, \ldots, \ell_{27}$.
\end{proof}

\subsection{the Galois group}

We know that the Galois group is equal to $O_6(\Z/2)$ with quotient $O_6^-(\Z/2)$.

We know it's not solvable, since $O_6^-(\Z/2)$ isn't (it's isomorphic to $W(E_6)$ which has an index two subgroup $U_4(2)$ which is simple).

\begin{question} How many bitangents, in what position, do I need to solve for the other one?
\end{question}


Given three bitangents $\ell_0, \ell_1, \ell_2$ whose points of intersection lie on a common conic, we can construct the cubic surface $S$ associated to $C$ and $\ell_0$, so $\ell_1$ and $\ell_2$ correspond to intersecting lines. Then the corresponding group fixes the tangent plane $L_1 + L_2 + L_3$, so it sits inside a subgroup of $W(E_6)$ isomorphic to $(A_4 \times A_4) \rtimes (\Z/2)^3$, hence solvable.

Given four bitangents $\ell_0, \ldots, \ell_3$ whose points of contact do \textit{not} lie on a conic, construct $S$ and then $\ell_0$ becomes 

\section{Steiner's conics}

In 1848, Jakob Steiner computed the count of smooth conics tangent to five given conics on the plane as 7776. In 1864, Chasles provided the correct answer, 3264, the namesake of the famous contemporary text on enumerative geometry \cite{3264}. In unpublished work, Higman proved the Galois group is $2$-transitive, and Harris demonstrated it is the full symmetric group $\Sigma_{3264}$ \cite[\S~IV]{Harris-Galois}. We'll recount this story here.

\subsection{Steiner's argument and Chasles' correction}

We let $\P^5 \cong \P\Gamma(\O_{\P^2}(2))$ the moduli space of planar conic forms. For a given smooth conic $C$, we let
\begin{align*}
    \Tan_C \subseteq \P^5
\end{align*}
the locus of conics which are tangent to $C$.

\begin{proposition}\label{prop:tanC-irred-hypersurface-deg6} 
We have that $\Tan_C$ is an irreducible hypersurface of degree six (see e.g. \cite[p.~290]{3264})
\end{proposition}
\begin{proof} We consider the incidence correspondence
\[ \begin{tikzcd}
    {\left\{ (C',p) \in \P^5 \times C \mid m_p(C\cdot C')\ge 2\right\}}\rar["\pi_2"]\dar["\pi_1" left] & C\\
    \Tan_C. & 
\end{tikzcd} \]
Note that $\pi_2^{-1}(p)$ consists of all conics passing through $p$ and having tangent line equal to $T_pC$ at $p$. This is a hypersurface of degree three in $\P^5$.\footnote{Being tangent to a given line is a degree two condition, and fixing a point of tangency along the line adds an additional condition.} Therefore we claim $\Tan_C$ is also an irreducible hypersurface. To compute the degree of $\Tan_C$, we take a general pencil of conics and compute the intersection with $\Tan_C$. A pencil of conics is $|\O_{\P^1}(2)|$, and hence cuts out a linear series on $C$ of degree four. This gives us a map $C \to \P^1$ of degree four, and Riemann--Hurwitz implies it has six branch points.
\end{proof}

\begin{theorem}[{\cite[p.~189]{Steiner-conics}}]
Given five smooth conics $C_1, \ldots, C_5$, we have by B\'{e}zout's theorem that
\begin{align*}
    \deg(\Tan_{C_1} \cap \cdots \cap \Tan_{C_5}) = 6^5 = 7776.
\end{align*}
\end{theorem}

%
What is wrong with this argument? Consider the point $[1:0: \ldots : 0]\in \P^5$. It corresponds to the ``conic'' given by $x^2 = 0$. This is not a conic, but rather a doubled line. Nevertheless, it is tangent to any other conic at the two points of intersection, hence we have a continuous family of conics counted in the intersection $\cap_{i=1}^5 \Tan_{C_i}$. We should view this as an inevitable feature of the fact that we used a not-so-good moduli space of conics to do this problem.

Ernest de Jonquières worked on this problem in the early 1860's, but it was Luigi Cremona who is credited with observing this error that the doubled lines are counted amongst the solutions \cite[pp.~118-119]{Kleiman-Chasles}. A corrected computation was announced by Michel Chasles\footnote{%
Apparently (and thankfully) Michel Chasles' infamous gullibility did not extend to his work in enumerative algebraic geometry, as he was able to correctly surmise the flaws in Jakob Steiner's work. Unfortunately, he was the victim of a prolonged scam in which he purchased forged letters written by Vrain-Denis Lucas, among them a love letter from Cleopatra to Julius Caesar written in French, and a falsified correspondence between Blaise Pascal and Isaac Newton validating Chasles' conspiracy theory that Newton had stolen his theory of gravity from Pascal \cite[VI.4]{Farquhar-deception}.%
}
in 1864 \cite{Chasles-conics}, which was part of the work which won him the Copley medal.

\begin{theorem}[{\cite{Chasles-conics}}] The number of smooth conics tangent to five is 3264.
\end{theorem}
\begin{proof} We follow the exposition in \cite{Kleiman-Chasles}, with some modern terminology. Given a family of conics, we let $\mu$ denote the number of conics passing through a given point, and dually $\nu$ the number tangent to a given line. The general idea (in Chasles theory of \textit{characteristics}) is that a condition on a family of conics, for instance being tangent to a fixed conic, can be expressed as a linear combination $\alpha\mu + \beta\nu$. We let $\mu^r \nu^s$ denote the number of conics passing through $r$ points and tangent to $s$ lines, and it can be easily computed that $\mu^{r} \nu^{5-r} = 2^{\min\{r,5-r\}}$ (cf.~\cite[p.~42]{Katz-book}).

Let's let $\alpha\mu + \beta\nu$ be the condition of tangency to a given conic, for some $\alpha,\beta\in\Z$. We want to solve for $\alpha$ and $\beta$, which in more contemporary language is solving for the degree of the hypersurface describing tangency (and we've seen that this is six). To do this, we note that a pencil of conics is describable as the conics passing through $4$ fixed points, hence the condition $\mu^4$. We then solve for
\begin{align*}
    6 = \mu^4 (\alpha\mu + \beta \nu) = \alpha + 2\beta.
\end{align*}
Since this problem is symmetric in pole-polar duality, we get that $\alpha = \beta = 2$. Hence the condition of being tangent to a fixed conic is precisely $2\mu + 2\nu$. To solve the problem we get
\begin{align*}
    (2\mu + 2\nu)^5 &= 2^5 \left( \mu^5 + 2\binom{5}{1} + 4\binom{5}{2} + 4\binom{5}{3} + 2\binom{5}{4} + 1  \right) \\
    &= 32(102) \\
    &= 3264.
\end{align*}
\end{proof}

\begin{remark} Let $X \subseteq \P^5 \times (\P^5)^\ast$,
be the closure of the space of those $(C,C^\ast)$ for which $C$ is a smooth conic. This is called the moduli of \textit{complete conics}. We can let $\mu$ and $\nu$ denote the pullback of the hyperplane class along the projections to $\P^5$ and $(\P^5)^\ast$, respectively. From this perspective, Chasles' computation can be regarded as occurring in the Chow ring $\CH^\ast(X)$ \cite[Chapter~8]{3264}.
\end{remark}



\begin{remark} There is some evidence that Chasles' work indirectly (through the work of Eugène Prouhet) inspired Schubert's calculus of conditions, which later became Hilbert's 15th problem \cite[p.~121]{Kleiman-Chasles}.
\end{remark}

A contemporary perspective on Chasles computation is given in terms of residual intersection, following Fulton and MacPherson. We refer the reader to \cite[13.5.5]{3264} for this story.


\subsection{Setting up the monodromy problem}

We let $\P^5 \cong \P\Gamma(\O_{\P^2}(2))$ be the moduli space of planar conics. The squaring map
\begin{align*}
    \O(1) \xto{\Delta} \O(1) \otimes \O(1) \xto{\mu} \O(2)
\end{align*}
induces a locus $W_1 \subseteq \P^5$ of doubled lines. We consider the incidence correspondence
\begin{align*}
    Y := \left\{ (C_1, \ldots, C_5;C) \colon C \text{ tangent to each }C_i \right\}\subseteq (\P^5)^{\times 5} \times (\P^5 \minus W_1).
\end{align*}
This comes with projection maps
\[ \begin{tikzcd}
    Y\dar["\pi_1" left]\rar["\pi_2" above] & \P^5\minus W_1\\
    (\P^5)^{\times 5} &
\end{tikzcd} \]
with fiber $\pi_1^{-1}(C_1,C_2, \ldots, C_5) = \Tan_{C_1}\cap \cdots \cap \Tan_{C_5} \subseteq \P^5 \minus W_1$. We let $X := (\P^5)^{\times5}$ and we want to study the Galois group $G$ of $\pi_1 \colon Y \to X$.

% \begin{center}
%     \begin{tabular}{l | l}
%     $W$ & $\P^5$ \\
%     $\til{W}$ & $\P^5 \minus \mathit{doubled lines}$ \\
%     $H_C$ & $\Tan_C$ \\
%     $X_C$ & $\Bitan_C$ \\
%     $Y_C$ & $\Flextan_C$
%     \end{tabular}
% \end{center}

\begin{notation} 
For any conic $C$ which is not a doubled line $C \in \til{W}$, we let $\Bitan_C \subseteq \Tan_C \subseteq \P^5$ be the locus of conics bitangent to $C$. We let $\Flextan_C \subseteq \Tan_C$ be the locus of conics with a flextangent to  $C$ (meaning a point of intersection of order $\ge 3$).
\end{notation}

\begin{proposition}[{\cite[p.~722]{Harris-Galois}}] 
\label{prop:irreducibility-conic-loci}
Let $C$ be a smooth conic.
\begin{enumerate}
    \item $\Tan_C$, $\Bitan_C$, and $\Flextan_C$ are irreducible.
    \item If $D$ is any other conic, not tangent to $C$, then $\Tan_C \cap \Tan_D$ and $\Tan_C \cap \Flextan_D$ are irreducible.
\end{enumerate}
\end{proposition}
\begin{proof}
todo
\end{proof}

\begin{proposition} The Galois group $G = \Gal(\pi_1)$ is transitive.
\end{proposition}
\begin{proof} Consider the projection
\begin{align*}
    \pi_2 \colon Y \to \P^5 \minus W_1.
\end{align*}
Over a smooth conic $C \in \P^5 \minus W_1$, the fiber $\pi_2^{-1}(C) = \Tan_C^5$. Since $\Tan_C$ is irreducible of dimension four by \Cref{prop:irreducibility-conic-loci}, we have that the fiber $\pi_2^{-1}(C)$ is irreducible of dimension 20. Suppose that $C$ is a conic of rank two\footnote{I assume this means a line pair?}, then the fibers $\pi_2^{-1}(C)$ are reducible but still of rank 20.

Altogether, we claim $Y$ can only have one irreducible component $\bar{Y}$, which is necessarily of dimension 25. All the points of $\Gamma$ (where $\Gamma$ is a generic fiber of $\pi_1$) necessarily lie on $\bar{Y}$. Hence $G$ is transitive.
\end{proof}

\begin{proposition} The Galois group $G$ is $2$-transitive.
\end{proposition}
\begin{proof} Fix a conic $C$, we want to see that $\Stab_G(C)$ is transitive. To that end, we introduce the notation\footnote{The notation $Y_C$ is overloaded in \cite[\S~IV]{Harris-Galois}} $Y_C$ for the incidence correspondence
\begin{align*}
    Y_C := \left\{ (C_1, \ldots, C_5 ; C') \colon C' \in \Tan_{C_i} \text{ for each }i \right\} \subseteq \Tan_C^{\times 5} \times (\P^5 \minus \left\{ W_1, C\right\}),
\end{align*}
with projections
\[ \begin{tikzcd}
    Y_C\dar["\pi_1" left]\rar["\pi_2" above] & {\P^5\minus\{W_1,C\}}\\
    \Tan_C^{\times 5} & 
\end{tikzcd} \]
It is clear that $\Stab_G(C)$ is exactly the Galois group of this new $\pi_1$, so we want to argue that it is transitive. Again we do this by reference to $\pi_2$.

Let $U \subseteq \P^5 \minus (W_1 \cup \left\{ C \right\})$ be the Zariski open locus of conics transverse to $C$. For any $D\in U$, we have that
\begin{align*}
    \pi_2^{-1}(D) = (\Tan_C \cap \Tan_D)^{\times 5},
\end{align*}
which by \Cref{prop:irreducibility-conic-loci} is irreducible of dimension 15. Therefore $Y_C$ has at most one irreducible component dominating $\Tan_C^{\times5}$.
\end{proof}

\begin{proposition} Fix five conics $C_1, \ldots, C_5$ with no two tangent, no three concurrent\footnote{Concurrent here means as points in $\P^5$. This can be rephrased to say that no three of them lie in a pencil.} and no three tangent to any line. Then the intersection $\Tan_{C_1} \cap \cdots \cap \Tan_{C_5} \cap (\P^5 \minus W_1)$ consists of isolated points with
\begin{align*}
    \sum_{C \in \Tan_{C_1} \cap \cdots \cap \Tan_{C_5}\cap \P^5\minus W_1} m_C(\Tan_{C_1} \cap \cdots \cap \Tan_{C_5}) = 3264.
\end{align*}
Moreover the local intersection multiplicity at some $C$ is given by
\begin{align*}
    m_C \left( \bigcap_{i=1}^5 \Tan_{C_i}\right) = \prod_i \left( \sum_{p\in C\cap C_i} m_p(C \cdot C_i)-1 \right).
\end{align*}
\end{proposition}

Note that at a generic such point, $C$ is tangent to each $C_i$ at a single point $p$, with two other transverse points $q_1, q_2$ of intersection, so that the term in the product at $C_i$ is exactly one, hence the conic $C$ contributes a $+1$ to the overall count.

To exhibit a simple transposition in the Galois group, we want to find an arrangement where we have 3263 smooth conics tangent to our given $C_i$'s. Equivalently, we can find a $C$ whose local intersection multiplicity is two, and all the other conics have local intersection multiplicity one.

\begin{lemma} The Galois group of the five conics problem contains a simple transposition.
\end{lemma}
\begin{proof} Pick a conic $C$, and select the $C_i$'s to be a generic point
\begin{align*}
    (C_1, \ldots, C_5) \in \Tan_C^{\times 4} \times \Flextan_C,
\end{align*}
so that $C_1, \ldots, C_4$ are generically tangent to $C$ and $C_5$ has a flextangent with $C$. Since the $C_i$'s are chosen suitably generically, they lie in the scope of the above proposition, and we have that
\begin{align*}
    m_C(\Tan_{C_1} \cap \cdots \cap \Tan_{C_5}) = 2.
\end{align*}
It remains to argue that all the other $C' \in \Tan_{C_1} \cap \cdots \cap \Tan_{C_5}$ have multiplicity one. We let
\begin{align*}
    K_C = \left\{ (C_1, \ldots, C_5; C') \mid C' \in \Tan_{C_i}\forall i \right\} \subseteq (\Tan_C^{\times 4} \times \Flextan_C) \times (\P^5 \minus W_1 \cup \{C\})
\end{align*}
We consider
\begin{align*}
    K_C^{\ge 2} \subseteq K_C
\end{align*}
the closed subvariety of all those tuples with $m_{C'}(\Tan_{C_1}, \ldots, \Tan_{C_5})$. Now consider the diagram
\[ \begin{tikzcd}
    K_C^{\ge2}\rar[hook]\ar[dr] & K_C\dar["\pi_1"]\rar["\pi_2"] & \P^5\minus W_1 \cup \{C\}\\
     & \Tan_C^{\times 4}\times\Flextan_C.
\end{tikzcd} \]
We want to argue that $K_C^{\ge 2} \to \Tan_C^{\times 4}\times\Flextan_C$ is not dominant.\footnote{todo - why?} Note that
\begin{align*}
    \pi_2^{-1}(C') = (\Tan_C \cap \Tan_{C'})^{\times4} \times (\Tan_C \cap \Flextan_{C'}) \subseteq K_C.
\end{align*}
This is irreducible of dimension 14 if $C'$ is smooth and transverse to $C$, it is reducible of dimension 14 if $C' \notin \Flextan_C$ and dimension 15 if $C' \in \Flextan_C$. Hence $K_C$ can have only one irreducible component $\overline{K_C}$ dominating $\Tan_C^{\times 4}\times\Flextan_C$ (since we observe that $\dim (\Tan_C^{\times 4}\times\Flextan_C) =19$). We claim that $\overline{K_C} \not\subseteq K_C^{\ge 2}$, since if $C'$ is smooth and not in $\Tan_C$, then a generic element $(D_1, \ldots, D_5;C') \in \pi_2^{-1}(C')$ has that $D_1, \ldots, D_5$ are all tangent to $C'$.

Finally, we want to argue that for a generic $(C_1, \ldots, C_5) \in \Tan_C^{\times 4}\times\Flextan_C$, we have that $Y$ is irreducible at $(C_1, \ldots, C_5;C)$. Note though that for a general $D\in \P^5\minus W_1$, the fiber $\pi_2^{-1}(D)$ is irreducible at all points $(D_1, \ldots, D_5;D)$ so that $D_i \notin \Bitan_D$. But the condition that $D_i \in \Bitan_D$ isn't open -- in other words no conic in a small neighborhood of $C$ will be bitangent to a conic in a small neighborhood of $C_i$.
\end{proof}

We have therefore proven:

\begin{theorem} The Galois group of the 3264 conics is $\Sigma_{3264}$.
\end{theorem}


%%%%%%%%%%%%%%%%%%%%%%%%%%%%%%%%%%%%%%%%%%%%%%%
\appendix

\section{Algebraic geometry terms and references}

Here are some terms and results, laid out in the order we need to use them throughout the notes.

\begin{definition}[{\cite[p.~91]{Hartshorne}}]
\label{def:generically-finite}
If $f \colon X \to Y$ is a morphism of varieties with $Y$ irreducible, we say it is \textit{generically finite} if $f^{-1}(\eta)$ is a finite set, where $\eta$ is the generic point in $Y$.
\end{definition}

\begin{remark} If $f$ is locally of finite type and qcqs, being generically finite admits some equivalent conditions (see e.g. \cite[02NW]{Stacks}).
\end{remark}

\begin{proposition}\label{prop:irred-criterion}
Let $f\colon X\to Y$ is a map with irreducible equidimensional fibers, with $Y$ irreducible. Suppose that either $X$ is equidimensional or $f$ is proper. Then $X$ is irreducible.
\end{proposition}
\begin{proof} A proof can be found in \url{https://public.websites.umich.edu/~mmustata/Note1_09.pdf}
\end{proof}




\printbibliography
\end{document}
